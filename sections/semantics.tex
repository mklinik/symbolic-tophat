% !TEX root=../main.tex


\section{Semantics}
\label{sec:semantics}



\subsection{Observations}

\todo{Extend for read-only editors.}

\begin{center}
  \usemacro{O-Value}
\end{center}

\begin{center}
  \usemacro{O-Failing}
\end{center}

% \usemacro{O-Inputs}

\fixme{Are these still correct?}



\subsection{Evaluation}

\begin{gather*}
  \boxed{\RelationE} \Break
  \userule{Sym-E-Value}\Quad
  \userule{Sym-E-Pair} \Break
  \userule{Sym-E-App} \Quad
  \userule{Sym-E-IfTrue} \Break
  \userule{Sym-E-IfFalse} \Break
  \userule{Sym-E-Ref} \Quad
  \userule{Sym-E-Deref} \Break
  \userule{Sym-E-Assign} \Quad
  \userule{Sym-E-Edit} \Break
  \userule{Sym-E-Enter}\Quad
  \userule{Sym-E-Update}\Break
  \userule{Sym-E-Then}\Quad
  \userule{Sym-E-Next}\Break
  \userule{Sym-E-And}\Break
  \userule{Sym-E-Or} \Break
  \userule{Sym-E-Xor}\Quad
  \userule{Sym-E-Fail}
\end{gather*}



\subsection{Normalisation}

\begin{gather*}
  \boxed{\RelationS} \Break
  \userule{Sym-S-ThenStay} \Break
  \userule{Sym-S-ThenFail} \Break
  \userule{Sym-S-ThenCont} \Break
  \userule{Sym-S-OrLeft} \Break
  \userule{Sym-S-OrRight} \Break
  \userule{Sym-S-OrNone} \Break
  \userule{Sym-S-Edit} \Quad
  \userule{Sym-S-Fill} \Break
  \userule{Sym-S-Update} \Quad
  \userule{Sym-S-Fail} \Break
  \userule{Sym-S-Xor} \Quad
  \userule{Sym-S-Next} \Break
  \userule{Sym-S-And}
\end{gather*}


\begin{gather*}
  \boxed{\RelationN} \Break
  \userule{Sym-N-Done} \Break
  \userule{Sym-N-Repeat}
\end{gather*}



\subsection{Handling}

\begin{gather*}
  \boxed{\RelationH} \Break
  \userule{Sym-H-Change} \Break
  \userule{Sym-H-Fill} \Quad
  \userule{Sym-H-Update}\Break
  \userule{Sym-H-Next} \Break
  \userule{Sym-H-PassThen} \Quad
  \userule{Sym-H-PassNext} \Break
  \userule{Sym-H-PickLeft} \Quad
  \userule{Sym-H-PickRight}\Break
  \userule{Sym-H-FirstAnd} \Quad
  \userule{Sym-H-SecondAnd} \Break
  \userule{Sym-H-FirstOr} \Quad
  \userule{Sym-H-SecondOr}
\end{gather*}

Note that \refrule{Sym-H-Empty} is omitted as it does nothing useful for symbolic execution.


\begin{gather*}
  \boxed{\RelationI} \Break
  \userule{Sym-I-Handle}
\end{gather*}


\subsection{Driving}

This section describes the top level symbolic execution function.

It uses the function drive, which takes $t,I,\sigma,\phi$, and is recursively called to produce a list of end states an predicates.
The drive function makes use of triple arrow, a small step semantics that generates all possible steps that can be taken.
Drive then filters this based on the following:

\begin{align*}
  \Value(t,\sigma) = v \wedge \text{SAT } \phi & \quad\checkmark\\
  \neg\text{SAT }\phi                          & \quad\times \\
  t_i=t_j \wedge \phi_i=\phi_j                 & \quad\times \\
  \text{otherwise}                             & \quad\circlearrowleft
\end{align*}

\begin{function}
  \signature{symEx :: \Task \rightarrow [(\Task,[\mathrm{Inputs}],\mathrm{Predicate})]} \\

  symEx t = drive\ t\ []\ \emptyset\ \True\\
\end{function}


\begin{function}
  \signature{drive :: \Task \times [\mathrm{Inputs}] \times \mathrm{State} \times \mathrm{Predicate} \rightarrow [(\Task,[\mathrm{Inputs}],\mathrm{Predicate})]} \\
  drive\ t\ I\ \sigma \ \phi  = map \ (concat . drive') (\Rrightarrow\ t\ I\ \sigma \ \phi)\\
                                  where\ drive'\ t'\ I\ \sigma'\ \phi' \mid \neg \text{SAT } \phi' = [ ]\\
                                                                       \mid \Value(t',\sigma') = v \wedge \text{SAT } \phi' = [(t',I',\phi')]\\
                                                                       \mid t' = t \wedge \phi' = \phi = []\\
                                                                       \mid otherwise = drive\ t'\ I'\ \sigma'\
                                                                        \phi'
\end{function}

In order to show that this actually holds, we need to prove some properties, see next section.
