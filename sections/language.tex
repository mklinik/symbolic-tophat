% !TEX root=../main.tex


\section{Calculus}
\label{sec:language}

Differences with the normal grammar are discussed in each subsection.


\subsection{Expressions}

\begin{figure}
\usemacro{G-Language}
\label{language}
\caption{Syntax of the host language}
\end{figure}

\begin{itemize}
  \item
    Symbols $s$ can be part of any expression.
    As locations $l$, they are not intended to be used by programmers,
    they are generated by the semantics.
    Symbols are intuitively a read input from the end user.
  \item
    Unary and binary operations are made explicit.
\end{itemize}

\usemacro{G-Pretasks}

\subsection{Types}

\usemacro{G-Types}

(Nothing changes here.)

\userule{S-Var}

\subsection{Values}

\usemacro{G-Values}

\begin{itemize}
  \item
    Symbols $s$ are now values,
    as are unary and binary operations containing symbols.
    Operations on normal values are normally evaluated.
    Note that pairs of symbols are allowed!
    This could be extended to lists of symbols in the future.
\end{itemize}

\usemacro{G-Tasks}

\begin{itemize}
  \item Read-only editors are added in the obvious way.
\end{itemize}



\subsection{Inputs}

\usemacro{G-Inputs}

\begin{itemize}
  \item Editors cannot be set to a concrete value $v$, but only to a symbol $s$.
\end{itemize}



\subsection{Predicates}

\usemacro{G-Predicates}

First stab at predicates.
\begin{itemize}
  \item Similar to values but should not contain other functions except for the operations \SMT solvers know about!
  \item \emph{Markus says}: Everything that can occur in the condition of an if-then-else ends up in path constraints, and will be passed to the SMT solver eventually.
  \item Should contain constants $c$.
  \item Should contain symbols $s$ and operations $u$ and $o$ on them.
\end{itemize}
