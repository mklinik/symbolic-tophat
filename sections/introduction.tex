% !TEX root=../main.tex

\section{Introduction}

The Task Oriented Programming paradigm (\TOP) is a useful abstraction over workflow programs.
The idea of \TOP is to describe what work needs to be done, in what order, by which person.
Then, from this specification, an application can be generated.
The \ITASKS framework~\cite{DBLP:conf/ppdp/PlasmeijerLMAK12} is an implementation of the paradigm in the functional programming language Clean.
The many applications developed in the iTasks framework clearly demonstrate the usefullness and need for this new programming paradigm.
It is being used by the Netherlands Royal Navy~\cite{jansen2018dynamic}, Dutch Tax Office~\cite{conf/sfp/StutterheimAP17} and the Dutch Coastal Guard~\cite{lijnse2012incidone}. % of moeten we hier verwijzen naar Consequence Management – Declarative Modelling of Maritime C2-systems
Furthermore, it has potential for application in domains like health-care and IoT~\cite{DBLP:conf/cgo/KoopmanLP18}.

Applications used in these type of domains and by these kind of organisations are often critical, where programming mistakes can have consequences.
% Currently, iTasks programs are verified by running manually written test-cases. % do we have a source?
% It is evident that testing alone is not sufficient.
In order to verify that a particular \TOPHAT program behaves as intended,
we would like to show that certain properties on the of a task hold.
A common way to do this is to write test cases manually,
or to generate random input and verify that the outcome fulfils the desired property.
Writing tests manually is labour intensive and cumbersome.
To test graphical and interactive applications one needs testers to click through the application,
maybe making use of a way to register and replay user events.
With random testing, there is no guarantee that all paths will be covered.

To overcome these issues, we apply symbolic execution to prove that the desired properties hold.
Instead of executing tasks with test input, or letting an user interactively test the application,
we run tasks on symbolic input.
This input consists of symbolic tokens that are allowed to be any value of the desired type.
When a task branches, we record what path conditions must hold over the symbolic input.

By means of symbolic execution, we can guarantee that certain conditions over the outcome of a \TOP program hold.
Since iTasks does not have a formal semantics, we instead apply symbolic execution to \TOPHAT~\cite{Steenvoorden2019},
by systematically changing the semantic rules of the original language.



\subsection{Contributions}

In this paper we propose a symbolic approach to verification instead of specifying or carrying out manual tests.
Our contributions to verification of interactive programs is as follows:

\begin{itemize}
  \item Provide symbolic execution semantics for \TOPHAT, a workflow specification language in spirit of \TOP.
  \item Prove soundness and completeness of the symbolic semantics with respect to the original \TOPHAT semantics.
  \item Release an implementation of the symbolic execution semantics in Haskell.
\end{itemize}



\subsection{Structure}

Section~\ref{sec:examples} will give a brief overview of \TOPHAT and introduce two examples to demonstrate the goal of our analysis.
Section~\ref{sec:intuition} describes the intuition behind symbolic execution and how it is applied to \TOPHAT.
Then in Section~\ref{sec:language}, the \TOPHAT language is defined.
Section~\ref{sec:semantics} goes on to define the formal semantics of the symbolic execution.
In Section~\ref{sec:properties}, soundness and completeness is shown for the symbolic execution semantics with respect to the original \TOPHAT semantics.
Then in Section~\ref{sec:relatedwork} related work is discussed, and Section~\ref{sec:conclusion} concludes.
