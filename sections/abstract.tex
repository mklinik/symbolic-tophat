% !TEX root=../main.tex

% Context
Task-Oriented Programming (TOP) is a programming paradigm that allows declarative specification of workflows.
TOP is typically used in domains where functional correctness is essential, and where failure can have financial or strategical consequences.
% Inquiry
In this paper we aim to make formal verification of software written in TOP easier.
Currently, only testing is used to verify that programs behave as intended.
% Approach
We use symbolic execution to guarantee that no aberrant behaviour can occur.
In previous work we presented TopHat, a formal language that implements the core aspects of TOP.
In this paper we develop a symbolic execution semantics for TopHat.
% Knowledge
Symbolic execution allows to prove that a given property holds for all possible execution paths of TopHat programs.

% Grounding
We show that the symbolic execution semantics is consistent with the original TopHat semantics, by proving soundness and completeness.
We present an implementation of the symbolic execution semantics in Haskell.
By running example programs, we validate our approach.
% Importance
This work represents a step forward in the formal verification of TOP software.


% Context: What is the broad context of the work? What is the importance of the general research area?
% Inquiry: What problem or question does the paper address? How has this problem or question been addressed by others (if at all)?
% Approach: What was done that unveiled new knowledge?
% Knowledge: What new facts were uncovered? If the research was not results oriented, what new capabilities are enabled by the work?
% Grounding: What argument, feasibility proof, artifacts, or results and evaluation support this work?
% Importance: Why does this work matter?
