% !TEX root=../main.tex



\section{Properties}
\label{sec:properties}

In this section we describe what it means for the symbolic execution semantics to be correct.
We prove it sound and complete with respect to the concrete semantics of \TOPHAT.
% Furthermore, we demonstrate that the pruning performed in Section~\ref{subsec:driving} is safe,
% and only removes branches that are either unfeasible, or infinite.

% \subsection{Removal of Not SAT}
%
% Subsection~\ref{subsec:driving} describes two criteria that are used to prune away branches of the symbolic execution.
% The first removes branches that result in a predicate that is not satisfiable.
% As stated in Definition~\ref{def:Sat}, this means that there is no substitution for the symbols in the predicate,
% that will make the predicate true.
%
% In other words, these symbolic branches have no counterpart in the concrete semantics, and are therefore invalid.
% This property is described in the following theorem.
%
% \begin{theorem}[Not sat is safe to remove]
%   \label{thm:notSat}
%
% For all tasks $t$ and states $\sigma$ such that
% $t,\sigma\drive{}\overline{t',\sigma',i,\phi}$\quad
% $t',\sigma'\drive{}\overline{t'',\sigma'',i',\phi'}$
%
% we have that $\neg\Sat(\phi\land\phi')$ implies
% that there is no mapping $M=[s_0\mapsto c_0,\cdots,s_n\mapsto c_n]$ such that
% $t,M\sigma\xrightarrow[]{Mi}\bar{t'},\bar{\sigma'}\xrightarrow[]{Mi'}\bar{t''},\bar{\sigma''}$
%
% \end{theorem}
%
% The proof for this theorem is listed in the appendix.
%
% It is important to note that this holds for all future states of the unsatisfiable symbolic executions too.
% This can be concluded from the way in which the predicates are constructed, new predicates are put in conjunction with the old one.
%
% %%%%%%%%% STUCK properties  %%%%%%%%%%%%%%%%%%%%%%%%%%%%%%%%
%
% \subsection{When are branches stuck?}
%
% The second pruning criteria deals with terms that allow for infinite input streams,
% but that will not result in a value.
%
% Observe the following small program. $\Edit 1 \Next \lambda x . \Edit x + 1$.
%
% The symbolic execution will tell us that we can do one of two things;
% send a new symbolic value to the editor, or we can send $\Continue$,
% to proceed to the right hand side of the expression.
%
% When we elect the first, sending a new symbolic value, we can then again send yet another
% symbolic value to the editor. And then again, and again and again.
%
% No extra information is gained, $s_0$ and $s_1$ are erased from the program, and are not contained in the state or
% predicate.
%
% \begin{lemma}[Stuck really is stuck]
% For all tasks $t$ and states $\sigma$ such that $t,\sigma \handle{i} t',\sigma',\phi$,
% then $t=t'$ and $\phi=\phi'$ implies that there is no step $t',\sigma'\handle{i'} t'',\sigma'',\phi'$ such that either $t'\neq t''$ or $\phi\neq\phi'$.
% \label{lemma:stuck}
% \end{lemma}
% \fixme{$\phi'$ is not in scope on the left side of the implication.}
% \\
% \fixme{this cannot be proven untill the drive function is fixed}

In order to relate the two semantics, we use the concrete inputs listed in \cref{fig:inputsConcrete}.

\begin{figure}[h]
  \usemacro{G-CInputs-Compact}
  \caption{Syntax of concrete inputs.}
  \label{fig:inputsConcrete}
\end{figure}


\subsection{Soundness}
\label{sec:soundess}




%For completeness and soundness we define what it means to completely evaluate a task, given a list of concrete inputs.

% \begin{function}
%   \signature{\mathit{eval} :: \Task \times [\textrm{Concrete inputs}] \times \mathrm{State} \rightarrow \Task \times \mathrm{State}} \\
%   \mathit{eval}\ t\ (j:js)\ \sigma = \begin{array}{ll}
%                               (t',\sigma')      & \Value(t',\sigma') \equiv v \\
%                               \mathit{eval}\ t'\ \sigma' & \Value(t',\sigma') \equiv \bot
%                                   \end{array}
%                               \textrm{with } t,s\xrightarrow[]{j} t',\sigma'
% \end{function}

% \begin{theorem}[Soundness of symbolic execution]
% \label{thm:sound}
%
% For all tasks $t$, mappings $M=[s_0\mapsto c_0,\cdots,s_n\mapsto c_n]$,
% for all elements $(t',I,\sigma,\phi)\in \mathit{drive}\ t\ [\ ]\ \emptyset\ \True$,
% $M \phi\equiv\True$ implies
% $eval\ t\ (M I)\ \emptyset=(t_n,\sigma_n)$,  $t_n=M t'$ and $\sigma_n=M \sigma$.
% \end{theorem}
% \fixme{this cannot be proven untill the drive function is fixed}

In order to validate the symbolic execution semantics, we want to show that for every symbolic execution there exists a corresponding concrete one.
This soundness property is expressed by \cref{thm:sounddrive}.

\begin{theorem}[Soundness of driving]
  \label{thm:sounddrive}

  For all tasks $t$, states $\sigma$ and mappings $M=[s_0\mapsto c_0,\cdots,s_n\mapsto c_n]$,
  such that $t,\sigma\drive{} \overline{t',\sigma',i,\phi}$
  we have $M\phi$ implies
  $t,\sigma M \xRightarrow[]{M i} t'',\sigma''$, $t'M \equiv t''$ and $\sigma' M \equiv \sigma''$.
\end{theorem}

The proof for this lemma is rather straightforward.
Since the driving semantics makes use of the handling and the normalisation semantics, we require two lemmas.
One showing that the handling semantics is sound, \cref{lem:soundhandle}, and one showing that the normalisation semantics is sound, \cref{lem:soundnorm}.

\begin{lemma}[Soundness of handling]
  \label{lem:soundhandle}

  For all tasks $t$, states $\sigma$ and mappings $M = [s_0\mapsto c_0,\cdots,s_n\mapsto c_n]$,
  such that $t,\sigma\handle{i} \overline{t',\sigma',\phi}$,
  we have $M\phi$ implies
  $t,M \sigma \xrightarrow[]{M i} t'',\sigma''$, $t'M \equiv t'' $ and $\sigma' M \equiv \sigma''$
\end{lemma}

\Cref{lem:soundhandle} is proven by induction over $t$.
The full proof is listed in \cref{sec:soundness-proofs}.

\begin{lemma}[Soundness of normalisation]
  \label{lem:soundnorm}

  For all expressions $e$, states $\sigma$ and mappings $M=[s_0\mapsto c_0,\cdots,s_n\mapsto c_n]$,
  such that $e,\sigma\normalise \overline{t,\sigma',\phi}$,
  we have $M\phi$ implies
  $e,M \sigma \bar{\normalise}t',\sigma''$, $t M \equiv t'$ and $\sigma' M \equiv \sigma''$.

\end{lemma}

Since \cref{lem:soundnorm} makes use of both the striding and the evaluation semantics,
we must show soundness for those too.

\begin{lemma}[Soundness of striding]
  \label{lem:soundstride}

  For all tasks $t$, states $\sigma$ and mappings $M=[s_0\mapsto c_0,\cdots,s_n\mapsto c_n]$,
  such that $t,\sigma\stride \overline{t',\sigma',\phi}$,
  $M \phi= \True$ implies
  $t,M\sigma \bar{\stride}\bar{t'},\bar{\sigma'}$, $M t'\ \equiv \bar{t'} \land M\sigma' \equiv \bar{\sigma'}$.

\end{lemma}

\begin{lemma}[Soundness of evaluation]
  \label{lem:soundeval}

  For all expressions $e$, states $\sigma$ and mappings $M=[s_0\mapsto c_0,\cdots,s_n\mapsto c_n]$,
  such that $e,\sigma\eval \overline{v,\sigma',\phi}$,
  we have that $M\phi \implies t,M\sigma \bar{\eval}\bar{v},\bar{\sigma'} \land Mv \equiv \bar{v} \wedge M\sigma' \equiv \bar{\sigma'}$.

\end{lemma}

The full proofs of \cref{lem:soundnorm,lem:soundstride,lem:soundeval} are listed in the appendix.





\subsection{Completeness}

We also want to show that for every concrete execution, a symbolic one exists.

In order to state this Theorem, we require a simulation relation $i\sim j$, which means that the symbolic input $i$ follows the same direction as the concrete input $j$.
This relation is defined below.

\begin{definition}[Input simulation]
  A symbolic input $i$ simulates a concrete input $j$ denoted as $i\sim j$ in the following cases.\\
  $s\sim a$, where $s$ is a symbol and $a$ a concrete action.\\
  $i\sim j\implies \First i \sim \First j$\\
  $i\sim j\implies \Second i \sim \Second j$
\end{definition}

This allows us to define the completeness property as listed in \cref{thm:completeDrive}.

% \begin{theorem}[Completeness of symbolic execution]
% For all tasks $t$ and input lists $[j_0,\cdots,j_n]$ such that $\mathit{eval}\ t\ [j_0,\cdots,j_n]\ \emptyset = (t',\sigma')$,
% then there exists an element $(t'',I,\sigma'',\phi'')\in \mathit{drive}\ t\ [\ ]\ \emptyset\ \True$ and mapping $[s_0\mapsto c_0,\cdots,s_n\mapsto c_n]$ such that
% $I[s_0\mapsto c_0,\cdots,s_n\mapsto c_n]=[j_0,\cdots,j_n]$ and $t'=t''[s_0\mapsto c_0,\cdots,s_n\mapsto c_n]$.
%   \label{thm:complete}
% \end{theorem}
%
% \fixme{Proof of thm 5.2 is required for this thm}



\begin{theorem}[Completeness of driving]
  \label{thm:completeDrive}
  For all $t,\sigma,j$ such that $t,\sigma \xRightarrow[]{j} \hat{t'},\hat{\sigma'}$
  there exists an $i\sim j$ such that $t,\sigma''\drive{} t'',\sigma''',i,\phi$.
\end{theorem}


The proof of \cref{thm:completeDrive} is rather simple.
We show that handling is complete (\cref{lem:completeHandle})
and that the subsequent normalisation is complete (\cref{lem:completeNormalise}).


\begin{lemma}[Completeness of handling]
  \label{lem:completeHandle}
  For all $t,\sigma,j$ such that $t,\sigma \xrightarrow[]{j} \hat{t'},\hat{\sigma'}$
  there exists an $i\sim j$ such that $t,\sigma''\handle{i} t'',\sigma''',\phi$.
\end{lemma}

\Cref{lem:completeHandle} is proved by induction over $t$.
We only need to show that every concrete execution is also a symbolic one.
The only change needed to convert from concrete to symbolic is the adaption of the input.

Since handling makes use of normalisation, striding, and evaluation, we need to prove that they too are complete.
These properties are listed in \cref{lem:completeNormalise,lem:completeStride,lem:completeEval}

\begin{lemma}[Completeness of normalisation]
  \label{lem:completeNormalise}
  For all $e,\sigma$ such that $e,\sigma \bar{\normalise}\hat{t'},\hat{\sigma'}$
  there exists a symbolic execution $t,\sigma''\normalise t'',\sigma''',\phi$.
\end{lemma}


\begin{lemma}[Completeness of striding]
  \label{lem:completeStride}
  For all $t,\sigma$ such that $t,\sigma \bar{\stride}\hat{t'},\hat{\sigma'}$
  there exists a symbolic execution $t,\sigma''\eval t'',\sigma''',\phi$.
\end{lemma}


\begin{lemma}[Completeness of evaluation]
  \label{lem:completeEval}
  For all $e,\sigma$ such that $e,\sigma \bar{\eval}\hat{v},\hat{\sigma'}$
  there exists a symbolic execution $e,\sigma''\eval v',\sigma''',\phi$.
\end{lemma}


The proofs of \cref{lem:completeNormalise,lem:completeStride,lem:completeEval} are trivial, since every concrete execution in these semantics is also a symbolic one.
The full proofs are listed in \cref{sec:completeness-proofs}.
