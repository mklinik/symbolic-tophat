% !TEX root=../main.tex


\section{Examples}
\label{sec:examples}

This section will briefly introduce the task-oriented programming language \TOPHAT,
accompanied by two examples to illustrate how the language works and what kind of properties we would like to prove.

\subsection{\TOPHAT}

\TOPHAT is a task-oriented programming language.
Its aim is to model real world collaboration.

Programs in \TOPHAT are called tasks.
The smallest elements of a task is called an editor.

Editors are the basic method for communicating with the outside world.
There are three different Editors.
\begin{description}
  \item[$\Edit v$] Valued editor.\\
    This editor holds a value $v$ of a certain type.
    A new value of that type can be given as input.
  \item[$\Enter \tau$] Unvalued editor.\\
    This editor holds no value, and can receive a value of type $\tau$.
    It will then turn into a valued editor.
  \item[$\Update l$] Shared editor.\\
    This editor refers to a shared location $l$.
    Its observable value is the value stored at that location.
    It can receive a new value, this value will then be stored at location $l$.
\end{description}

Editors can be combined into tasks using combinators.
These combinators describe the way people collaborate.
The following combinators are available in \TOPHAT.

\begin{description}
  \item[$t \Then e$] Step.\\
  Users can work on task $t$.
  As soon as $t$ yields a value, that value is passed on to the right hand side, with which it continues.
  \item[$t \Next e$] User Step.\\
  Users can work on task $t$.
  When they are done, and $t$ yields a value, the user can send a continue event to the combinator.
  The value of $t$ is then passed on to the right hand side, with which it continues.
  \item[$t_1 \And t_1$] Composition.\\
  Users can work on tasks $t_1$ and $t_2$ at the same time.
  \item[$t_1 \Or t_2$] Choice.\\
  The system chooses between $t_1$ or $t_2$. If either of those returns a value, the system chooses that task.
  \item[$e_1 \Xor e_1$] User choice.\\
  A user has to make a choice between either the left or the right hand side.
  The user continues to work on the chosen task.
\end{description}

In addition to editors and combinators, \TOPHAT also contains a fail task $\Fail$.
This task is used by programmers to indicate that a task is not reachable or viable.
For example, when the right hand side of a step combinator is $\Fail$, the step will not proceed onto that task.

This language of tasks and combinators is embedded in the simply typed lambda calculus, augmented with references, pairs, if-then-else, booleans, integers and string, and unary and binary operations on these constants.
The full syntax of this host language is listed in Section~\ref{expressions}.

The references present in the host language allow tasks to communicate with each other,
sharing information that is globally available.

Finally, several observations can be made over tasks.
Using the value function $\Value$, the current value of a task can be determined.
The value function is a partial function, since not all tasks have a value.
For example, $\Enter \tau$ holds no value.
We can also observe wether or not a task is failing, by means of the failing function $\Failing$.
The step combinator makes use of both functions in order to determine if it can step.
First, it uses $\Value$ to see if the left hand side produces a value.
If that is the case, it uses the $\Failing$ function to see if it is safe to step to the right hand side using that value.
The complete definition of the value and failing function are listed in Section~\ref{subsec:observations}.


\subsection{Flightbooking}


\subsection{Tax subsidy request}

\begin{TASK}
  enter <<BSN, Address, Date>>
      >>= \ <<bsn, address, today>> .
  provideDocuments <&> companyApprove
      >>= \ <<<<invoiceAmount, invoiceDate>>, cApproved>> .
  officerApprove invoiceAmount invoiceDate today cApproved
      >>= \ decision .
  let subsidyAmount = if decision then min 600 (invoiceAmount / 10) else 0 in
  edit <<subsidyAmount, decision, cApproved, invoiceAmount, invoiceDate, today>>
  provideDocuments = enter <<Amount, Date>>
  companyApprove = edit True <?> edit False
  officerApprove = invoiceAmount invoiceDate today cApproved = something something
\end{TASK}
