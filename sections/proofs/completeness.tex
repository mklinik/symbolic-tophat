% !TEX root=../../main.tex





%
% \begin{proof}[Proof of Theorem~\ref{thm:complete}]
%   \fixme{we cannot prove this, definition of drive function is incorrect}
% \end{proof}



\begin{proof}[Proof of Lemma~\ref{lem:completeEval}]
This lemma holds trivially; assume $\sigma''=\sigma$, then we have $t,\sigma\eval v,\sigma',\True$.
In other words; every concrete evaluation is also a valid symbolic execution.
\end{proof}



\begin{proof}[Proof of Lemma~\ref{lem:completeStride}]
This lemma holds trivially; assume $\sigma''=\sigma$, then we have $t,\sigma\stride t',\sigma',\True$.
In other words; every concrete evaluation is also a valid symbolic execution.
\end{proof}



\begin{proof}[Proof of Lemma~\ref{lem:completeNormalise}]
This lemma holds trivially; assume $\sigma''=\sigma$, then we have $t,\sigma\normalise t',\sigma',\True$.
In other words; every concrete evaluation is also a valid symbolic execution.
\end{proof}



\begin{proof}[Proof of Lemma~\ref{lem:completeHandle}]

  We prove Lemma~\ref{lem:completeHandle} by induction over $t$.\\

  \case{$t=\Edit v$}
  {One rule applies in this case, namely \userule{H-Change}\\
  Take $i=s$ and assume $\sigma''=\sigma$. $s\sim v'$ holds by definition.
  Then by the Sym-H-Change rule, we know that a symbolic execution exists.

  }

  \case{$t=\Enter \tau$}
  {One rule applies in this case, namely \userule{H-Fill}\\
  Take $i=s$ and assume $\sigma''=\sigma$. $s\sim v$ holds by definition.
  Then by the Sym-H-Fill rule,
  we know that a symbolic execution exists.
  }

  \case{$t=\Update l$}
  {One rule applies in this case, namely \userule{H-Update}\\
  Take $i=s$ and assume $\sigma''=\sigma$. $s\sim v'$ holds by definition.
  Then by the Sym-H-Update rule,
  we know that a symbolic execution exists.
   }

  \case{$t=t_1\Next e_2$}
  {Two rules apply in this case\\
    \case{\userule{H-Next}}
    {
    Take $i=s$ and assume $\sigma''=\sigma$. $s\sim \Continue$ holds by definition.
    Then by the Sym-H-Next rule, we know that a symbolic execution exists.
    }
    \case{\userule{H-PassNext}}
    {
    By application of the induction hypothesis, we obtain the following.\\
    For all $t_1,\sigma,j$ such that $t_1,\sigma\xrightarrow[j]{}t_1',\sigma'$ there exists an $i\sim j$ such that $t_1'',\sigma''\handle{}t_1''',\sigma''',i,\phi$.\\
    From this we can conclude that there exists a symbolic execution $t_1\Next e_2,\sigma\handle{} t_1'''\Next e_2,\sigma''',i,\phi$, and that $i\sim j$.
    }
  }


  \case{$t=t_1\Then e_2$}
  {
  One rule applies in this case, namely \userule{H-PassThen}\\

  By application of the induction hypothesis, we obtain the following.\\
  For all $t_1,\sigma,j$ such that $t_1,\sigma\xrightarrow[j]{}t_1',\sigma'$ there exists an $i\sim j$ such that $t_1'',\sigma''\handle{}t_1''',\sigma''',i,\phi$.\\
  From this we can conclude that there exists a symbolic execution $t_1\Then e_2,\sigma\handle{} t_1'''\Then e_2,\sigma''',i,\phi$, and $i\sim j$.
  }
  \case{$t=e_1\Xor e_2$}
  {
  Two rules apply in this case.\\
  \case{\userule{H-PickLeft}}
    {
      Take $i=s$. $s\sim \Left$ holds by definition.\\
      Lemma~\ref{lem:completeNormalise} gives us the following.\\
      There exists a symbolic execution $e_1,\sigma\normalise t_1,\sigma_1,\phi$.\\
      There exists a symbolic execution $e_2,\sigma_1\normalise t_2,\sigma_2,\phi$.\\

      We can now conclude that a symbolic execution exists.
      Either by the \refrule{Sym-H-PickLeft} rule, in case $\Failing(t_2,\sigma_2)$, or by the \refrule{Sym-H-Pick} rule in case $\neg\Failing(t_2,\sigma_2)$.
    }
  \case{\userule{H-PickRight}}
    {
    Take $i=s$. $s\sim \Left$ holds by definition.\\
    Lemma~\ref{lem:completeNormalise} gives us the following.\\
    There exists a symbolic execution $e_1,\sigma\normalise t_1,\sigma_1,\phi$.\\
    There exists a symbolic execution $e_2,\sigma_1\normalise t_2,\sigma_2,\phi$.\\

    We can now conclude that a symbolic execution exists.
    Either by the \refrule{Sym-H-PickRight} rule, in case $\Failing(t_1,\sigma_1)$, or by the \refrule{Sym-H-Pick} rule in case $\neg\Failing(t_1,\sigma_1)$.
    }
  }
  \case{$t=t_1\Or t_2$}
    {
    Two rules applies in this case.\\
    \case{\userule{H-FirstOr}}
    {
    Take $i=\First i$.

    By application of the induction hypothesis, we obtain the following.\\
    For all $t_1,\sigma,j$ such that $t_1,\sigma\xrightarrow[]{j}t_1',\sigma'$ there exists an $i\sim j$ such that $t_1'',\sigma''\handle{}t_1''',\sigma''',i,\phi$.\\

    From this, we can conclude that $\First i\sim \First j$.
    From \refrule{Sym-H-Or}, and the conclusion of the induction hypothesis,
    we can conclude that there exists an $i$ such that $t_1\Or t_2,\sigma\handle{}t_1'\Or t_2,\sigma',i,\phi$.
    }
    \case{\userule{H-SecondOr}}
    {
    Take $i=\Second i$.

    By application of the induction hypothesis, we obtain the following.\\
    For all $t_2,\sigma,j$ such that $t_2,\sigma\xrightarrow[]{j}t_2',\sigma'$ there exists an $i\sim j$ such that $t_2'',\sigma''\handle{}t_2''',\sigma''',i,\phi$.\\

    From this, we can conclude that $\Second i\sim \Second j$.
    From \refrule{Sym-H-Or}, and the conclusion of the induction hypothesis,
    we can conclude that there exists an $i$ such that $t_1\Or t_2,\sigma\handle{}t_1\Or t_2',\sigma',i,\phi$.
    }
    }
  \case{$t=t_1\And t_2$}
    {
    Two rules applies in this case.\\
    \case{\userule{H-FirstAnd}}
    {
    Take $i=\First i$.

    By application of the induction hypothesis, we obtain the following.\\
    For all $t_1,\sigma,j$ such that $t_1,\sigma\xrightarrow[]{j}t_1',\sigma'$ there exists an $i\sim j$ such that $t_1'',\sigma''\handle{}t_1''',\sigma''',i,\phi$.\\

    From this, we can conclude that $\First i\sim \First j$.
    From \refrule{Sym-H-And}, and the conclusion of the induction hypothesis,
    we can conclude that there exists an $i$ such that $t_1\And t_2,\sigma\handle{}t_1'\Or t_2,\sigma',i,\phi$.
    }
    \case{\userule{H-SecondAnd}}
    {
    Take $i=\Second i$.

    By application of the induction hypothesis, we obtain the following.\\
    For all $t_2,\sigma,j$ such that $t_2,\sigma\xrightarrow[]{j}t_2',\sigma'$ there exists an $i\sim j$ such that $t_2'',\sigma''\handle{}t_2''',\sigma''',i,\phi$.\\

    From this, we can conclude that $\First i\sim \Second j$.
    From \refrule{Sym-H-And}, and the conclusion of the induction hypothesis,
    we can conclude that there exists an $i$ such that $t_1\And t_2,\sigma\handle{}t_1\And t_2',\sigma',i,\phi$.
    }
    }

\end{proof}



\begin{proof}[Proof of Theorem~\ref{thm:completeDrive}]
  The driving semantics only consists of one rule, namely \userule{I-Handle}.\\
  By Lemma~\ref{lem:completeHandle} we obtain the following.\\
  $t,\sigma\xrightarrow[j]{}\bar{t'},\bar{\sigma'}\implies \exists i . t,\sigma\handle{}t',\sigma',i,\phi \land i\sim j$\\
  Then by Lemma~\ref{lem:completeNormalise} we obtain the following.\\
  $\bar{t'},\bar{\sigma'}\bar{\normalise}\bar{t''},\bar{\sigma''}\implies \bar{t'},\bar{\sigma'}\normalise t'',\sigma'',\phi'$\\
  From the above, together with the Sym-I-Handle rule, we can conclude that there exists a symbolic execution $t,\sigma\drive{} t',\sigma',i,\phi\land i\sim j$.

\end{proof}
