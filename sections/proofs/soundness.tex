% !TEX root=../../main.tex


% \begin{proof}[Proof of Theorem~\ref{thm:sound}]
%     \fixme{we cannot prove this, definition of drive function is incorrect}
% \end{proof}

\section{Soundness proofs}
\label{sec:soundness-proofs}

\begin{proof}[Proof of Lemma~\ref{lem:soundeval}]
  We prove Lemma~\ref{lem:soundeval} by induction over $e$.

\case{$e=v$}
  {One rule applies, namely \userule{SE-Value}
  Since this rule does not generate constraints, any $M$ will do.
  Since neither the state, nor the expression is altered by the evaluation rule \userule{E-Value},
  this case holds true trivially.
  }

\case{$e=\tuple{e_1,e_2}$}
  {One rule applies, namely \userule{SE-Pair}\\
  Provided that $M\phi_1\wedge M\phi_2$,
  we need to demonstrate that  \userule{E-Pair} with $\hat{\sigma}=M\sigma$,
  $M\tuple{v_1,v_2} \equiv \tuple{\hat{v_1},\hat{v_2}}$ and $M\sigma''\equiv\hat{\sigma''}$.

  From the induction hypothesis, we obtain the following.\\
  $\forall M_1 .  M_1\phi_1 \implies e_1, M_1\sigma \hat{\eval}\hat{v_1},\hat{\sigma'}\land  M_1v_1\equiv \hat{v_1} \land  M_1\sigma'\equiv\hat{\sigma'}$ and
  $\forall M_2 . M_2\phi_2 \implies e_2,M_2\sigma' \hat{\eval}\hat{v_2},\hat{\sigma''}\land M_2v_2\equiv \hat{v_2} \wedge M_2\sigma'\equiv\hat{\sigma''}$

  Since $M$ satisfies both $\phi_1$ and $\phi_2$,
  and we know that $M\sigma'\equiv \hat{\sigma'}$
  we obtain that $e_1,M\sigma \hat{\eval}\hat{v_1},\hat{\sigma'}$,
  $e_2,M\sigma'\hat{\eval}\hat{v_2},\hat{\sigma''}$, $M\sigma'\equiv \hat{\sigma'}$,
 $M v_1\equiv \hat{v_1}$ and $M v_2 \equiv \hat{v_2}$ and therefore $M\tuple{v_1,v_2} \equiv \tuple{\hat{v_1},\hat{v_2}}$.
  From the IH we directly obtain that $M \sigma'' \equiv\hat{\sigma''}$.
  }

\case{$e=\Fst e$}
{
  One rule applies, namely \userule{SE-First}\\
  Provided that $M\phi$,
  we need to demonstrate that \userule{E-First} with $\hat{\sigma}=M\sigma$,
  $M v_1\equiv \hat{v_1}$ and $M\sigma'\equiv\hat{\sigma'}$.

  From the induction hypothesis, we obtain the following.\\
  $\forall M_1 . M_1\phi \implies e, M_1\sigma\hat{\eval}\tuple{\hat{v_1},\hat{v_2}},\hat{\sigma'}\land M_1\tuple{v_1,v_2}\equiv \tuple{\hat{v_1},\hat{v_2}}\land M_1\sigma'\equiv\hat{\sigma'}$

  Since $M$ satisfies $\phi$,
  we directly obtain that $\Fst e,\sigma\hat{\eval}\hat{v_1}$,
  $M v_1\equiv \hat{v_1}$ and $M\sigma'\equiv\hat{\sigma'}$.
  }

\case{$e=\Snd e$}
{ One rule applies, namely \userule{SE-Second}\\
  Provided that $M\phi$,
  we need to demonstrate that \userule{E-Second} with $\hat{\sigma}=M\sigma$,
  $M v_2\equiv \hat{v_2}$ and $M\sigma'\equiv\hat{\sigma'}$.

  From the induction hypothesis, we obtain the following.\\
  $\forall M_1 . M_1\phi \implies e, M_1\sigma\hat{\eval}\tuple{\hat{v_1},\hat{v_2}},\hat{\sigma'}\land M_1\tuple{v_1,v_2}\equiv \tuple{\hat{v_1},\hat{v_2}}\land M_1\sigma'\equiv\hat{\sigma'}$

  Since $M$ satisfies $\phi$,
  we directly obtain that $\Snd e,\sigma\hat{\eval}\hat{v_2}$,
  $M v_2\equiv \hat{v_2}$ and $M\sigma'\equiv\hat{\sigma'}$.
  }

\case{$e=e_1::e_2$}
  {One rule applies, namely \userule{SE-Cons}\\
  Provided that $M\phi$,
  we need to demonstrate that \userule{E-Cons} with $bar{\sigma}=M\sigma$,
  $M v_1::v_2\equiv \hat{v_1}::\hat{v_2}$ and $M\sigma''\equiv\hat{\sigma''}$.

  From the induction hypothesis, we obtain the following.\\
  $\forall M_1 .  M_1\phi_1 \implies e_1, M_1\sigma \hat{\eval}\hat{v_1},\hat{\sigma'}\land  M_1v_1\equiv \hat{v_1} \land  M_1\sigma'\equiv\hat{\sigma'}$ and
  $\forall M_2 . M_2\phi_2 \implies e_2,M_2\sigma' \hat{\eval}\hat{v_2},\hat{\sigma''}\land M_2v_2\equiv \hat{v_2} \wedge M_2\sigma'\equiv\hat{\sigma''}$

  Since $M$ satisfies both $\phi_1$ and $\phi_2$,
  and we know that $M\sigma'\equiv \hat{\sigma'}$
  we obtain that $e_1,M\sigma \hat{\eval}\hat{v_1},\hat{\sigma'}$,
  $e_2,M\sigma'\hat{\eval}\hat{v_2},\hat{\sigma''}$, $M\sigma'\equiv \hat{\sigma'}$,
 $M v_1\equiv \hat{v_1}$ and $M v_2 \equiv \hat{v_2}$ and therefore $M(v_1::v_2) \equiv \hat{v_1}::\hat{v_2}$.
  From the IH we directly obtain that $M \sigma'' \equiv\hat{\sigma''}$.}

\case{$e=\Head e$}
  {One rule applies, namely \userule{SE-Head}\\
  Provided that $M\phi$,
  we need to demonstrate that \userule{E-Head} with $\hat{\sigma}=M\sigma$,
  $M v_1\equiv \hat{v_1}$ and $M\sigma'\equiv\hat{\sigma'}$.

  From the induction hypothesis, we obtain the following.\\
  $\forall M_1 . M_1\phi \implies e, M_1\sigma\hat{\eval}\hat{v_1}::\hat{v_2},\hat{\sigma'}\land M_1 (v_1::v_2)\equiv \hat{v_1}::\hat{v_2}\land M_1\sigma'\equiv\hat{\sigma'}$

  Since $M$ satisfies $\phi$,
  we directly obtain that $\Head e,\sigma\hat{\eval}\hat{v_1}$,
  $M v_1\equiv \hat{v_1}$ and $M\sigma'\equiv\hat{\sigma'}$.}

\case{$e=\Tail e$}
  {One rule applies, namely \userule{SE-Tail}\\
    Provided that $M\phi$,
    we need to demonstrate that \userule{E-Tail} with $\hat{\sigma}=M\sigma$,
    $M v_2\equiv \hat{v_2}$ and $M\sigma'\equiv\hat{\sigma'}$.

    From the induction hypothesis, we obtain the following.\\
    $\forall M_1 . M_1\phi \implies e, M_1\sigma\hat{\eval}\hat{v_1}::\hat{v_2},\hat{\sigma'}\land M_1(v_1::v_2)\equiv \hat{v_1}::\hat{v_2}\land M_1\sigma'\equiv\hat{\sigma'}$

    Since $M$ satisfies $\phi$,
    we directly obtain that $\Tail e,\sigma\hat{\eval}\hat{v_2}$,
    $M v_2\equiv \hat{v_2}$ and $M\sigma'\equiv\hat{\sigma'}$.}

\case{$e=e_1 e_2$}
  {One rule applies, namely \userule{SE-App}\\
  Provided that $M\phi_1 \land M\phi_2\land M\phi_3$,
  we need to demonstrate that
  \userule{E-App} with $\hat{\sigma}=M\sigma$,
   $M v_1 \equiv \hat{v_1}$ and $M\sigma'''\equiv\hat{\sigma'''}$.

  From the induction hypothesis, we obtain the following.\\
  $\forall M_1 . M_1\phi_1 \implies e_1,M_1\sigma\hat{\eval}\lambda x : \tau.\hat{e_1'},\hat{\sigma'}
  \land M_1\lambda x : \tau.e_1' \equiv \lambda x : \tau.\hat{e_1'} \land M_1\sigma'\equiv\hat{\sigma'}$
  and\\
  $\forall M_2 . M_2\phi_2 \implies e_2,M_2\sigma'\hat{\eval}\hat{v_2},\hat{\sigma''}
  \land M_2 v_2 \equiv \hat{v_2} \land M_2\sigma'' \equiv\hat{\sigma''}$
  and\\
  $\forall M_3 . M_3\phi_3 \implies \hat{e_1'}[x\mapsto \hat{v_2}],M_3\sigma''\hat{\eval}\hat{v_1},\hat{\sigma'''}
  \land M_3 v_1\equiv \hat{v_1} \land M_3 \sigma'''\equiv\hat{\sigma'''}$.

  Since $M$ satisfies both $\phi_1$, $\phi_2$ and $\phi_3$, and we know that
  $M\sigma' \equiv\hat{\sigma'}$ and $M\sigma'' \equiv\hat{\sigma''}$,
  we obtain that $e_1,M\sigma \hat{\eval}\lambda x : \tau.\hat{e_1'},\hat{\sigma'}$, $e_2,M\sigma' \hat{\eval}\hat{v_2},\hat{\sigma''}$ and $\hat{e_1'}[x\mapsto \hat{v_2}],M\sigma'' \hat{\eval}\hat{v_1},\hat{\sigma'''}$.
  We can then directly conclude that $M v_1 \equiv \hat{v_1}$ and $M\sigma'''\equiv\hat{\sigma'''}$.
  }

\case{$e=\If{e_1}{e_2}{e_3}$}
   {One rule applies, namely \userule{SE-If}\\

   In case that $M\phi_1\land M\phi_2 \land Mv_1$,
   we need to demonstrate that
   \userule{E-IfTrue} with $\hat{\sigma}=M\sigma$,
   $M v_2 = \hat{v_2}$ and $M\sigma''=\hat{\sigma''}$.

   From the induction hypothesis, we obtain the following.\\
   $\forall M_1 . M_1\phi_1 \implies e_1,M_1\sigma\hat{\eval}\hat{v_1},\hat{\sigma'}
   \land M_1v_1 \equiv \hat{v_1} \land M_1\sigma'\equiv\hat{\sigma'}$
   and\\
   $\forall M_2 . M_2\phi_2 \implies e_2,M_2\sigma'\hat{\eval}\hat{v_2},\hat{\sigma''}
   \land M_2 v_2 \equiv \hat{v_2} \land M_2\sigma'' \equiv\hat{\sigma''}$.

   Since $M$ satisfies $\phi_1$, and $M v_1=\True$, we know from the application of the induction hypothesis above, that $\hat{v_1}=\True$.\\
   Furthermore, $M$ satisfies $\phi_2$, so we directly obtain that $Mv_2=\hat{v_2}$ and $M\sigma''=\hat{\sigma''}$.

   In case that $M\phi_1\land M\phi_3 \land M\neg v_1$,
   we need to demonstrate that
   \userule{E-IfFalse} with $\hat{\sigma}=M\sigma$,
   $M v_3 = \hat{v_3}$ and $M\sigma''=\hat{\sigma''}$.\\

   From the induction hypothesis, we obtain the following.\\
   $\forall M_1 . M_1\phi_1 \implies e_1,M_1\sigma\hat{\eval}\hat{v_1},\hat{\sigma'}
   \land M_1v_1 \equiv \hat{v_1} \land M_1\sigma'\equiv\hat{\sigma'}$
   and\\
   $\forall M_3 . M_3\phi_3 \implies e_3,M_3\sigma'\hat{\eval}\hat{v_3},\hat{\sigma''}
   \land M_3 v_3 \equiv \hat{v_3} \land M_3\sigma'' \equiv\hat{\sigma''}$.

   Since $M$ satisfies $\phi_1$, and $M v_1=\False$, we know from the application of the induction hypothesis above, that $\hat{v_1}=\False$.\\
   Furthermore, $M$ satisfies $\phi_3$, so we directly obtain that $Mv_3=\hat{v_3}$ and $M\sigma''=\hat{\sigma''}$.
  }

\case{$e=\Ref e$}
  {One rule applies, namely \userule{SE-Ref}\\
  Provided that $M\phi$,
  we need to demonstrate that \userule{E-Ref} with $\hat{\sigma}=M\sigma$,
  $M l \equiv l$ and $M\sigma'[l\mapsto v]\equiv\hat{\sigma'}[l\mapsto\hat{v}]$.

  From the induction hypothesis, we obtain the following.\\
  $\forall M_1 .  M_1\phi \implies e, M_1\sigma \hat{\eval}\hat{v},\hat{\sigma'}\land  M_1v\equiv \hat{v} \wedge  M_1\sigma'\equiv\hat{\sigma'}$.

  We assume that the assignment of location references happens in a deterministic manner, and that we can therefore conclude that exactly the same $l$ is used in both cases. Since $l$ cannot contain any symbols, $M l \equiv l$ holds trivially.

  Since $M$ satisfies $\phi$,
  we obtain that $e,M\sigma \hat{\eval}\hat{v},\hat{\sigma'}$ and $M v\equiv \hat{v}$.
  This, together with $M \sigma' \equiv\hat{\sigma'}$ obtained from the induction hypothesis, we can conclude that $M\sigma'[l\mapsto v]\equiv\hat{\sigma'}[l\mapsto\hat{v}]$.
  }

\case{$e=!e$}
  {One rule applies, namely \userule{SE-Deref}\\
  Provided that $M\phi$, we need to demonstrate that \userule{E-Deref} with $\hat{\sigma}=M\sigma$,
  $M \sigma'(l) \equiv \hat{\sigma'}(l)$ and $M\sigma'\equiv\hat{\sigma'}$.

  From the induction hypothesis, we obtain the following.\\
  $\forall M_1 .  M_1\phi \implies e, M_1\sigma \hat{\eval}l,\hat{\sigma'}\land  M_1l\equiv l \land  M_1\sigma'\equiv\hat{\sigma'}$.

  Note that since $l$ cannot contain any symbols, $M l \equiv l$ holds trivially.

  Since $M$ satisfies $\phi$,
  we immediately obtain $e,M\sigma \hat{\eval}l,\hat{\sigma'}$,
  and $M\sigma'\equiv\hat{\sigma'}$.
}

\case{$e=e_1:=e_2$}
  {
  One rule applies, namely \userule{SE-Assign}\\
  Provided that $M\phi_1\wedge M\phi_2$,
  we need to demonstrate that \userule{E-Assign} with $\hat{\sigma}=M\sigma$,
  $M\unit \equiv \unit$, which holds true trivially,
  and $M\sigma''[l\mapsto v_2]\equiv\hat{\sigma''}[l\mapsto\hat{v_2}]$.

  From the induction hypothesis, we obtain the following.\\
  $\forall M_1 .  M_1\phi_1 \implies e_1, M_1\sigma \hat{\eval}l,\hat{\sigma'}\land  M_1 l\equiv l \land  M_1\sigma'\equiv\hat{\sigma'}$ and\\
  $\forall M_2 . M_2\phi_2 \implies e_2,M_2\sigma' \hat{\eval}\hat{v_2},\hat{\sigma''}\land M_2v_2\equiv \hat{v_2} \land M_2\sigma'\equiv\hat{\sigma''}$

  Since $M$ satisfies both $\phi_1$ and $\phi_2$, and we know that $M\sigma'\equiv \hat{\sigma'}$,
  we obtain that $e_1,M\sigma \hat{\eval}l,\hat{\sigma'}$,
  $e_2,M\sigma'\hat{\eval}\hat{v_2},\hat{\sigma''}$,
  $M l\equiv l$, $M v_2 \equiv \hat{v_2}$ and $M\sigma''\equiv\hat{\sigma''}$ and therefore $M\sigma''[l\mapsto v_2]\equiv\hat{\sigma''}[l\mapsto\hat{v_2}]$.
  }

\case{$e=\Edit e$}
  {One rule applies, namely \userule{SE-Edit}\\
  Provided that $M\phi$,
  we need to demonstrate that \userule{E-Edit} with $\hat{\sigma}=M\sigma$,
  $M \Edit v \equiv \Edit \hat{v}$ and $M\sigma'\equiv\hat{\sigma'}$.

  From the induction hypothesis, we obtain the following.\\
  $\forall M_1 .  M_1\phi \implies e, M_1\sigma \hat{\eval}\hat{v},\hat{\sigma'}\land  M_1v\equiv \hat{v} \land  M_1\sigma'\equiv\hat{\sigma'}$.

  Since $M$ satisfies $\phi$,
  we obtain that $e,M\sigma \hat{\eval}\hat{v},\hat{\sigma'}$,
  $M \Edit v\equiv \Edit \hat{v}$.
  We can furthermore directly conclude that $\sigma' M\equiv\hat{\sigma'}$.

  }

\case{$e=\Enter \tau$}
  {
  One rule applies, namely \userule{SE-Enter}\\
  Provided that $M\phi$, we need to demonstrate that \userule{E-Enter} with $\hat{\sigma}=M\sigma$,
  $M \Enter \tau \equiv \Enter \tau$, which holds trivially since types do not hold symbols,
  and $M\sigma\equiv\hat{\sigma}$, which also hols trivially from the premise.
  }

\case{$e=\Update e$}
  {One rule applies, namely \userule{SE-Update}\\
  Provided that $M\phi$, we need to demonstrate that \userule{E-Update} with $\hat{\sigma}=M\sigma$,
  $M \Update l \equiv \Update l$ and $M\sigma'\equiv\hat{\sigma'}$.

  From the induction hypothesis, we obtain the following.\\
  $\forall M_1 .  M_1\phi \implies e, M_1\sigma \hat{\eval}l,\hat{\sigma'}\land  M_1 l\equiv l \land  M_1\sigma'\equiv\hat{\sigma'}$.

  Since $M$ satisfies $\phi$,
  we obtain that $e,M\sigma \hat{\eval}l,\hat{\sigma'}$,
  and $M \Update l\equiv \Update l$.
  We can furthermore directly conclude that $M \sigma' \equiv\hat{\sigma'}$.

  }

\case{$e=e_1\Then e_2$}
  {One rule applies, namely \userule{SE-Then}\\
  Provided that $M\phi$, we need to demonstrate that \userule{E-Then} with $\hat{\sigma}=M\sigma$,
  $M t_1\Then e_2 \equiv \hat{t_1}\Then e_2$ and $M\sigma'\equiv\hat{\sigma'}$.

  From the induction hypothesis, we obtain the following.\\
  $\forall M_1 .  M_1\phi \implies e, M_1\sigma \hat{\eval}\hat{t_1},\hat{\sigma'}\and  M_1 t_1\equiv \hat{t_1} \land  M_1\sigma'\equiv\hat{\sigma'}$.

  Since $M$ satisfies $\phi$,
  we obtain that $e,M\sigma \hat{\eval}\hat{t_1},\hat{\sigma'}$
  and $M t_1\Then e_2 \equiv \hat{t_1}\Then e_2$.
  We can furthermore directly conclude that $M \sigma' \equiv\hat{\sigma'}$.

  }

\case{$e=e_1\Next e_2$}
  {One rule applies, namely \userule{SE-Next}\\
  Provided that $M\phi$, we need to demonstrate that \userule{E-Next} with $\hat{\sigma}=M\sigma$,
  $M t_1\Next e_2 \equiv \hat{t_1}\Then e_2$ and $M\sigma'\equiv\hat{\sigma'}$.

  From the induction hypothesis, we obtain the following.\\
  $\forall M_1 .  M_1\phi \implies e, M_1\sigma \hat{\eval}\hat{t_1},\hat{\sigma'}\land  M_1 t_1\equiv \hat{t_1} \land  M_1\sigma'\equiv\hat{\sigma'}$.

  Since $M$ satisfies $\phi$,
  we obtain that $e,M\sigma \hat{\eval}\hat{t_1},\hat{\sigma'}$
  and $M t_1\Next e_2 \equiv \hat{t_1}\Next e_2$.
  We can furthermore directly conclude that $M \sigma' \equiv\hat{\sigma'}$.

  }

\case{$e=e_1\Or e_2$}
  {One rule applies, namely \userule{SE-Or}\\
  Provided that $M\phi_1\wedge M\phi_2$, we need to demonstrate that \userule{E-Or} with $\hat{\sigma}=M\sigma$, $M t_1\Or t_2 \equiv \hat{t_1}\Or\hat{t_2}$ and $M\sigma''\equiv\hat{\sigma''}$.

  From the induction hypothesis, we obtain the following.\\
  $\forall M_1 .  M_1\phi_1 \implies e_1, M_1\sigma \hat{\eval}\hat{t_1},\hat{\sigma'}\land  M_1 t_1\equiv \hat{t_1} \land  M_1\sigma'\equiv\hat{\sigma'}$ and\\
  $\forall M_2 . M_2\phi_2 \implies e_2,M_2\sigma \hat{\eval}\hat{t_1},\hat{\sigma'}\land M_2t_2\equiv \hat{t_2} \land M_2\sigma'\equiv\hat{\sigma'}$

  Since $M$ satisfies both $\phi_1$ and $\phi_2$, and we know that $M\sigma'\equiv \hat{\sigma'}$,
  we obtain that $e_1,M\sigma \hat{\eval}\hat{t_1},\hat{\sigma'}$,
  $e_2,M\sigma'\hat{\eval}\hat{t_2},\hat{\sigma}$, $M t_1\equiv \hat{t_1}$ and $M t_2 \equiv \hat{t_2}$ and therefore $M t_1\Or t_2 \equiv \hat{t_1}\Or\hat{t_2}$.
  From the IH we directly obtain that $M \sigma'' \equiv\hat{\sigma''}$.

  }

\case{$e=e_1\Xor e_2$}
  {  One rule applies, namely \userule{SE-Xor}\\
    Provided that $M\phi$, we need to demonstrate that \userule{E-Xor} with $\hat{\sigma}=M\sigma$,
    $M e_1\Xor e_2 \equiv \hat{e_1}\Xor\hat{e_2}$,
    which holds trivially,
    and $M\sigma\equiv\hat{\sigma}$, which also holds trivially from the premise.

  }

\case{$e=\Fail$}
  {  One rule applies, namely \userule{SE-Fail}\\
    Provided that $M\phi$, we need to demonstrate that \userule{E-Fail} with $\hat{\sigma}=M\sigma$,
    $M \Fail \equiv \Fail$, which holds trivially since fail do not hold symbols,
    and $M\sigma\equiv\hat{\sigma}$, which also hols trivially from the premise.

  }
\end{proof}



\begin{proof}[Proof of Lemma~\ref{lem:soundstride}]
  We prove Lemma~\ref{lem:soundstride} by induction over $t$.

\case{$t=t_1\Then e_2$}
  {
  Three rules apply.\\
  \case{\userule{SS-ThenStay}}
    {Provided that $M\phi\equiv\True$
    we need to demonstrate that
    \userule{S-ThenStay} with $\hat{\sigma}=M\sigma$,
    $M t_1'\Then e_2 \equiv \hat{t_1'}\Then e_2 $ and $ M\sigma'\equiv \hat{\sigma'}$.

    From the induction hypothesis, we obtain the following.\\
    $\forall M_1 . M_1 \phi \implies t_1,M_1\sigma \hat{\stride} \hat{t_1'},\hat{\sigma'}\land M_1 t_1'\equiv\hat{t_1'}\land M_1\sigma' \equiv \hat{\sigma'}$.

    Since $M$ satisfies $\phi$,
    we know that
    $t_1,M\sigma \hat{\stride} \hat{t_1'},\hat{\sigma'}$
    and $M t_1'\equiv\hat{t_1'}$,
    and therefore also $M t_1'\Then e_2 \equiv \hat{t_1'}\Then e_2$,
    and from the induction hypothesis, we directly obtain  $M\sigma'\equiv \hat{\sigma'}$.
    }
  \case{\userule{SS-ThenFail}}
    {Provided that $M\phi\equiv\True$
    we need to demonstrate that
    \userule{S-ThenFail} with $\hat{\sigma}=M\sigma$,
    $M t_1'\Then e_2 \equiv \hat{t_1'}\Then e_2$ and $M\sigma'\equiv \hat{\sigma'}$.

    From the induction hypothesis, we obtain the following.\\
    $\forall M_1 . M_1 \phi \implies t_1,M_1\sigma \hat{\stride} \hat{t_1'},\hat{\sigma'}\land M_1 t_1'\equiv\hat{t_1'}\land M_1\sigma' \equiv \hat{\sigma'}$.

    Since $M$ satisfies $\phi$,
    we know that
    $t_1,M\sigma \hat{\stride} \hat{t_1'},\hat{\sigma'}$
    and $M t_1'\equiv\hat{t_1'}$,
    and therefore also $M t_1'\Then e_2 \equiv \hat{t_1'}\Then e_2$,
    and from the induction hypothesis, we directly obtain  $M\sigma'\equiv \hat{\sigma'}$.
    }
  \case{\userule{SS-ThenCont}}
    {Provided that $M\phi_1\land M\phi_2$
    we need to demonstrate that
    \userule{S-ThenCont} with $\hat{\sigma}=M\sigma$,
    $M t_2 \equiv \hat{t_2}\Then e_2$ and $M\sigma''\equiv \hat{\sigma''}$.

    From the induction hypothesis, we obtain the following.\\
    $\forall M_1 . M_1 \phi_1 \implies t_1,M_1\sigma \hat{\stride} \hat{t_1'},\hat{\sigma'}\implies M_1 t_1'\equiv\hat{t_1'}\land M_1\sigma' \equiv \hat{\sigma'}$.\\
    From Lemma~\ref{lem:soundeval} we know that\\
    $\forall M_2 . M_2 \phi_2 \implies e_2\hat{v_1}M_2\sigma'\hat{\eval}\hat{t_2},\hat{\sigma''}\and M_2 t_2\equiv \hat{t_2}\land M_2\sigma''\equiv \hat{\sigma''}$.

    Since $M$ satisfies both $\phi_1$ and $\phi_2$,
    we know that
    $t_1,M\sigma \hat{\stride} \hat{t_1'},\hat{\sigma'}$ and $e_2\hat{v_1}M\sigma'\hat{\eval}\hat{t_2},\hat{\sigma''}$,
    $M t_2\equiv\hat{t_2}$,
    and from the induction hypothesis, we directly obtain  $M\sigma''\equiv \hat{\sigma''}$.

    }
  }

\case{$t=t_1\Or t_2$}
  {
  Three rules apply.\\
  \case{\userule{SS-OrLeft}}
    {Provided that $M\phi\equiv\True$
    we need to demonstrate that
    \userule{S-OrLeft} with $\hat{\sigma}=M\sigma$,
    $M t_1'\equiv \hat{t_1'}$ and $M\sigma'\equiv \hat{\sigma'}$.

    From the induction hypothesis, we obtain the following.\\
    $\forall M_1 . M_1 \phi \implies t_1,M_1\sigma \hat{\stride} \hat{t_1'},\hat{\sigma'}\and M_1 t_1'\equiv\hat{t_1'}\land M_1\sigma' \equiv \hat{\sigma'}$.

    Since $M$ satisfies $\phi$, we know that $t_1,M\sigma \hat{\stride} \hat{t_1'},\hat{\sigma'}$,
    $M t_1'\equiv\hat{t_1'}$ and $M\sigma'\equiv \hat{\sigma'}$.

    }
  \case{\userule{SS-OrRight}}
    {Provided that $M\phi_1\land M\phi_2$
    we need to deomstrate that
    \userule{S-OrRight} with $\hat{\sigma}=M\sigma$,
    $M t_2'\equiv \hat{t_2'}$ and $M\sigma''\equiv \hat{\sigma''}$.

    From the induction hypothesis, we obtain the following.\\
    $\forall M_1 . M_1 \phi_1 \implies t_1,M_1\sigma \hat{\stride} \hat{t_1'},\hat{\sigma'}\land M_1 t_1'\equiv\hat{t_1'}\land M_1\sigma' \equiv \hat{\sigma'}$ and\\
    $\forall M_2 . M_2 \phi_2 \implies t_2,M_2\sigma' \hat{\stride} \hat{t_2'},\hat{\sigma''}\land M_2 t_2'\equiv\hat{t_2'}\land M_2\sigma'' \equiv \hat{\sigma''}$.

    Since $M$ satisfies both $\phi_1$ and $\phi_2$,
    we know that
    $t_1,M\sigma \hat{\stride} \hat{t_1'},\hat{\sigma'}$ and $t_2,M\sigma'\hat{\stride}\hat{t_2'},\hat{\sigma''}$,
    $M t_2'\equiv\hat{t_2'}$ and $M\sigma''\equiv \hat{\sigma''}$.

    }
  \case{\userule{SS-OrNone}}
    {Provided that $M\phi_1\land M\phi_2$
    we need to demonstrate that
    \userule{S-OrNone} with $\hat{\sigma}=M\sigma$,
    $M t_1'\Or t_2'\equiv \hat{t_1'}\Or\hat{t_2'}$ and $M\sigma''\equiv \hat{\sigma''}$.

    From the induction hypothesis, we obtain the following.\\
    $\forall M_1 . M_1 \phi_1 \implies t_1,M_1\sigma \hat{\stride} \hat{t_1'},\hat{\sigma'}\land M_1 t_1'\equiv\hat{t_1'}\land M_1\sigma' \equiv \hat{\sigma'}$ and\\
    $\forall M_2 . M_2 \phi_2 \implies t_2,M_2\sigma' \hat{\stride} \hat{t_2'},\hat{\sigma''}\land M_2 t_2'\equiv\hat{t_2'}\land M_2\sigma'' \equiv \hat{\sigma''}$.

    Since $M$ satisfies both $\phi_1$ and $\phi_2$,
    we know that $t_1,M\sigma \hat{\stride} \hat{t_1'},\hat{\sigma'}$ and $t_2,M\sigma'\hat{\stride}\hat{t_2'},\hat{\sigma''}$,
    $M t_1'\Or t_2'\equiv \hat{t_1'}\Or\hat{t_2'}$ and $M\sigma''\equiv \hat{\sigma''}$.

    }
  }

\case{$t=\Edit v$}
  {One rule applies, namely \userule{SS-Edit}\\
  Provided that $M\True$, we need to demonstrate that \userule{S-Edit} with $\hat{\sigma}=M\sigma$,
  $M\Edit v \equiv \Edit \hat{v}$ and $M\sigma \equiv \hat{\sigma}$.
  This holds trivially.

  }

\case{$t=\Enter \tau$}
  {One rule applies, namely \userule{SS-Fill}\\
  Provided that $M\True$, we need to demonstrate \userule{S-Fill} with $\hat{\sigma}=M\sigma$,
  $M\Enter \tau \equiv \Enter \tau$ and $M\sigma \equiv \hat{\sigma}$.
  This holds trivially.
  }

\case{$t=\Update l$}
  {One rule applies, namely \userule{SS-Update}\\
  Provided that $M\True$, we need to demonstrate \userule{S-Update} with $\hat{\sigma}=M\sigma$,
  $M\Update l \equiv \Update l$ and $M\sigma \equiv \hat{\sigma}$.
  This holds trivially.
  }

\case{$t=\Fail$}
  {One rule applies, namely \userule{SS-Fail}\\
  Provided that $M\True$, we need to demonstrate that \userule{S-Fail} with $\hat{\sigma}=M\sigma$,
  $M\Fail \equiv \Fail$ and $M\sigma \equiv \hat{\sigma}$.
  This holds trivially.
  }

\case{$t=e_1\Xor e_2$}
  {One rule applies, namely \userule{SS-Xor}\\
  Provided that $M\True$, we need to demonstrate that \userule{S-Xor} with $\hat{\sigma}=M\sigma$,
  $M e_1\Xor e_2 \equiv e_1\Xor e_2$ and $M\sigma \equiv \hat{\sigma}$.
  This holds trivially.
  }

\case{$t=t_1\Next e_2$}
  {One rule applies, namely \userule{SS-Next}\\
  Provided that $M\phi$,
  we need to demonstrate that \userule{S-Xor} with $\hat{\sigma}=M\sigma$,
  $M t_1'\Next e_2\equiv \hat{t_1'}\Next e_2$ and $M \sigma'\equiv\hat{\sigma'}$.

  From the induction hypothesis, we obtain the following.\\
  $\forall M_1 . M_1 \phi \implies t_1,M_1\sigma \hat{\stride} \hat{t_1'},\hat{\sigma'}\land M_1 t_1'\equiv\hat{t_1'}\land M_1\sigma' \equiv \hat{\sigma'}$.

  Since $M$ satisfies $\phi$, we directly obtain $t_1,M_1\sigma \hat{\stride} \hat{t_1'},\hat{\sigma'}$,
  $M t_1'\Next e_2\equiv \hat{t_1'}\Next e_2$ and $M \sigma'\equiv\hat{\sigma'}$.

  }

\case{$t=t_1\And t_2$}
  {One rule applies, namely \userule{SS-And}\\
  Provided that $M\phi_1\land M\phi_2$
  we need to demonstrate $\userule{S-And}$ with $\hat{\sigma}=M\sigma$,
  $M t_1'\And t_2'\equiv \hat{t_1'}\And\hat{t_2'}$ and $M\sigma''\equiv \hat{\sigma''}$.

  From the induction hypothesis, we obtain the following.\\
  $\forall M_1 . M_1 \phi_1 \implies t_1,M_1\sigma \hat{\stride} \hat{t_1'},\hat{\sigma'}\and M_1 t_1'\equiv\hat{t_1'}\land M_1\sigma' \equiv \hat{\sigma'}$ and\\
  $\forall M_2 . M_2 \phi_2 \implies t_2,M_2\sigma' \hat{\stride} \hat{t_2'},\hat{\sigma''}\and M_2 t_2'\equiv\hat{t_2'}\land M_2\sigma'' \equiv \hat{\sigma''}$.

  Since $M$ satisfies both $\phi_1$ and $\phi_2$,
  we know that $t_1,M\sigma \hat{\stride} \hat{t_1'},\hat{\sigma'}$ and $t_2,M\sigma'\hat{\stride}\hat{t_2'},\hat{\sigma''}$,
  $M t_1'\Or t_2'\equiv \hat{t_1'}\Or\hat{t_2'}$ and $M\sigma''\equiv \hat{\sigma''}$.

  }

\end{proof}



\begin{proof}[Proof of Lemma~\ref{lem:soundnorm}]
  We prove Lemma~\ref{lem:soundnorm} by induction over $e$.

  The base case is when the SN-Done rule applies.\\
  \userule{SN-Done}\\

  Provided that $M\phi_1\land M\phi_2$\\
  we need to demonstrate that
  \userule{N-Done} with $\hat{\sigma}=M\sigma$,
  $M t\equiv \hat{t}$ and $M\sigma'\equiv \hat{\sigma'}$.

  By Lemma~\ref{lem:soundeval} and Lemma~\ref{lem:soundstride}, we know that\\
  $\forall M_1. M_1\phi_1 \implies e,M_1\sigma \hat{\eval}\hat{t},\hat{\sigma'}\land M_1 t \equiv \hat{t} \land M_1 \sigma'\equiv \hat{\sigma'}$ and\\
  $\forall M_2.M_2\phi_2\implies t,M_2\sigma'\hat{\stride}\hat{t'},\hat{\sigma''}\land M_2 t'\equiv\hat{t'}\land M_2\sigma''\equiv \hat{\sigma''}$.

  We assume $M$ to satisfy both $\phi_1$ and $\phi_2$, we have $e,M\sigma \hat{\eval}\hat{t},\hat{\sigma'}$ since $M\sigma\equiv \hat{\sigma}$.

  The only induction step is when\\
  \userule{SN-Repeat} applies.
  In this case, where we have that $M\phi_1\land M\phi_2 \land M\phi_3$,
  we need to demonstrate that
  \userule{N-Repeat} with $\hat{\sigma}=M\sigma$,
  $M t''\equiv \hat{t''}$ and $M\sigma'''\equiv \hat{\sigma'''}$.

  Again by Lemma~\ref{lem:soundeval} and Lemma~\ref{lem:soundstride}, we know that\\
  $\forall M_1. M_1\phi_1 \implies e,M_1\sigma \hat{\eval}\hat{t},\hat{\sigma'}\land M_1 t \equiv \hat{t} \land M_1 \sigma'\equiv \hat{\sigma'}$ and\\
  $\forall M_2.M_2\phi_2\implies t,M_2\sigma'\hat{\stride}\hat{t'},\hat{\sigma''}\land M_2 t'\equiv\hat{t'}\land M_2\sigma''\equiv \hat{\sigma''}$.\\
  Furthermore, we know by applying the induction hypothesis that $\forall M_3.M_3\phi_3 \implies t',M_3\sigma''\hat{\normalise} \hat{t''},\hat{\sigma'''}\land M_3 t''\equiv \hat{t''}\land M_3 \sigma'''\equiv \hat{\sigma'''}$.

  Since $M$ satisfies $\phi_1$, $\phi_2$ and $\phi_3$,
  we can conclude that

  $e,M\sigma \hat{\eval}\hat{t},\hat{\sigma'}$, $M t \equiv \hat{t} \land M \sigma'\equiv \hat{\sigma'}$ and $t,M\sigma'\hat{\stride}\hat{t'},\hat{\sigma''}$ and $M t'\equiv\hat{t'}\land M\sigma''\equiv \hat{\sigma''}$.
  This finally gives us $t',M\sigma''\hat{\normalise} \hat{t''},\hat{\sigma'''}$ from which we can conclude that which we needed to prove,
  namely $M t''\equiv \hat{t''}\land M\sigma'''\equiv \hat{\sigma'''}$.
\end{proof}



\begin{proof}[Proof of Lemma~\ref{lem:soundhandle}]
  We prove Lemma~\ref{lem:soundhandle} by induction over $t$.\\

  \case{$t=\Edit v$}
    {One rule applies, namely \userule{SH-Change}\\
    Provided that $M\True$ we need to demonstrate that \userule{H-Change} with $\hat{\sigma}=M\sigma$ and $M s = v'$,
    $M \Edit s\equiv \Edit v'$ and $ M\sigma\equiv \hat{\sigma}$.

    This follows trivially from the premise.

    }

  \case{$t=\Enter \tau$}
  {One rule applies, namely \userule{SH-Fill}\\
  Provided that $M\True$ we need to demonstrate that \userule{H-Fill} with $\hat{\sigma}=M\sigma$ and $M s = v$,
  $M \Edit s \equiv \Edit v$ and $ M\sigma\equiv \hat{\sigma}$.

  This follows trivially from the premise.

  }

  \case{$t=\Update l$}
  {One rule applies, namely \userule{SH-Update}\\
  Provided that $M\True$
  we need to demonstrate that  \userule{H-Update} with $\hat{\sigma}=M\sigma$ and $M s = v$,
  $M \Update l \equiv \Update l$ and $ M\sigma[l\mapsto s]\equiv \hat{\sigma}[l\mapsto v]$.

  \userule{H-Update} with $\hat{\sigma}=M\sigma$ follows trivially.
  $M \Update l \equiv \Update l$ follows trivially, since locations cannot contain symbols. $ M\sigma[l\mapsto s]\equiv \hat{\sigma}[l\mapsto v]$ can be concluded from the fact that $\hat{\sigma}=M\sigma$ and $M s = v$.

  }

  \case{$t=t_1\Next e_2$}
  {In this case, two rules apply.\\


  \case{\userule{SH-Next}}

  {In the case $M\phi_1$, we need to demonstrate that \userule{H-PassNext} with $\hat{\sigma}=M\sigma$ and $j= M i$,
  $M t_1' \Next e_2 \equiv \hat{t_1'}\Next e_2$ and $M\sigma'\equiv\hat{\sigma'}$.

  By the induction hypothesis we obtain the following.\\
  $\forall M_1 . M_1 \phi_1 \implies t_1,M_1\sigma \xrightarrow[]{M_1 i} \hat{t_1'},\hat{\sigma'}\land M_1 t_1'\equiv\hat{t_1'}\land M_1\sigma' \equiv \hat{\sigma'}$

  Since $M$ satisfies $\phi$, we have $t_1,M\sigma \xrightarrow[]{M i} \hat{t_1'},\hat{\sigma'}$, $M\sigma'\equiv\hat{\sigma'}$,
  which we needed to show, as well as $M t_1' \Next e_2 \equiv \hat{t_1'}\Next e_2$ since this can be concluded from $M t_1'\equiv \hat{t_1'}$.

  In the case $M\phi_2$, we need to demonstrate that \userule{H-Next} with $\hat{\sigma}=M\sigma$,
  $M t_2 \equiv \hat{t_2}$ and $M\sigma'\equiv\hat{\sigma'}$.

  From Lemma~\ref{lem:soundnorm} we obtain that $\forall M_1. M_1 \phi \implies e_2 v_1,M\sigma\hat{\normalise}\hat{t_2},\hat{\sigma'}\land M t_2\equiv\hat{t_2}\land M \sigma'\equiv\hat{\sigma'}$.

  This gives us exactly what we needed to prove this case.
    }

  \case{\userule{SH-PassNext}}
  {
  Provided that $M\phi$, we need to demonstrate that \userule{H-PassNext} with $\hat{\sigma}=M\sigma$ and $j= M i$,
  $M t_1' \Next e_2 \equiv \hat{t_1'}\Next e_2$ and $M\sigma'\equiv\hat{\sigma'}$.

  By the induction hypothesis we obtain the following.\\
  $\forall M_1 . M_1 \phi_1 \implies t_1,M_1\sigma \xrightarrow[]{M_1 i} \hat{t_1'},\hat{\sigma'}\land M_1 t_1'\equiv\hat{t_1'}\land M_1\sigma' \equiv \hat{\sigma'}$

  Since $M$ satisfies $\phi$, we have $t_1,M\sigma \xrightarrow[]{M i} \hat{t_1'},\hat{\sigma'}$, $M\sigma'\equiv\hat{\sigma'}$,
  which we needed to show, as well as $M t_1' \Next e_2 \equiv \hat{t_1'}\Next e_2$ since this can be concluded from $M t_1'\equiv \hat{t_1'}$.
  }
  }

  \case{$t=t_1\Then e_2$}
  {One rule applies, namely \userule{SH-PassThen}\\
  Provided that $M\phi$, we need to demonstrate that \userule{H-PassThen} with $\hat{\sigma}=M\sigma$ and $j= M i$,
  $M t_1'\Then e_2\equiv \hat{t_1'}\Then e_2$ and $M\sigma'\equiv\hat{\sigma'}$.

  By the induction hypothesis we obtain the following.\\
  $\forall M_1 . M_1 \phi_1 \implies t_1,M_1\sigma \xrightarrow[]{M_1 i} \hat{t_1'},\hat{\sigma'}\land M_1 t_1'\equiv\hat{t_1'}\land M_1\sigma' \equiv \hat{\sigma'}$

  Since $M$ satisfies $\phi$, we have $t_1,M\sigma \xrightarrow[]{M i} \hat{t_1'},\hat{\sigma'}$ and $M\sigma'\equiv\hat{\sigma'}$,
  which we needed to show, as well as $M t_1' \Then e_2 \equiv \hat{t_1'}\Then e_2$ since this can be concluded from $M t_1'\equiv \hat{t_1'}$.

  }

  \case{$t=e_1\Xor e_2$}
  {
  In this case, three rules apply.\\
    \case{\userule{SH-Pick}}
    {
    Either we have that $M(\phi_1\wedge s=\Left)$ or $M(\phi_2\wedge s=\Right)$.
    In the first case, the proof is identical to the SH-PickLeft rule.
    In the second cse, the proof is identical to the SH-PickRight rule.
    }

    \case{\userule{SH-PickLeft}}
    {Provided that $M(\phi_2\wedge s=\Left)$, we need to demonstrate that \userule{H-PickLeft} with $\hat{\sigma}=M\sigma$,
    $M t_1\equiv \hat{t_1}$ and $M\sigma'\equiv \hat{\sigma'}$.

    From Lemma~\ref{lem:soundnorm} we obtain that $\forall M_1. M_1 \phi \implies e_1,M\sigma\hat{\normalise}\hat{t_1},\hat{\sigma'}\land M t_1\equiv\hat{t_1}\land M \sigma'\equiv\hat{\sigma'}$.

    Since $M$ satisfies $\phi$, we have $e_1,M\sigma \hat{\normalise}[] \hat{t_1},\hat{\sigma'}$ and $M\sigma'\equiv\hat{\sigma'}$,
    which we needed to show, as well as $M t_1 \equiv \hat{t_1}$.

    }
    \case{\userule{SH-PickRight}}
    {Provided that $M(\phi_2\wedge s=\Right)$ we need to demonstrate that \userule{H-PickRight} with $\hat{\sigma}=M\sigma$,
    $M t_2\equiv \hat{t_2}$ and $M\sigma'\equiv \hat{\sigma'}$.

    From Lemma~\ref{lem:soundnorm} we obtain that $\forall M_1. M_1 \phi \implies e_2,M\sigma\hat{\normalise}\hat{t_2},\hat{\sigma'}\land M t_2\equiv\hat{t_2}\land M \sigma'\equiv\hat{\sigma'}$.

    Since $M$ satisfies $\phi$, we have $e_2,M\sigma \hat{\normalise}[] \hat{t_2},\hat{\sigma'}$ and $M\sigma'\equiv\hat{\sigma'}$,
    which we needed to show, as well as $M t_2 \equiv \hat{t_2}$.
    }
  }

  \case{$t=t_1\And t_2$}
  {
  In this case, two rules apply.\\
    \case{\userule{SH-FirstAnd}}
    {Provided that $M\phi$, we need to demonstrate that \userule{H-FirstAnd} with $\hat{\sigma}=M\sigma$,
    $M t_1'\And t_2\equiv \hat{t_1'}\And t_2$ and $M\sigma'\equiv \hat{\sigma'}$.

    By the induction hypothesis we obtain the following.\\
    $\forall M_1 . M_1 \phi_1 \implies t_1,M_1\sigma \xrightarrow[]{M_1 i} \hat{t_1'},\hat{\sigma'}\land M_1 t_1'\equiv\hat{t_1'}\land M_1\sigma' \equiv \hat{\sigma'}$

    Since $M$ satisfies $\phi$, we have $t_1,M\sigma\xrightarrow[]{M i} \hat{t_1'},\hat{\sigma'}$ and $M\sigma'\equiv\hat{\sigma'}$,
    which we needed to show, as well as $M t_1'\And t_2\equiv \hat{t_1'}\And t_2$, which follows from $M t_1' \equiv \hat{t_1'}$.

    }
    \case{\userule{SH-SecondAnd}}
    {Provided that $M\phi$, we need to demonstrate that \userule{H-SecondAnd} with $\hat{\sigma}=M\sigma$,
    $M t_1\And t_2'\equiv t_1\And \hat{t_2}$ and $M\sigma'\equiv \hat{\sigma'}$.

    By the induction hypothesis we obtain the following.\\
    $\forall M_1 . M_1 \phi_1 \implies t_2,M_1\sigma \xrightarrow[]{M_1 i} \hat{t_2'},\hat{\sigma'}\land M_1 t_2'\equiv\hat{t_2'}\land M_1\sigma' \equiv \hat{\sigma'}$

    Since $M$ satisfies $\phi$, we have $t_2,M\sigma\xrightarrow[]{M i} \hat{t_2'},\hat{\sigma'}$ and $M\sigma'\equiv\hat{\sigma'}$,
    which we needed to show, as well as $M t_1\And t_2'\equiv t_1\And \hat{t_2'}$, which follows from $M t_2' \equiv \hat{t_2'}$.}
  }

  \case{$t=e_1\Or e_2$}
  {One rule applies, namely \userule{SH-Or}\\

  In the case that $M\phi_1$, we need to demonstrate that \userule{H-FirstOr} with $\hat{\sigma}=M\sigma$ and $M\First i =\First j$,
  $M t_1'\Or t_2\equiv \hat{t_1'}\And t_2$ and $M\sigma'\equiv \hat{\sigma'}$.

  By the induction hypothesis we obtain the following.\\
  $\forall M_1 . M_1 \phi_1 \implies t_1,M_1\sigma \xrightarrow[]{M_1 i} \hat{t_1'},\hat{\sigma'}\land M_1 t_1'\equiv\hat{t_1'}\land M_1\sigma' \equiv \hat{\sigma'}$.

  Since $M$ satisfies $\phi$, we have $t_1,M\sigma\xrightarrow[]{M i} \hat{t_1'},\hat{\sigma'}$ and $M\sigma'\equiv\hat{\sigma'}$,
  which we needed to show, as well as $M t_1'\Or t_2\equiv \hat{t_1'}\And t_2$, which follows from $M t_1' \equiv \hat{t_1'}$.

  In the case that $M\phi_2$, we need to demonstrate that \userule{H-SecondOr} with $\hat{\sigma}=M\sigma$ and $M\Second i = \Second j$,
  $M t_1\Or t_2'\equiv t_1\And \hat{t_2}$ and $M\sigma'\equiv \hat{\sigma'}$.

  By the induction hypothesis we obtain the following.\\
  $\forall M_1 . M_1 \phi_1 \implies t_2,M_1\sigma \xrightarrow[]{M_1 i} \hat{t_2'},\hat{\sigma'}\land M_1 t_2'\equiv\hat{t_2'}\land M_1\sigma' \equiv \hat{\sigma'}$

  Since $M$ satisfies $\phi$, we have $t_2,M\sigma\xrightarrow[]{M i} \hat{t_2'},\hat{\sigma'}$ and $M\sigma'\equiv\hat{\sigma'}$,
  which we needed to show, as well as $M t_1\Or t_2'\equiv t_1\And \hat{t_2'}$, which follows from $M t_2' \equiv \hat{t_2'}$.}


\end{proof}



\begin{proof}[Proof of Lemma~\ref{thm:sounddrive}]
  We prove Lemma~\ref{thm:sounddrive} as follows.
  There is only one rule that applies, namely \userule{SI-Handle}.

  Provided that $M\phi_1\land\phi_2$, we need to demonstrate that \userule{I-Handle} with $\hat{\sigma}=M\sigma$ and $j=M i$,
  $M t'' \equiv \hat{t''}$ and $M\sigma''\equiv \hat{\sigma''}$.


  Lemma~\ref{lem:soundhandle} and Lemma~\ref{lem:soundnorm} respectively give us that\\
$\forall M_1 . M_1 \phi_1 \implies t_1,M_1\sigma \xrightarrow[]{M_1 i} \hat{t'},\hat{\sigma'}\land M_1 t'\equiv\hat{t'}\land M_1\sigma' \equiv \hat{\sigma'}$ and \\
$\forall M_2 . M_2 \phi_2 \implies t',M_2\sigma' \hat{\normalise} \hat{t''},\hat{\sigma''}\land M_2 t''\equiv\hat{t''}\land M_2\sigma'' \equiv \hat{\sigma''}$.

Since $M$ satisfies both $\phi_1$ and $\phi_2$,
we obtain exactly what we needed to prove,
namely
$t_1,M\sigma \xrightarrow[]{M i} \hat{t'},\hat{\sigma'}$,
$t',M\sigma' \hat{\normalise} \hat{t''},\hat{\sigma''}$,
$M t'' \equiv \hat{t''}$ and $M\sigma''\equiv \hat{\sigma''}$.

\end{proof}
