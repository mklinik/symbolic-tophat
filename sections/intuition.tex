% !TEX root=../main.tex

\section{Intuition}
\label{sec:intuition}

This section briefly introduces the task-oriented programming language \TOPHAT,
and discuss our vision about symbolic evaluation of this language.

The \TOPHAT language consists two parts, the host language and the task language.
Programs in \TOPHAT are called \emph{tasks}.
The basic elements of tasks are editors.
Using combinators, tasks can be combined into larger tasks.

The task language is embedded in a simply typed lambda calculus with references, conditionals, booleans, integers, strings, pairs, lists and unary and binary operations on these types.
The references present in the host language allow tasks to communicate with each other, sharing information across task boundaries.
The full syntax of the host language is listed in Section~\ref{sec:language}.
Next, we discuss the main constructs of the task language.


\subsection{Editors}

Editors are the basic method for communicating with the outside world.
They are an abstraction over widgets in a \GUI library or on webpage forms.
Users can change the value hold by an editor,
in the same way they can manipulate widgets in a \GUI.
The appearance of an editor is driven by its type.
A string editor can be represented by an ordinary input field,
a date by a calendar, and a location by a pin on a map.
\TOP implementations should be able to derive this automatically.

There are three different editors.
\begin{description}
  \item[$\Edit v$] Valued editor.\\
    This editor holds a value $v$ of a certain type.
    A new value of that type can be given as input.
  \item[$\Enter \tau$] Unvalued editor.\\
    This editor holds no value, and can receive a value of type $\tau$.
    It will then turn into a valued editor.
  \item[$\Update l$] Shared editor.\\
    This editor refers to a shared location $l$.
    Its observable value is the value stored at that location.
    It can receive a new value, this value will then be stored at location $l$.
\end{description}



\subsection{Combinators}

Editors can be combined into bigger tasks using combinators.
These combinators describe the way people collaborate.
Tasks can be performed after each other or next to each other.
Also, one can make a choice to perform one of two tasks.

The following combinators are available in \TOPHAT.
\begin{description}
  \item[$t \Then e$] Step.\\
    Users can work on task $t$.
    As soon as $t$ yields a value, that value is passed on to the right hand side expression $e$, with which it continues.
  \item[$t \Next e$] User Step.\\
    Users can work on task $t$.
    When $t$ yields a value, users can send a continue event to the combinator.
    The value of $t$ is then passed on to the right hand side, with which it continues.
  \item[$t_1 \And t_1$] Composition.\\
    Users can work on tasks $t_1$ and $t_2$ at the same time.
  \item[$t_1 \Or t_2$] Choice.\\
    The system chooses between $t_1$ or $t_2$,
    based on which task first has an observable value.
    If both of the task have a value, the system chooses the left one.
  \item[$e_1 \Xor e_1$] User choice.\\
    A user has to make a choice between either the left or the right hand side.
    The user continues to work on the chosen task.
\end{description}

In addition to editors and combinators, \TOPHAT also contains a fail task ($\Fail$).
This task is used by programmers to indicate that a task is not reachable or viable.
For example, when the right hand side of a step combinator is $\Fail$, the step will not proceed onto that task.



\subsection{Observations}

Several observations can be made on tasks.
Using the value function $\Value$, the current value of a task can be determined.
The value function is a partial function, since not all tasks have a value.
For example, $\Enter \tau$ holds no value,
where $\Edit v$ has value $v$.
In particular, steps do not have a value.
One can also observe wether or not a task is failing, by means of the failing function $\Failing$.
Although $\Fail$ is clearly failing,
a parallel combination of failing tasks is also failing ($\Fail \And \Fail$).

The step combinator makes use of both functions in order to determine if it can step.
First, it uses $\Value$ to see if the left hand side produces a value.
If that is the case, it uses the $\Failing$ function to see if it is safe to step to the right hand side using that value.
The complete definition of the value and failing function are discussed in Section~\ref{subsec:observations}.
