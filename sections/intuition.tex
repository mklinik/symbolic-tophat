% !TEX root=../main.tex

\section{Intuition}
\label{sec:intuition}

This section briefly introduces the task-oriented programming language \TOPHAT,
and discusses our vision about symbolic evaluation of this language.

The \TOPHAT language consists two parts, the host language and the task language.
Programs in \TOPHAT are called \emph{tasks}.
The basic elements of tasks are editors.
Using combinators, tasks can be combined into larger tasks.

The task language is embedded in a simply typed lambda calculus with references, conditionals, booleans, integers, strings, pairs, lists and unary and binary operations on these types.
References allow tasks to communicate with each other, sharing information across task boundaries.
The full syntax of the host language is listed in \cref{sec:language}.
Next, we discuss the main constructs of the task language.


\subsection{Editors}

Editors are the basic method for programs to communicate with the outside world.
They are an abstraction over widgets in a \GUI library or on webpage forms.
Users can change the value held by an editor, in the same way they can manipulate widgets in a \GUI.

When a \TOP implementation generates an application from a task specification, it derives user interfaces for the editors.
The appearance of an editor is influenced by its type.
For example, an editor for a string can be represented by a simple input field, a date by a calendar, and a location by a pin on a map.

There are three different editors in \TOPHAT.
\begin{description}
  \item[$\Edit v$] Valued editor.\\
    This editor holds a value $v$ of a certain type.
    The user can replace the value by a new value of the same type.
  \item[$\Enter \tau$] Unvalued editor.\\
    This editor holds no value, and can receive a value of type $\tau$.
    When that happens, it turns into a valued editor.
  \item[$\Update l$] Shared editor.\\
    This editor refers to a store location $l$.
    Its observable value is the value stored at that location.
    When it receives a new value, this value will be stored at location $l$.
\end{description}



\subsection{Combinators}

Editors can be combined into larger tasks using combinators.
Combinators describe the way people collaborate.
Tasks can be performed in sequence or in parallel, or there is a choice between two tasks.

The following combinators are available in \TOPHAT.
\begin{description}
  \item[$t \Then e$] Step.\\
    Users can work on task $t$.
    As soon as $t$ has a value, that value is passed on to the right hand side $e$, with which it continues.
  \item[$t \Next e$] User Step.\\
    Users can work on task $t$.
    When $t$ has a value, the step becomes enabled.
    Users can then send a continue event to the combinator.
    When that happens, the value of $t$ is passed to the right hand side, with which it continues.
  \item[$t_1 \And t_1$] Composition.\\
    Users can work on tasks $t_1$ and $t_2$ in parallel.
  \item[$t_1 \Or t_2$] Choice.\\
    The system chooses between $t_1$ or $t_2$,
    based on which task first has a value.
    If both tasks have a value, the system chooses the left one.
  \item[$e_1 \Xor e_1$] User choice.\\
    A user has to make a choice between either the left or the right hand side.
    The user continues to work on the chosen task.
\end{description}

In addition to editors and combinators, \TOPHAT also contains the fail task ($\Fail$).
Programmers can use this task to indicate that a task is not reachable or viable.
For example, when the right hand side of a step combinator is $\Fail$, the step will not proceed to that task.



\subsection{Observations}

Several observations can be made on tasks.
Using the value function $\Value$, the current value of a task can be determined.
The value function is a partial function, since not all tasks have a value.
For example empty editors and steps do not have a value.

One can also observe whether or not a task is failing, by means of the failing function $\Failing$.
The task $\Fail$ is failing, as is a parallel combination of failing tasks ($\Fail \And \Fail$).

The step combinator makes use of both functions in order to determine if it can step.
First, it uses $\Value$ to see if the left hand side produces a value.
If that is the case, it uses the $\Failing$ function to see if it is safe to step to the right hand side.
The complete definition of the value and failing function are discussed in \cref{subsec:observations}.
