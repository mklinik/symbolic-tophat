% !TEX root=../main.tex



\section{Intuition}
\label{sec:intuition}

In order to verify that a particular \TOPHAT program behaves as intended,
we would like to show that certain relations between the input and output of tasks hold.
A common way to do this is to write test cases manually,
or to generate random input and verify that the outcome fulfils the desired relation.

Writing tests manually is labour intensive and cumbersome.
With random testing, there is no guarantee that all paths will be covered.

To overcome these issues, we apply symbolic execution to prove that the desired properties hold.
The idea of symbolic execution is simple.
Instead of executing tasks with test input, we run them on symbolic input.
This input consists of symbols that are allowed to be any value of the desired type.
When the task branches, we record what conditions must hold over the symbolic input.

When a task has an observable value, we consider it done.
This is the point where symbolic execution terminates.

This execution results in a list of all possible program results,
guarded by certain constraints on the symbolic input.
If these possible outcomes agree with the stated property,
it is guaranteed that his will hold for all paths.

\begin{example}
Consider the following example program.

\begin{TASK}
  enter String >>= \ x if palindrome x then edit x else fail
\end{TASK}

It asks the user to input a value of type $\String$.
This value is then passed on to the right hand side.
If the value is a palindrome, as determined by the function with that name,
an editor containing the entered value is returned.
Otherwise, the user is asked to input a different value, until the condition holds.

Imagine we would like to prove that no matter what value is given as input,
the only possible outcome value we can get is a palindrome.

The symbolic execution enters a symbol $s$ as input into the editor on the left.
The execution then arrives at a branching point.
In order to take the then-branch, the condition $\text{palindrome}\ s$ needs to hold.
This branch will then result in $\Edit s$.
We can now observe the task value, $s$, which is guarded by $\text{palindrome}\ s$.
The else branch applies if the condition does not hold.
But this does not yield a value, it instead asks for additional input.

So, from the symbolic execution we obtain that the only outcome of this program is $s \mid \text{palindrome}\ s$.
From that outcome, we can conclude that no matter what input is given, the only result value possible has to be a palindrome.
\end{example}
