%% For double-blind review submission, w/o CCS and ACM Reference (max submission space)
%\documentclass[acmsmall,review,anonymous]{acmart}\settopmatter{printfolios=true,printccs=false,printacmref=false}
%% For double-blind review submission, w/ CCS and ACM Reference
% \documentclass[acmsmall,review,anonymous]{acmart}\settopmatter{printfolios=true}
%% For single-blind review submission, w/o CCS and ACM Reference (max submission space)
% \documentclass[acmsmall,review]{acmart}\settopmatter{printfolios=true,printccs=false,printacmref=false}
%% For single-blind review submission, w/ CCS and ACM Reference
% \documentclass[acmsmall,review]{acmart}\settopmatter{printfolios=true}
%% For author draft version, w/o CCS and ACM Reference (max submission space)
% \documentclass[sigconf]{acmart}\settopmatter{printfolios=true,printccs=false,printacmref=false}
%% For final camera-ready submission, w/ required CCS and ACM Reference
%\documentclass[acmsmall]{acmart}\settopmatter{}
\documentclass[sigconf]{acmart}


% !TEX root=../main.tex


%% Basics %%%%%%%%%%%%%%%%%%%%%%%%%%%%%%%%%%%%%%%%%%%%%%%%%%%%%%%%%%%%%%%%%%%%%%

%% Fixes %%

\usepackage{underscore}


%% Fonts %%

\usepackage[utf8]{inputenc}
% \usepackage[T1]{fontenc}
%%NOTE: T1 doesn't have the `Th` and `Qu` ligatures :-(
\usepackage[OT1]{fontenc}

\usepackage{stmaryrd}
\usepackage{mathtools}
\usepackage{eurosym}

\usepackage{amsthm} %%NOTE: here because defines \openbox which will also be defined by newtxmath...
\usepackage{xcolor}


% \usepackage{tgpagella}
% \usepackage{lucidabr}

\usepackage{libertine}
\usepackage[varqu]{zi4}
\usepackage[libertine]{newtxmath}


%% Programming %%

\usepackage{xargs}
\usepackage{ifthen}


%% Layout %%

% \usepackage{microtype}
\usepackage{xspace}


%% Additions %%%%%%%%%%%%%%%%%%%%%%%%%%%%%%%%%%%%%%%%%%%%%%%%%%%%%%%%%%%%%%%%%%%

%% Textual %%

% \usepackage{titlesec}
\usepackage{paralist}
\usepackage{quoting}


%% Maths %%

\usepackage{amsmath}
\usepackage[retainorgcmds]{IEEEtrantools}


%% Graphics %%

\usepackage{graphicx}
\usepackage{xcolor}
\usepackage{dblfloatfix}
\usepackage[export]{adjustbox}
% \usepackage[xcolor]{mdframed}
\usepackage{tikz}
\usetikzlibrary{trees}

%% Tabulations %%

\usepackage{booktabs}
\usepackage{array}


%% Listings %%

\usepackage[final]{listings}


%% References & Bibliography %%

\usepackage[capitalize]{cleveref}
\usepackage{natbib}
% \usepackage[natbibapa,nodoi]{apacite}

% !TEX root=../main.tex


%% Fixes %%

\frenchspacing

% \newlength{\hugeskipamount}
% \setlength{\hugeskipamount}  {1.2500\baselineskip plus 0.3750\baselineskip minus 0.3750\baselineskip}
% \setlength{\bigskipamount}   {0.7500\baselineskip plus 0.2500\baselineskip minus 0.2500\baselineskip}
% \setlength{\medskipamount}   {0.3750\baselineskip plus 0.1250\baselineskip minus 0.1250\baselineskip}
% \setlength{\smallskipamount} {0.1875\baselineskip plus 0.0625\baselineskip minus 0.0625\baselineskip}

% \widowpenalty=150
% \clubpenalty=150


%% Section spacing %%
%%NOTE: requires 'titlesec'

% \titlespacing*{\section}{0pt}{\hugeskipamount}{\bigskipamount}
% \titlespacing*{\subsection}{0pt}{\bigskipamount}{\medskipamount}
% \titlespacing*{\paragraph}{0pt}{\medskipamount}{1em}


%% Math spacing %%

\setlength{\abovedisplayskip}{\smallskipamount}
\setlength{\belowdisplayskip}{\smallskipamount}

% \setlength{\topsep}{\smallskipamount}



%% Float spacing %%

\setlength{\abovecaptionskip}{\medskipamount}
\setlength{\floatsep}        {\medskipamount}
\setlength{\textfloatsep}    {\bigskipamount}
\setlength{\intextsep}       {\bigskipamount}
\setlength{\dblfloatsep}     {\medskipamount}
\setlength{\dbltextfloatsep} {\bigskipamount}


%% List spacing %%
%% NOTE: requires `paralist`

\setlength{\pltopsep}   {\medskipamount}
\setlength{\plpartopsep}{\parskip}
\setlength{\plitemsep}  {\parskip}
\setlength{\plparsep}   {\parskip}


%% Listings spacing %%

% \lstset
%   {aboveskip=\smallskipamount
%   ,belowskip=\smallskipamount
%   }


%% Tabular strech %%
%% NOTE: requires `array`

% \renewmacro{arraystretch}
%   {1.1}


%% Quoting %%
%%NOTE: requires 'quoting'

\quotingsetup
  {font=itshape
  ,leftmargin=\parindent
  ,listvskip}


%% Boxes %%

% \mdfsetup
%   {hidealllines=true
%   ,backgroundcolor=lightgray
%   }

% !TEX root=../main.tex


\input macros/auxiliaries


%% Fixes %%%%%%%%%%%%%%%%%%%%%%%%%%%%%%%%%%%%%%%%%%%%%%%%%%%%%%%%%%%%%%%%%%%%%%%

\let\texttilde\textasciitilde


%% Text %%%%%%%%%%%%%%%%%%%%%%%%%%%%%%%%%%%%%%%%%%%%%%%%%%%%%%%%%%%%%%%%%%%%%%%%

\providemacro{marginnote}
  {\marginpar}
\providemacro{smallcaps}
  {\textsc}
\providemacro{marginwidth}
  {\marginparwidth}

\newmacro{alert}[1]
  {\textbf{#1}}
\newmacro{divert}[1]
  {\textcolor{gray}{#1}}
\newmacro{enquote}[1]
  {``#1''}
\newmacro{fixme}[1]
  {\colorbox{yellow}{#1}\marginnote{\colorbox{yellow}{$\star$}}}
\newmacro{todo}[1]
  {\textcolor{red}{$\star$}\marginnote{\textcolor{red}{#1}}}
  % {}
\newmacro{type}[1]
  {\texttt{#1}}

\newmacro{add}[1]
  {\textcolor{green}{#1}}
\newmacro{remove}[1]
  {\textcolor{red}{#1}}
\newmacro{change}[1]
  {\textcolor{orange}{#1}}
\newmacro{adjust}[2]
  {\remove{#1} \add{#2}}

\newenvironment{fadeout}
  {\color{gray}}
  {}
\newenvironment{emphasize}
  {\begin{quote}\itshape}
  {\end{quote}}
\newenvironment{margintext}[1]
  {\begin{marginfigure}
     \subsection*{#1}}
  {\end{marginfigure}}


%% Lists %%
%% NOTE: requires `paralist`

%% Use compact lists by default
\renewenvironment{itemize}
  {\begin{compactitem}}
  {\end{compactitem}}
\renewenvironment{enumerate}
  {\begin{compactenum}}
  {\end{compactenum}}
\renewenvironment{description}
  {\begin{compactdesc}}
  {\end{compactdesc}}
%% Define starred versions as in-paragraph-lists
\newenvironment{itemize*}
  {\begin{inparaitem}}
  {\end{inparaitem}}
\newenvironment{enumerate*}[1][1=(i)]
  {\begin{inparaenum}[#1]}
  {\end{inparaenum}}
\newenvironment{description*}
  {\begin{inparadesc}}
  {\end{inparadesc}}


%% Quotations %%
%% NOTE: requires `quoting`

\let\quote\quoting
\let\endquote\endquoting
\renewenvironment{quotation}
  {\ClassError{Please use the `quote` environment instead of `quotation`}}


%% Column types %%
%% NOTE: requires `array`

\newcolumntype{L}{>{$}l<{$}}
\newcolumntype{C}{>{$}c<{$}}
\newcolumntype{R}{>{$}r<{$}}
\newcolumntype{T}{>{\ttfamily}l}
\newcolumntype{S}{>{\sffamily}l}


%% References %%
%%NOTE: requires `cleveref`

\let\refer\cref
\let\Refer\Cref


%% Citations %%
%% NOTE: requires `natbib`

\let\cite\citep
\let\Cite\Citep
\let\textcite\citet
\let\Textcite\Citet


%% Blocks and Boxes %%

\newenvironment{block}
  {\begin{center}}
  {\end{center}}
% \newenvironment{box}
%   {\begin{mdframed}}
%   {\end{mdframed}}


%% Logos %%

%% \newlogo[.name.]{.text.}
\newmacro{newlogo}[2][1]
  {\ifthenelse{\isempty{#1}}
     {\newlogoaux{#2}{\smallcaps{\lowercase{#2}}}}
     {\newlogoaux{#1}{#2}}}
\newmacro{newlogoaux}[2]
  {\newmacro{#1}{#2\xspace}}


%% Languages %%%%%%%%%%%%%%%%%%%%%%%%%%%%%%%%%%%%%%%%%%%%%%%%%%%%%%%%%%%%%%%%%%%

%%NOTE: `\mathrel` gives a single space width between keywords but removes it after another relational operator.
%%      `\mathop`  gives just a small skip, but doesn't has above bug.
\newmacro{newoperator}[1]
  {\newmathcommand{#1}[op]}
\newmacro{newkeyword}[2][1]
  %%FIXME: this is to complicated: {\newoperator{\ifthenelse{\isempty{#1}}{#2}{#1}}{\text{\sffamily\bfseries #2}}}
  {\ifthenelse{\isempty{#1}}
    {\newoperator{#2}{\text{\normalfont\sffamily\bfseries #2}}}
    {\newoperator{#1}{\text{\normalfont\sffamily\bfseries #2}}}}
\newmacro{newvalue}[2][1]
  {\ifthenelse{\isempty{#1}}
    {\newoperator{#2}{\text{\normalfont\sffamily #2}}}
    {\newoperator{#1}{\text{\normalfont\sffamily #2}}}}
\newmacro{newtype}[2][1]
  {\ifthenelse{\isempty{#1}}
    {\newoperator{#2}{\text{\normalfont\sffamily\scshape #2}}}
    {\newoperator{#1}{\text{\normalfont\sffamily\scshape #2}}}}


%% Math %%%%%%%%%%%%%%%%%%%%%%%%%%%%%%%%%%%%%%%%%%%%%%%%%%%%%%%%%%%%%%%%%%%%%%%%

%% Boxes %%

\newmacro{obox}[2]
  {\makebox[0pt][l]{\ensuremath{#2}}\phantom{\ensuremath{#1}}}

\newmacro{highlight}[1]
  {\colorbox{lightgray}{\ensuremath{#1}}}

%% Spacing %%

\newmacro{Quad}
  {\hspace{1.5em}}
\newmacro{Break}
  {\\[\smallskipamount]}



%% Fractions %%

\newmacro{upon}
  {\genfrac{}{}{0pt}{0}}



%% Symbols %%

%% NOTE: change this to \emptyset when using a font that includes a nice standard emptyset
\let\nothing\varnothing



%% Braces %%

\let\<\langle
\let\>\rangle

\newmathcommand{llbrace}[open] {\{\!|}
\newmathcommand{rrbrace}[close]{|\!\}}

\newmacro{set}[1]
  {\ensuremath{\{#1\}}}
\newmacro{tuple}[1]
  {\ensuremath{\<#1\>}}


%% Operators %%

\let\lt<
\let\gt>
\let\To\Rightarrow

\newmathcommand{pp}[bin]
  {+\!\!+}
\newmathcommand{Mid}
  {\;\mid\;}


%% Shortcuts %%

\newmacro{powerset}[1]
  % {2^{#1}}
  {\mathcal{P}(#1)}

\newmathcommand{n}{\underline{n}}

\newmathcommand{NN}  [bb]{N}
\newmathcommand{ZZ}  [bb]{Z}
\newmathcommand{EE}  [bb]{E}
\newmathcommand{OO}  [bb]{O}
\newmathcommand{QQ}  [bb]{A}
\newmathcommand{RR}  [bb]{R}
\newmathcommand{CC}  [bb]{C}
\newmathcommand{HH}  [bb]{H}

\newmathcommand{LL}  [bb]{L}
\newmathcommand{UU}  [bb]{U}
\newmathcommand{BB}  [bb]{B}
\renewmathcommand{SS}[bb]{S}

\let\to\rightarrow
% \let\implies\Rightarrow
% \let\implies\Longrightarrow
\let\implies\subset
\let\infers\vdash


%% Hints and local definitions %%

\newmacro{hint}[1]
  {\quad\text{\{ #1 \}}}

\newmathcommand{when}[op]
  {\mathbf{when}}
\newmathcommand{where}[op]
  {\mathbf{where}}
\renewmathcommand{and}[op]
  {\mathbf{and}}
\newmathcommand{otherwise}[op]
  {\mathbf{otherwise}}
\newmathcommand{impossible}[op]
  {\mathrm{impossible}}


%% Environments %%

\let\group\begingroup

\newenvironment*{marginequation}[1][1=0pt]
  {\begin{marginfigure}[#1]\equation}
  {\endequation\end{marginfigure}}

\newenvironment*{marginequation*}[1][1=0pt]
  {\begin{marginfigure}[#1]\equation\nonumber}
  {\endequation\end{marginfigure}}


\newenvironment*{function}
  {\begin{tabular}{@{}L@{\ \ }C@{\ \ }L@{}}}
  {\end{tabular}}
\newmacro{signature}[1]
  {\multicolumn{3}{@{}L@{}}{#1}}
\newmacro{inset}[1]
  {\multicolumn{3}{L}{\quad #1}}


\newenvironment*{grammar}
  %%NOTE: the `@{}` suppreses `\tabcolsep` before the first column
  {\begin{block}\begin{tabular}{@{}rRCLl}}
  {\end{tabular}\end{block}}
\newenvironment*{grammar*}
  %%NOTE: the `@{}` suppreses `\tabcolsep` before the first column
  {\begin{block}\begin{tabular}{@{}RLl}}
  {\end{tabular}\end{block}}



%% Theorems %%

% \newtheoremstyle{plain}%
%   {\medskipamount}% space above
%   {\medskipamount}% space below
%   {\itshape}% body font
%   {0pt}% indent amount
%   {\bfseries}% head font
%   {.}% punctuation after head
%   {.5em}% spacing after head
%   {\thmname{#1}\thmnumber{ #2}\thmnote{ {\normalfont(#3)}}}% head spec
% \newtheoremstyle{definition}%
%   {\medskipamount}% space above
%   {\medskipamount}% space below
%   {\normalfont}% body font
%   {0pt}% indent amount
%   {\bfseries}% head font
%   {.}% punctuation after head
%   {.5em}% spacing after head
%   {\thmname{#1}\thmnumber{ #2}\thmnote{ {\normalfont(#3)}}}% head spec
%
%
\theoremstyle{acmplain}

\newtheorem{theorem}{Theorem}[section]
\newtheorem{conjecture}[theorem]{Conjecture}
\newtheorem{proposition}[theorem]{Proposition}
\newtheorem{lemma}[theorem]{Lemma}
\newtheorem{corollary}[theorem]{Corollary}


\theoremstyle{acmdefinition}

\newtheorem{example}[theorem]{Example}
\newtheorem{definition}[theorem]{Definition}



%% Inference rules %%

\newmacro{placerule}[4][1,4]
  {\ensuremath{
    \upon
      {\text{\smallcaps{#1}}\hfill}
      {\dfrac{#2}{#3}\ #4}
  }}

\newmacro{newrule}[4][4]
  {\newmacro{#1}{\placerule[#1]{#2}{#3}[#4]}}
\newmacro{userule}
  {\usemacro}
\newmacro{refrule}[1]
  {\ifthenelse{\isundefined{#1}}
    {\GenericError{}{Rule `#1` is not defined}{}{}}
    {\textsc{#1}}}
% \newmacro{refrule}
%   {\textsc}

% !TEX root=../main.tex


%% Styles %%%%%%%%%%%%%%%%%%%%%%%%%%%%%%%%%%%%%%%%%%%%%%%%%%%%%%%%%%%%%%%%%%%%%%

\lstdefinestyle{natural}
  {columns=fullflexible
  ,gobble=2
  ,breaklines=true
  ,breakatwhitespace=true
  ,literate=
    %{.}{{$\cdot$}}1
    %{.}{{\ }}1
    {<<}{{$\<$}}1
    {>>}{{$\>$}}1
    {->}{{$\to$\ }}2
    % {--}{{--}}1
    %{_}{{\ }}1
    %{\ "}{{\ \textquotedblleft}}2
    %{"\ }{{\textquotedblright\ }}2
  ,basicstyle={\sffamily}
  ,keywordstyle=[1]{\bfseries}
  ,keywordstyle=[2]{\scshape}
  ,keywordstyle=[3]{}
  %,commentstyle={\itshape}
  %,identifierstyle={\itshape}
  ,emphstyle={\itshape}
  %,stringstyle={\rmfamily}
  ,showstringspaces=false
  ,texcl=true
  ,mathescape=true
  %,escapechar=\$
  %,escapeinside={\{\}}
  ,xleftmargin=1\parindent
  }

\lstdefinestyle{flexible}
  {columns=flexible
  ,gobble=2
  ,fontadjust=true
  ,basicstyle={\ttfamily\small}
  ,commentstyle={\itshape}
  ,keywordstyle={\bfseries}
  %,identifierstyle={\itshape}
  %,stringstyle={\ttfamily}
  ,emphstyle={\itshape}
  ,showstringspaces=false
  ,texcl=true
  ,mathescape=true
  %,escapechar=\$
  %,escapeinside={\{\}}
  ,xleftmargin=1\parindent
  }

\lstdefinestyle{literate}
  {style=natural
  ,literate=
    {\\}{{$\lambda$}}1
    {\\\$}{{\$}}1 %NOTE: otherwise eaten by `\`, NOTE: prevents \$ to be parsed as math escape
    {\\/}{{$\vee$}}1
    {/\\}{{$\wedge$}}1
    {A.}{{$\forall$}}1
    {E.}{{$\exist$}}1
    {->}{{$\rightarrow$ }}1
    {<-}{{$\leftarrow$}}1
    {==}{{$\equiv$\ }}1
    {/=}{{$\nequiv$\ }}1
    {<=}{{$\leq$}}1
    {>=}{{$\geq$}}1
    {>>=}{{>>=}}3 %NOTE: otherwise eaten by `>=`
    {\{|}{{$\{\!|\!$}}1
    {|\}}{{$\!|\!\}$}}1
    {\{|*|\}}{{$\{\!|\!\!\star\!\!|\!\}$}}3
  }


%% Definitions %%%%%%%%%%%%%%%%%%%%%%%%%%%%%%%%%%%%%%%%%%%%%%%%%%%%%%%%%%%%%%%%%

%% Tasks %%

\lstdefinelanguage{tasks}
  {sensitive=true
  ,morekeywords=[1]{let,in,if,then,else,case,of,ref,type}
  ,morekeywords=[2]{Bool,Int,String,Unit,List, Ref,Task, Passenger,Seat,Booking, Snack}
  ,moreemph={a,b,c,d,e,f,g,h,i,j,k,l,m,n,o,p,q,r,s,t,u,v,w,x,y,z as,bs,cs,ds,es,fs,gs,hs,is,js,ks,ls,ms,ns,os,ps,qs,rs,ss,ts,us,vs,ws,xs,ys,zs}
  ,morestring=[b]"
  ,morecomment=[l]--
  ,morecomment=[n]{\{-}{-\}}
  }[keywords,strings,comments]
\lstdefinestyle{tasks}
  {style=natural
  ,literate=
    {\\}{{$\lambda$}}1
    {<<}{{$\<$}}1
    {>>}{{$\>$ }}1
    {->}{{$\to$ }}1
    {==}{{$\equiv$ }}1
    {/=}{{$\nequiv$ }}1
    {<=}{{$\leq$ }}1
    {>=}{{$\geq$ }}1
    {*}{{$\times$ }}1
    {`elem`}{{$\in$ }}1
    {\\/}{{$\vee$ }}1
    {/\\}{{$\wedge$ }}1
    {>>=}{{$\Then$ }}1
    {>>?}{{$\Next$ }}1
    {<&>}{{$\And$ }}1
    {<|>}{{$\Or$ }}1
    {<?>}{{$\Xor$ }}1
    {++}{{$\pp$ }}1
    {edit}{{$\Edit$}}1
    {enter}{{$\Enter$}}1
    {update}{{$\Update$}}1
    {fail}{{$\Fail$ }}1
  }

\lstnewenvironment{TASK}[1][]
  {\lstset{language=tasks,style=tasks,#1}}
  {}
\newmacro{TS}[1][1]
  {\lstinline[language=tasks,style=tasks,#1]}
\newmacro{includeTASK}[2][]
  {\lstinputlisting[language=tasks,style=tasks,#1]{#2}}


%% Flows %%

\lstdefinelanguage{flows}
  {sensitive=true
  ,morekeywords=[1]{module,where,define,using,as,yielding,share,holding,with,do,for,fork,then,when,next,done,on,and,or,not,readonly,writeonly,readwrite}
  ,morekeywords=[2]{Bool,Int,String,Shared,List, Date,Document,Photo, Citizen,Company,Declaration}
  ,morekeywords=[3]{True,False,Just,Nothing,List}
  ,morestring=[b]"
  ,morecomment=[l]--
  ,morecomment=[n]{\{-}{-\}}
  }[keywords,strings,comments]

% \lstMakeShortInline[language=flows,style=natural] | % |
\lstnewenvironment{FLOW}[1][]
  {\lstset{language=flows,style=natural,#1}}
  {}
\newmacro{FL}[1][1]
  {\lstinline[language=flows,style=natural,#1]}
\newmacro{includeFLOW}[2][]
  {\lstinputlisting[language=flows,style=natural,#1]{#2}}


%% Clean %%

\lstdefinelanguage{clean}
  {sensitive=true
  %,alsoletter={ABCDEFGHIJKLMNOPQRSTUVWXYZabcdefghijklmnopqrstuvwxyz_`}
  %,alsoletter={~!@\#$\%^\&*-+=?<>:|\\} %$
  ,morekeywords={from,definition,implementation,import,module,system,code,inline,if,case,of,let,let!,in,where,with,class,instance,generic,derive,dynamic,infix,infixl,infixr}
  ,morestring=[b]"
  ,morestring=[b]'
  ,morecomment=[l]//
  ,morecomment=[n]{/*}{*/}
  }[keywords,strings,comments]

\lstnewenvironment{CLEAN}[1][]
  {\lstset{language=clean,style=flexible,#1}}
  {}
\newmacro{CL}[1][1]
  {\lstinline[language=clean,style=flexible,#1]}
\newmacro{includeCLEAN}[2][1]
  {\lstinputlisting[language=clean,style=flexible,#1]{#2}}

% !TEX root=../main.tex


\newlogo[ITASKS]{iTasks}
\newlogo[MTASKS]{mTasks}
\newlogo{TOP}

\newlogo{BPMN}
\newlogo{BPEL}
\newlogo{UML}
\newlogo{WFN}
\newlogo{YAWL}
\newlogo{IFTTT}

\newlogo{CCS}
\newlogo{CSP}
\newlogo[PICALC]{$\pi$-calculus}

\newlogo[ESTEREL]{Esterel}
\newlogo[HIPHOP]{HopHop}
\newlogo{FRP}

\newlogo{HTML}
\newlogo{XML}
\newlogo{IOT}

\newlogo{DSL}
\newlogo{EDSL}
\newlogo{GUI}

\newlogo{STW}
\newlogo{NWO}

\newlogo{IO}
\newlogo{SML}
\newlogo{ML}
\newlogo{UI}
\newlogo{ID}

% !TEX root=../main.tex



%% Host language %%%%%%%%%%%%%%%%%%%%%%%%%%%%%%%%%%%%%%%%%%%%%%%%%%%%%%%%%%%%%%%


\newkeyword[IF]  {if}
\newkeyword[THEN]{then}
\newkeyword[ELSE]{else}

\newkeyword[Let]{let}
\newkeyword[In]{in}

\newkeyword[Ref] {ref}


\newmacro{If}[3]
  {\IF #1 \THEN #2 \ELSE #3}



%% Values %%


\newmathcommand{unit}{\<\>}


\newvalue{True}
\newvalue{False}
\newvalue[Not]{not}


\newmacro{str}[1]
  {\text{``#1''}}

\newoperator{Length}{\mathrm{length}}


\newvalue[Map]{map}
\newvalue[Fst]{fst}
\newvalue[Snd]{snd}
\newvalue[Assoc]{assoc}



%% Types %%


\newtype{Unit}
\newtype{Bool}
\newtype{Nat}
\newtype{Int}
\newtype{String}
\newtype[Reference]{Ref}
\newtype{Task}
\newtype{Maybe}

\newtype{Euro}



%% Object language %%%%%%%%%%%%%%%%%%%%%%%%%%%%%%%%%%%%%%%%%%%%%%%%%%%%%%%%%%%%%


\newmacro{TOPHAT}
  {$\widehat{\text{\smallcaps{top}}}$\xspace}


\let\And\relax
\newoperator{Then}  {\blacktriangleright}
\newoperator{Next}  {\vartriangleright}
\newoperator{And}   {\Join}
\newoperator{Or}    {\blacklozenge}
\newoperator{Xor}   {\lozenge}
\newoperator{Edit}  {\square}
\newoperator{View}  {\overline{\square}}
\newoperator{Enter} {\boxtimes}
\newoperator{Update}{\blacksquare}
\newoperator{Watch} {\overline{\blacksquare}}
\newoperator{Fail}  {\lightning}
\newoperator{At}    {@}

\newoperator{AndOr} {\DEPRECATED}



%% Events %%


\newvalue[Left]   {L}
\newvalue[Right]  {R}


\newvalue[Empty]   {E}
\newvalue[Continue]{C}
\newvalue[Pick]    {P}


\newvalue[First]  {F}
\newvalue[Second] {S}
\newvalue[Here]   {H}



%% Semantic functions %%%%%%%%%%%%%%%%%%%%%%%%%%%%%%%%%%%%%%%%%%%%%%%%%%%%%%%%%%


\newmathcommand{eval}[rel]
  {\;\downarrow\;}
\newmathcommand{stride}[rel]
  % {\;\rightarrow\!\shortmid\;}
  {\;\rightsquigarrow\;}
\newmathcommand{normalise}[rel]
  {\;\Downarrow\;}
\newmacro{handle}
  {\mathrel{\;\xrightarrow{}\;}}
\newmacro{drive}
  {\mathrel{\;\xRightarrow{}\;}}


\newmathcommand{Value}[cal]
  {V}
\newmathcommand{Inputs}[cal]
  {I}
\newmathcommand{Interface}[cal]
  {U}
\newmathcommand{Failing}[cal]
  {F}
\newmathcommand{Watching}[cal]
  {W}
\newmathcommand{Dirty}
  {\Delta}
\newmathcommand{UserInterface}[cal]
  {U}

%% Proofs %%%%%%%%%%%%%%%%%%%%%%%%%%%%%%%%%%%%%%%%%%%%%%%%%%%%%%%%%%%%%%%%%%%%%%


% \newcommand{\case}[2]{
%     \noindent\textbf{Case} #1\\
%     \vspace{5mm}
%     \indent\begin{minipage}{\dimexpr\textwidth-3cm}
%     #2
%   \end{minipage}\\\\}

\newcommand{\case}[2]{
  \bigskip
  \noindent\textbf{Case} #1
  \nopagebreak[4]
  \smallskip
  \par
  \begingroup
    \leftskip\parindent
    \noindent
    #2
    \par
  \endgroup
}

% \newcommand{\case}[2]{
%   \noindent
%   \begin{tabular*}{\textwidth}{lp{0.8\textwidth}}
%     \textbf{Case} & #1 \\
%     \addlinespace
%     & #2
%   \end{tabular*}
%   \medskip
% }



%% Depricated %%%%%%%%%%%%%%%%%%%%%%%%%%%%%%%%%%%%%%%%%%%%%%%%%%%%%%%%%%%%%%%%%%

\newvalue[Execute] {<depricated>}

% !TEX root=main.tex


%% Typing %%%%%%%%%%%%%%%%%%%%%%%%%%%%%%%%%%%%%%%%%%%%%%%%%%%%%%%%%%%%%%%%%%%%%%

\newrule{T-Sym}
  {s:\beta \in \Gamma}
  {\Gamma,\Sigma \infers s:\beta}


%% Evaluation %%%%%%%%%%%%%%%%%%%%%%%%%%%%%%%%%%%%%%%%%%%%%%%%%%%%%%%%%%%%%%%%%%


\newmacro{RelationSE}
  {e,\sigma \eval \overline{v,\sigma',\highlight{\phi}}}


\newrule{SE-Value}
  {}
  {v,\sigma\eval v,\highlight{\sigma,\True}}


\newrule{SE-App}
  {e_1,\sigma\eval \overline{\lambda x:\tau.e_1',\sigma',\highlight{\phi_1}} \Quad
   e_2,\sigma'\eval \overline{v_2,\sigma'',\highlight{\phi_2}} \Quad
   e_1'[x\mapsto v_2],\sigma''\eval \overline{v_1,\sigma''',\highlight{\phi_3}}}
  {e_1 e_2,\sigma \eval \overline{v_1,\sigma''',\highlight{\phi_1\land\phi_2\land\phi_3}}}

\newrule{SE-If}
  {e_1,\sigma\eval \overline{v_1,\sigma',\highlight{\phi_1}} \Quad
   \highlight{e_2,\sigma'\eval \overline{v_2,\sigma'',\phi_2}} \Quad
   \highlight{e_3,\sigma'\eval \overline{v_3,\sigma''',\phi_3}}}
  {\If{e_1}{e_2}{e_3},\sigma\eval \highlight{\overline{v_2,\sigma'',\phi_1 \land \phi_2\land v_1} \cup \overline{v_3,\sigma''',\phi_1 \land \phi_3 \land \lnot v_1}}}

\newrule{SE-Pair}
  {e_1,\sigma\eval \overline{v_1,\sigma',\highlight{\phi_1}} \Quad
   e_2,\sigma'\eval \overline{v_2,\sigma'',\highlight{\phi_2}}}
  {\tuple{e_1,e_2},\sigma\eval\overline{\tuple{v_1,v_2},\sigma'',\highlight{\phi_1\land\phi_2}}}

\newrule{SE-First}
  {e_1,\sigma\eval\overline{v_1,\sigma',\highlight{\phi}}}
  {\Fst\tuple{e_1,e_2},\sigma\eval\overline{v_1,\sigma',\highlight{\phi}} }

\newrule{SE-Second}
  {e_2,\sigma\eval\overline{v_2,\sigma',\highlight{\phi}}}
  {\Snd\tuple{e_1,e_2},\sigma\eval\overline{v_2,\sigma',\highlight{\phi}} }


%%%%%%%

\newrule{SE-Cons}
  {e_1,\sigma \eval \overline{v_1,\sigma',\highlight{\phi_1}}\Quad
   e_2,\sigma' \eval \overline{v_2,\sigma'',\highlight{\phi_2}}}
  {e_1 :: e_2,\sigma \eval \overline{v_1:: v_2,\sigma'',\highlight{\phi_1\land\phi_2}}}

\newrule{SE-Head}
  {e,\sigma \eval \overline{v_1::v_2,\sigma',\highlight{\phi}}}
  {\Head e,\sigma \eval \overline{v_1,\sigma',\highlight{\phi}}}

\newrule{SE-Tail}
{e,\sigma \eval \overline{v_1::v_2,\sigma',\highlight{\phi}}}
{\Tail e,\sigma \eval \overline{v_2,\sigma',\highlight{\phi}}}


%%%%%
\newrule{SE-Ref}
  {e,\sigma\eval \overline{v,\sigma',\highlight{\phi}} \Quad
   l\not\in Dom(\sigma')}
  {\Ref e,\sigma\eval \overline{l,\sigma'[l\mapsto v],\highlight{\phi}}}

\newrule{SE-Deref}
  {e,\sigma\eval \overline{l,\sigma',\highlight{\phi}}}
  {!e,\sigma\eval \overline{\sigma'(l),\sigma',\highlight{\phi}}}

\newrule{SE-Assign}
  {e_1,\sigma\eval \overline{l,\sigma',\highlight{\phi_1}} \Quad
   e_2,\sigma'\eval \overline{v_2,\sigma'',\highlight{\phi_2}}}
  {e_1:=e_2,\sigma\eval \overline{\unit,\sigma''[l\mapsto v_2],\highlight{\phi_1\wedge\phi_2}}}

\newrule{SE-Edit}
  {e,\sigma \eval \overline{v,\sigma',\highlight{\phi}}}
  {\Edit e , \sigma\eval \overline{\Edit v,\sigma',\highlight{\phi}}}

\newrule{SE-Enter}
  {}
  {\Enter \tau,\sigma \eval \Enter \tau,\sigma,\highlight{\True}}

\newrule{SE-Update}
  {e,\sigma\eval \overline{l,\sigma',\highlight{\phi}}}
  {\Update e ,\sigma\eval \overline{\Update l,\sigma',\highlight{\phi}}}


\newrule{SE-Fail}
  {}
  {\Fail,\sigma \eval \Fail,\sigma,\highlight{\True}}


\newrule{SE-Then}
  {e_1 ,\sigma\eval \overline{t_1,\sigma',\highlight{\phi}}}
  {e_1 \Then e_2,\sigma \eval \overline{t_1 \Then e_2,\sigma',\highlight{\phi}}}

\newrule{SE-Next}
  {e_1 ,\sigma\eval \overline{t_1,\sigma',\highlight{\phi}}}
  {e_1 \Next e_2 ,\sigma\eval \overline{t_1 \Next e_2,\sigma',\highlight{\phi}}}


\newrule{SE-And}
  {e_1 ,\sigma\eval \overline{t_1 ,\sigma',\highlight{\phi_1}} \Quad
   e_2 ,\sigma'\eval \overline{t_2,\sigma'',\highlight{\phi_2}}}
  {e_1 \And e_2 ,\sigma\eval \overline{t_1 \And t_2,\sigma'',\highlight{\phi_1\land\phi_2}}}


\newrule{SE-Or}
  {e_1 ,\sigma\eval \overline{t_1 ,\sigma',\highlight{\phi_1}} \Quad
   e_2 ,\sigma'\eval \overline{t_2,\sigma'',\highlight{\phi_2}}}
  {e_1 \Or e_2 ,\sigma\eval \overline{t_1 \Or t_2,\sigma'',\highlight{\phi_1\land\phi_2}}}

\newrule{SE-Xor}
  {}
  {e_1 \Xor e_2 ,\sigma\eval e_1 \Xor e_2,\sigma,\highlight{\True}}

%% Normalisation %%%%%%%%%%%%%%%%%%%%%%%%%%%%%%%%%%%%%%%%%%%%%%%%%%%%%%%%%%%%%%%


\newmacro{RelationSS}
  {t,\sigma\stride \overline{t',\sigma',\highlight{\phi}}}


\newrule{SS-Edit}
  { }
  {\Edit v,\sigma \stride \Edit v,\sigma,\highlight{\True}}

\newrule{SS-Fill}
  { }
  {\Enter \tau,\sigma \stride \Enter \tau,\sigma,\highlight{\True}}

\newrule{SS-Update}
  { }
  {\Update l,\sigma \stride \Update l,\sigma,\highlight{\True}}


\newrule{SS-Fail}
  { }
  {\Fail,\sigma \stride \Fail,\sigma,\highlight{\True}}


\newrule{SS-ThenStay}
  {t_1,\sigma \stride \overline{t_1',\sigma',\highlight{\phi}}}
  {t_1 \Then e_2,\sigma \stride \overline{t_1' \Then e_2,\sigma',\highlight{\phi}}}
  [\Value(t_1',\sigma') = \bot]

\newrule{SS-ThenFail}
  {t_1,\sigma \stride \overline{t_1',\sigma',\highlight{\phi}} \Quad
   e_2\ v_1,\sigma' \eval \overline{t_2,\sigma'',\_}}
  {t_1 \Then e_2,\sigma \stride \overline{t_1' \Then e_2,\sigma',\highlight{\phi}}}
  [\Value(t_1',\sigma') = v_1 \land \Failing(t_2,\sigma'')]

\newrule{SS-ThenCont}
  {t_1,\sigma \stride \overline{t_1',\sigma',\highlight{\phi_1}} \Quad
   e_2\ v_1,\sigma' \eval \overline{t_2 ,\sigma'',\highlight{\phi_2}}}
   % t_2,\sigma'' \stride t_2',\sigma'''}
  {t_1 \Then e_2,\sigma \stride \overline{t_2,\sigma'',\highlight{\phi_1\land\phi_2}}}
  [\Value(t_1',\sigma') = v_1 \land \lnot\Failing(t_2,\sigma'')]

\newrule{SS-Next}
  {t_1,\sigma \stride \overline{t_1',\sigma',\highlight{\phi}}}
  {t_1 \Next e_2,\sigma \stride \overline{t_1' \Next e_2,\sigma',\highlight{\phi}}}


\newrule{SS-And}
  {t_1,\sigma  \stride \overline{t_1',\sigma',\highlight{\phi_1} } \Quad
   t_2,\sigma' \stride \overline{t_2',\sigma'',\highlight{\phi_2}}}
  {t_1 \And t_2,\sigma \stride \overline{t_1' \And t_2',\sigma'',\highlight{\phi_1\land\phi_2}}}


\newrule{SS-OrLeft}
  {t_1,\sigma  \stride \overline{t_1',\sigma',\highlight{\phi}}}
  {t_1 \Or t_2,\sigma \stride \overline{t_1',\sigma',\highlight{\phi}}}
  [\Value(t_1',\sigma') = v_1]

\newrule{SS-OrRight}
  {t_1,\sigma  \stride \overline{t_1',\sigma',\highlight{\phi_1}}  \Quad
   t_2,\sigma' \stride \overline{t_2',\sigma'',\highlight{\phi_2}}}
  {t_1 \Or t_2,\sigma \stride \overline{t_2',\sigma'',\highlight{\phi_1\land\phi_2}}}
  [\Value(t_1',\sigma') = \bot \land \Value(t_2',\sigma'') = v_2]

\newrule{SS-OrNone}
  {t_1,\sigma  \stride \overline{t_1',\sigma' ,\highlight{\phi_1}} \Quad
   t_2,\sigma' \stride \overline{t_2',\sigma'',\highlight{\phi_2}}}
  {t_1 \Or t_2,\sigma \stride \overline{t_1' \Or t_2',\sigma'',\highlight{\phi_1\land\phi_2}}}
  [\Value(t_1',\sigma') = \bot \land \Value(t_2',\sigma'') = \bot]


\newrule{SS-Xor}
  {\ }
  {e_1 \Xor e_2,\sigma \stride e_1 \Xor e_2,\sigma,\highlight{\True}}

\newrule{SS-Eval}
    {e,\sigma \eval \overline{e',\sigma',\highlight{\phi_1}}  \Quad
     e',\sigma' \stride \overline{e'',\sigma'',\highlight{\phi_2}}}
    {e,\sigma \stride \overline{e'',\sigma'',\highlight{\phi_1\land\phi_2}}}
    [e \neq e']

%% Normalisation %%


\newmacro{RelationSN}
  {e,\sigma \normalise \overline{t,\sigma',\highlight{\phi}}}


\newrule{SN-Done}
    {e,\sigma \eval \overline{t,\sigma',\highlight{\phi_1}}  \Quad
     t,\sigma' \stride \overline{t',\sigma'',\highlight{\phi_2}}}
    {e,\sigma \normalise \overline{t,\sigma',\highlight{\phi_1}}}
    [\sigma'=\sigma'' \land t=t']

\newrule{SN-Repeat}
    {\upon{e,\sigma \eval \overline{t,\sigma',\highlight{\phi_1}}}
     {{t,\sigma' \stride \overline{t',\sigma'',\highlight{\phi_2}}}
     {t',\sigma'' \normalise \overline{t'',\sigma''',\highlight{\phi_3}}}}}
    {e,\sigma \normalise \overline{t'',\sigma''',\highlight{\phi_1 \land \phi_2 \land \phi_3}}}
    [\sigma'\neq \sigma''\vee t\neq t']



%% Handling %%


\newmacro{RelationSH}
  {t,\sigma \handle{} \overline{t',\sigma',\highlight{i,\phi}}}


\newrule{SH-Change}
  { \text{fresh }s}
  {\Edit v,\sigma \handle{} \Edit s,\sigma,\highlight{s,\True}}
  [v,s:\tau]

% \newrule{SH-Empty}
%   { }
%   {\Edit v,\sigma \handle{\Empty} \Enter \tau,\highlight{\sigma,\True}}
%   [v : \tau]

\newrule{SH-Fill}
  { \text{fresh }s}
  {\Enter \tau,\sigma \handle{} \Edit s,\sigma,\highlight{s,\True}}
  [s:\tau]

\newrule{SH-Update}
  { \text{fresh }s}
  {\Update l,\sigma \handle{} \Update l,\sigma[l \mapsto s],\highlight{s,\True}}
  [\sigma(l),s:\tau]

\newrule{SH-PassThen}
  {t_1,\sigma \handle{} \overline{t_1',\sigma',\highlight{i,\phi}}}
  {t_1 \Then e_2,\sigma \handle{} \overline{t_1' \Then e_2,\sigma',\highlight{i,\phi}}}

\newrule{SH-PassNext}
  {t_1,\sigma \handle{} \overline{t_1',\sigma',\highlight{i,\phi}}}
  {t_1 \Next e_2,\sigma \handle{} \overline{t_1' \Next e_2,\sigma',\highlight{i,\phi}}}
  [\Value{(t_1,\sigma)} = \bot]

\newrule{SH-PassNextFail}
  {t_1,\sigma \handle{} \overline{t_1',\sigma_1',\highlight{i,\phi}} \Quad
  e_2\ v_1,\sigma \normalise \overline{t_2,\sigma_2',\highlight{\vphantom{i}\_}}}
  {t_1 \Next e_2,\sigma \handle{} \overline{t_1' \Next e_2,\sigma_1',\highlight{i,\phi}}}
  [\Value{(t_1,\sigma)} = v_1 \land \Failing{(t_2,\sigma_2')}]

\newrule{SH-Next}
  {t_1,\sigma \handle{} \overline{t_1',\sigma_1',\highlight{i_1,\phi_1}} \Quad
  e_2\ v_1,\sigma \normalise \overline{t_2,\sigma_2',\highlight{\phi_2}}}
  {t_1 \Next e_2,\sigma \handle{} \highlight{\overline{t_1' \Next e_2,\sigma_1',i_1,\phi_1}\cup\overline{t_2,\sigma_2',\Continue,\phi_2}}}
  [\Value{(t_1,\sigma)} = v_1 \land \neg\Failing{(t_2,\sigma')}]


\newrule{SH-And}
  {t_1,\sigma \handle{} \overline{t_1',\sigma_1',\highlight{i_1,\phi_1}} \Quad
   t_2,\sigma \handle{} \overline{t_2',\sigma_2',\highlight{i_2,\phi_2}}}
  {t_1 \And t_2,\sigma \handle{} \highlight{\overline{t_1' \And t_2,\sigma_1',\First i_1,\phi_1}\cup \overline{t_1 \And t_2',\sigma_2'',\Second i_2,\phi_2}}}

\newrule{SH-Or}
  {t_1,\sigma \handle{} \overline{t_1',\sigma_1',\highlight{i_1,\phi_1}}\Quad
  t_2,\sigma \handle{} \overline{t_2',\sigma_2',\highlight{i_2,\phi_2} }}
  {t_1 \Or t_2,\sigma \handle{} \highlight{\overline{t_1' \Or t_2,\sigma_1',\First i_1,\phi_1}\cup\overline{t_1 \Or t_2',\sigma_2',\Second i_2,\phi_2}}}


\newrule{SH-PickLeft}
  {e_1,\sigma \normalise \overline{t_1,\highlight{\sigma_1,\phi_1}} \Quad
   e_2,\sigma \normalise \overline{t_2,\highlight{\sigma_2,\phi_2}}}
  {e_1 \Xor e_2,\sigma \handle{} t_1,\sigma_1,\highlight{\Left,\phi_1}}
  [\neg\Failing(t_1,\sigma_1) \land \Failing(t_2,\sigma_2)]

\newrule{SH-PickRight}
  {e_1,\sigma \normalise \overline{t_1,\highlight{\sigma_1,\phi_1}} \Quad
   e_2,\sigma \normalise \overline{t_2,\highlight{\sigma_2,\phi_2}}}
  {e_1 \Xor e_2,\sigma \handle{} t_2,\sigma_2,\highlight{\Right,\phi_2}}
  [\Failing(t_1,\sigma_1) \land \neg\Failing(t_2,\sigma_2)]

\newrule{SH-Pick}
  {e_1,\sigma \normalise \overline{t_1,\highlight{\sigma_1,\phi_1}} \Quad
   e_2,\sigma \normalise \overline{t_2,\highlight{\sigma_2,\phi_2}}}
  {e_1 \Xor e_2,\sigma \handle{} \highlight{\overline{t_1,\sigma_1,\Left,\phi_1}\cup\overline{t_2,\sigma_2,\Right,\phi_2}}}
  [\neg\Failing(t_1,\sigma_1) \land \neg\Failing(t_2,\sigma_2)]


%% Driving %%


\newmacro{RelationSI}
  {t,\sigma \drive{} \overline{t',\sigma',\highlight{i,\phi}}}


\newrule{SI-Handle}
  {t,\sigma \handle{} \overline{t',\sigma',\highlight{i,\phi_1}} \Quad
   t',\sigma' \normalise \overline{t'',\sigma'',\highlight{\phi_2}}}
  {t,\sigma \drive{} \overline{t'',\sigma'',\highlight{i,\phi_1 \land \phi_2}}}

% !TEX root=main.tex


%% Typing %%%%%%%%%%%%%%%%%%%%%%%%%%%%%%%%%%%%%%%%%%%%%%%%%%%%%%%%%%%%%%%%%%%%%%


\newmacro{RelationT}
  {\Gamma,\Sigma \infers e : \tau}


\newrule{T-ConstBool}
  {c\in B}
  {\Gamma,\Sigma\infers c : \Bool}

\newrule{T-ConstInt}
  {c\in I}
  {\Gamma,\Sigma\infers c : \Int}

\newrule{T-ConstString}
  {c\in S}
  {\Gamma,\Sigma\infers c : \String}


\newrule{T-Unit}
  { }
  {\Gamma,\Sigma\infers \unit : \Unit}


\newrule{T-Var}
  {x:\tau\in\Gamma}
  {\Gamma,\Sigma\infers x:\tau}


\newrule{T-Abs}
  {\Gamma[x:\tau_1] ,\Sigma \infers e:\tau_2}
  {\Gamma,\Sigma \infers \lambda x : \tau_1 . e :\tau_1 \to \tau_2}

\newrule{T-App}
  {\Gamma,\Sigma \infers e_1:\tau_1\to\tau_2 \Quad
   \Gamma,\Sigma \infers e_2:\tau_1}
  {\Gamma,\Sigma \infers e_1 e_2 :\tau_2}


\newrule{T-If}
  {\Gamma,\Sigma \infers e_1:\Bool \Quad
   \Gamma,\Sigma \infers e_2:\tau \Quad
   \Gamma,\Sigma \infers e_3:\tau}
  {\Gamma,\Sigma \infers \If{e_1}{e_2}{e_3}:\tau}


\newrule{T-Pair}
    {\Gamma,\Sigma \infers e_1 : \tau_1  \Quad
     \Gamma,\Sigma \infers e_2 : \tau_2}
    {\Gamma,\Sigma \infers \tuple{e_1, e_2} :\tau_1 \times \tau_2}

\newrule{T-First}
  {\Gamma,\Sigma\infers e_1:\tau}
  {\Gamma,\Sigma\infers \Fst \tuple{e_1,e_2}:\tau}

  \newrule{T-Second}
    {\Gamma,\Sigma\infers e_2:\tau}
    {\Gamma,\Sigma\infers \Snd \tuple{e_1,e_2}:\tau}

%%%%%
\newrule{T-ListEmpty}
  { }
  {\Gamma,\Sigma\infers [\ ]_\beta : \List\beta}

\newrule{T-ListCons}
  {\Gamma,\Sigma\infers e_1:\beta \Quad
   \Gamma,\Sigma\infers e_2:\List\beta}
  {\Gamma,\Sigma\infers e_1 :: e_2 : \List \beta}

\newrule{T-ListHead}
  {\Gamma,\Sigma\infers e:\List\beta}
  {\Gamma,\Sigma\infers \Head e:\beta}

\newrule{T-ListTail}
    {\Gamma,\Sigma\infers e:\List\beta}
    {\Gamma,\Sigma\infers \Tail e:\List\beta}

%%%%%


\newrule{T-Ref}
  {\Gamma,\Sigma \infers e:\beta}
  {\Gamma,\Sigma \infers \Ref e :\Reference \beta}

\newrule{T-Deref}
  {\Gamma,\Sigma \infers e:\Reference \beta}
  {\Gamma,\Sigma\infers\ !e:\beta}

\newrule{T-Assign}
  {\Gamma,\Sigma\infers e_1:\Reference \beta \Quad
   \Gamma,\Sigma\infers e_2:\beta}
  {\Gamma,\Sigma\infers e_1 := e_2:\Unit}

\newrule{T-Loc}
  {\Sigma(l) = \beta}
  {\Gamma,\Sigma\infers l:\Reference \beta}


\newrule{T-Edit}
  {\Gamma,\Sigma \infers e : \tau}
  {\Gamma,\Sigma \infers \Edit e : \Task \tau}

\newrule{T-Enter}
  {}
  {\Gamma,\Sigma \infers \Enter \tau : \Task \tau}

\newrule{T-Update}
  {\Gamma,\Sigma \infers e : \Reference \beta}
  {\Gamma,\Sigma \infers \Update e : \Task \beta}


\newrule{T-Fail}
  {}
  {\Gamma,\Sigma \infers \Fail : \Task \tau}


\newrule{T-Then}
  {\upon{\Gamma,\Sigma \infers e_1 : \Task \tau_1}
   {\Gamma,\Sigma \infers e_2 : \tau_1 \to \Task \tau_2}}
  {\Gamma,\Sigma \infers e_1 \Then e_2 : \Task \tau_2}


\newrule{T-Next}
  {\upon{\Gamma,\Sigma \infers e_1 : \Task \tau_1}
   {\Gamma,\Sigma \infers e_2 : \tau_1 \to \Task \tau_2}}
  {\Gamma,\Sigma \infers e_1 \Next e_2 : \Task \tau_2}


\newrule{T-And}
  {\Gamma,\Sigma \infers e_1 : \Task \tau_1 \Quad
   \Gamma,\Sigma \infers e_2 : \Task \tau_2}
  {\Gamma,\Sigma \infers e_1 \And e_2 : \Task\,(\tau_1 \times \tau_2)}


\newrule{T-Or}
  {\upon{\Gamma,\Sigma \infers e_1 : \Task \tau}
   {\Gamma,\Sigma \infers e_2 : \Task \tau}}
  {\Gamma,\Sigma \infers e_1 \Or e_2 : \Task \tau}


\newrule{T-Xor}
  {\upon{\Gamma,\Sigma \infers e_1 : \Task \tau}
   {\Gamma,\Sigma \infers e_2 : \Task \tau}}
  {\Gamma,\Sigma \infers e_1 \Xor e_2 : \Task \tau}


%% Evaluation %%%%%%%%%%%%%%%%%%%%%%%%%%%%%%%%%%%%%%%%%%%%%%%%%%%%%%%%%%%%%%%%%%

\newmacro{RelationE}
  {e,\hat{\sigma} \hat{\eval} \hat{v},\hat{\sigma}'}


\newrule{E-Value}
  {}
  {v,\hat{\sigma}\hat{\eval} v,\hat{\sigma}}


\newrule{E-App}
  {e_1,\hat{\sigma}\hat{\eval} \lambda x:\tau.\hat{e_1}',\hat{\sigma}'\Quad
   e_2,\hat{\sigma}'\hat{\eval} \hat{v_2},\hat{\sigma}''\Quad
   e_1'[x\mapsto v_2],\hat{\sigma}''\hat{\eval} \hat{v_1},\hat{\sigma}'''}
  {e_1 e_2,\hat{\sigma} \hat{\eval} \hat{v_1},\hat{\sigma}'''}


\newrule{E-IfTrue}
  {e_1,\hat{\sigma}\hat{\eval} \True,\hat{\sigma}'\Quad
   e_2,\hat{\sigma}'\hat{\eval} \hat{v_2},\hat{\sigma}''}
  {\If{e_1}{e_2}{e_3},\hat{\sigma}\hat{\eval} \hat{v_2},\hat{\sigma}''}

\newrule{E-IfFalse}
  {e_1,\hat{\sigma}\hat{\eval} \False,\hat{\sigma}' \Quad
   e_3,\hat{\sigma}'\hat{\eval} \hat{v_3},\hat{\sigma}''}
  {\If{e_1}{e_2}{e_3},\hat{\sigma}\hat{\eval} \hat{v_3},\hat{\sigma}''}


\newrule{E-Pair}
  {e_1,\hat{\sigma}\hat{\eval} \hat{v_1},\hat{\sigma}' \Quad
   e_2,\hat{\sigma}'\hat{\eval} \hat{v_2},\hat{\sigma}''}
  {\tuple{e_1,e_2},\hat{\sigma}\hat{\eval}\tuple{\hat{v_1},\hat{v_2}},\hat{\sigma}''}

\newrule{E-First}
  {e_1,\hat{\sigma}\hat{\eval}\hat{v_1},\hat{\sigma}'}
  {\Fst\tuple{e_1,e_2},\hat{\sigma}\hat{\eval}\hat{v_1},\hat{\sigma}'}

\newrule{E-Second}
  {e_2,\hat{\sigma}\hat{\eval}\hat{v_2},\hat{\sigma}'}
  {\Snd\tuple{e_1,e_2},\hat{\sigma}\hat{\eval} \hat{v_2},\hat{\sigma}' }

%%%%%%%%%

\newrule{E-Cons}
  {e_1,\hat{\sigma}\hat{\eval}\hat{v_1},\hat{\sigma}'\Quad
   e_2,\hat{\sigma}'\hat{\eval}\hat{v_2},\hat{\sigma}''}
  {e_1 :: e_2,\hat{\sigma}\hat{\eval}\hat{v_1}::\hat{v_2},\hat{\sigma}''}

\newrule{E-Head}
  {e,\hat{\sigma}\hat{\eval} \hat{v_1}::\hat{v_2},\hat{\sigma}'}
  {\Head e,\hat{\sigma}\hat{\eval}\hat{v_1},\hat{\sigma}'}

\newrule{E-Tail}
{e,\hat{\sigma}\hat{\eval} \hat{v_1}::\hat{v_2},\hat{\sigma}'}
{\Tail e,\hat{\sigma}\hat{\eval}\hat{v_2},\hat{\sigma}'}

%%%%%%


\newrule{E-Ref}
  {e,\hat{\sigma}\hat{\eval} \hat{v},\hat{\sigma}' \Quad
   l\not\in Dom(\hat{\sigma}')}
  {\Ref e,\hat{\sigma}\hat{\eval} l,\hat{\sigma}'[l\mapsto \hat{v}]}

\newrule{E-Deref}
  {e,\hat{\sigma}\hat{\eval} l,\hat{\sigma}'}
  {!e,\hat{\sigma}\hat{\eval} \hat{\sigma}'(l),\hat{\sigma}'}

\newrule{E-Assign}
  {e_1,\hat{\sigma}\hat{\eval} l,\hat{\sigma}' \Quad
   e_2,\hat{\sigma}'\hat{\eval} \hat{v_2},\hat{\sigma}''}
  {e_1:=e_2,\hat{\sigma}\hat{\eval} \unit,\hat{\sigma}''[l\mapsto \hat{v_2}]}

\newrule{E-Edit}
  {e,\hat{\sigma} \hat{\eval} \hat{v},\hat{\sigma}'}
  {\Edit e , \hat{\sigma}\hat{\eval} \Edit \hat{v},\hat{\sigma}'}

\newrule{E-Enter}
  {}
  {\Enter \tau,\hat{\sigma} \hat{\eval} \Enter \tau,\hat{\sigma}}

\newrule{E-Update}
  {e,\hat{\sigma}\hat{\eval} l,\hat{\sigma}'}
  {\Update e ,\hat{\sigma}\hat{\eval} \Update l,\hat{\sigma}'}


\newrule{E-Fail}
  {}
  {\Fail,\hat{\sigma} \hat{\eval} \Fail,\hat{\sigma}}


\newrule{E-Then}
  {e_1 ,\hat{\sigma}\hat{\eval} \hat{t_1},\hat{\sigma}'}
  {e_1 \Then e_2,\hat{\sigma}\hat{\eval}\hat{t_1} \Then e_2,\hat{\sigma}'}

\newrule{E-Next}
  {e_1 ,\hat{\sigma}\hat{\eval} \hat{t_1},\hat{\sigma}'}
  {e_1 \Next e_2 ,\hat{\sigma}\hat{\eval} \hat{t_1} \Next e_2,\hat{\sigma}'}


\newrule{E-And}
  {e_1 ,\hat{\sigma}\hat{\eval}\hat{ t_1 },\hat{\sigma}'\Quad
   e_2 ,\hat{\sigma}'\hat{\eval} \hat{t_2},\hat{\sigma}''}
  {e_1 \And e_2 ,\hat{\sigma}\hat{\eval}\hat{ t_1} \And \hat{t_2},\hat{\sigma}''}


\newrule{E-Or}
  {e_1 ,\hat{\sigma}\hat{\eval}\hat{ t_1} ,\hat{\sigma}'\Quad
   e_2 ,\hat{\sigma}'\hat{\eval} \hat{t_2},\hat{\sigma}''}
  {e_1 \Or e_2 ,\hat{\sigma}\hat{\eval} \hat{t_1} \Or \hat{t_2},\hat{\sigma}''}

\newrule{E-Xor}
  {}
  {e_1 \Xor e_2 ,\hat{\sigma}\hat{\eval} e_1 \Xor e_2,\hat{\sigma}}


%% Normalisation %%%%%%%%%%%%%%%%%%%%%%%%%%%%%%%%%%%%%%%%%%%%%%%%%%%%%%%%%%%%%%%

\newmacro{RelationS}
  {t,\hat{\sigma} \hat{\stride} \hat{t'},\hat{\sigma}'}


\newrule{S-Edit}
  { }
  {\Edit v,\hat{\sigma} \hat{\stride} \Edit v,\hat{\sigma}}

\newrule{S-Fill}
  { }
  {\Enter \tau,\hat{\sigma} \hat{\stride} \Enter \tau,\hat{\sigma}}

\newrule{S-Update}
  { }
  {\Update l,\hat{\sigma} \hat{\stride} \Update l,\hat{\sigma}}


\newrule{S-Fail}
  { }
  {\Fail,\hat{\sigma} \hat{\stride} \Fail,\hat{\sigma}}


\newrule{S-ThenStay}
  {t_1,\hat{\sigma} \hat{\stride} \hat{t_1}',\hat{\sigma}'}
  {t_1 \Then e_2,\hat{\sigma} \hat{\stride} \hat{t_1}' \Then e_2,\hat{\sigma}'}
  [\Value(\hat{t_1}',\hat{\sigma}') = \bot]

\newrule{S-ThenFail}
  {t_1,\hat{\sigma} \hat{\stride} \hat{t_1}',\hat{\sigma}' \Quad
   e_2\ \hat{v_1},\hat{\sigma}' \hat{\eval} \hat{t_2},\hat{\sigma}''}
  {t_1 \Then e_2,\hat{\sigma} \hat{\stride} \hat{t_1}' \Then e_2,\hat{\sigma}'}
  [\Value(\hat{t_1}',\hat{\sigma}') = \hat{v_1} \land \Failing(\hat{t_2},\hat{\sigma}'')]

\newrule{S-ThenCont}
  {t_1,\hat{\sigma} \hat{\stride} \hat{t_1}',\hat{\sigma}'  \Quad
   e_2\ \hat{v_1},\hat{\sigma}' \hat{\eval} \hat{t_2 },\hat{\sigma}''}
  {t_1 \Then e_2,\hat{\sigma} \hat{\stride} \hat{t_2},\hat{\sigma}''}
  [\Value(\hat{t_1}',\hat{\sigma}') = \hat{v_1} \land \lnot\Failing(\hat{t_2},\hat{\sigma}'')]

\newrule{S-Next}
  {t_1,\hat{\sigma} \hat{\stride} \hat{t_1}',\hat{\sigma}'}
  {t_1 \Next e_2,\hat{\sigma} \hat{\stride} \hat{t_1}' \Next e_2,\hat{\sigma}'}


\newrule{S-And}
  {t_1,\hat{\sigma}  \hat{\stride} \hat{t_1}',\hat{\sigma}'  \Quad
   t_2,\hat{\sigma}' \hat{\stride} \hat{t_2}',\hat{\sigma}''}
  {t_1 \And t_2,\hat{\sigma} \hat{\stride} \hat{t_1}' \And \hat{t_2}',\hat{\sigma}''}


\newrule{S-OrLeft}
  {t_1,\hat{\sigma}  \hat{\stride} \hat{t_1}',\hat{\sigma}'}
  {t_1 \Or t_2,\hat{\sigma} \hat{\stride} \hat{t_1}',\hat{\sigma}'}
  [\Value(\hat{t_1}',\hat{\sigma}') = \hat{v_1}]

\newrule{S-OrRight}
  {t_1,\hat{\sigma}  \hat{\stride} \hat{t_1}',\hat{\sigma}'  \Quad
   t_2,\hat{\sigma}' \hat{\stride} \hat{t_2}',\hat{\sigma}''}
  {t_1 \Or t_2,\hat{\sigma} \hat{\stride} \hat{t_2}',\hat{\sigma}''}
  [\Value(\hat{t_1}',\hat{\sigma}') = \bot \land \Value(\hat{t_2}',\hat{\sigma}'') = \hat{v_2}]

\newrule{S-OrNone}
  {t_1,\hat{\sigma}  \hat{\stride }\hat{t_1}',\hat{\sigma}'  \Quad
   t_2,\hat{\sigma' }\hat{\stride} \hat{t_2}',\hat{\sigma}''}
  {t_1 \Or t_2,\hat{\sigma} \hat{\stride} \hat{t_1}' \Or \hat{t_2}',\hat{\sigma}''}
  [\Value(\hat{t_1}',\hat{\sigma}') = \bot \land \Value(\hat{t_2}',\hat{\sigma}'') = \bot]


\newrule{S-Xor}
  { }
  {e_1 \Xor e_2,\hat{\sigma} \hat{\stride} e_1 \Xor e_2,\hat{\sigma}}

\newrule{S-Eval}
    {e,\hat{\sigma} \hat{\eval} \hat{e}',\hat{\sigma}'  \Quad
     e',\hat{\sigma}' \hat{\stride} \hat{e}'',\hat{\sigma}''}
    {e,\hat{\sigma} \hat{\stride} \hat{e}'',\hat{\sigma}''}
    [e \neq \hat{e}']


%% Normalisation %%

\newmacro{RelationN}
  {e,\hat{\sigma} \hat{\normalise} \hat{t},\hat{\sigma}'}


\newrule{N-Done}
    {e,\hat{\sigma} \hat{\eval} \hat{t},\hat{\sigma}' \Quad
     \hat{t},\hat{\sigma}' \hat{\stride} \hat{t}',\hat{\sigma}''}
    {e,\hat{\sigma} \hat{\normalise} \hat{t},\hat{\sigma}'}
    [\hat{\sigma}'=\hat{\sigma}'' \land \hat{t}=\hat{t}']

\newrule{N-Repeat}
    {e,\hat{\sigma} \hat{\eval} \hat{t},\hat{\sigma}'  \Quad
     \hat{t},\hat{\sigma}' \hat{\stride} \hat{t}',\hat{\sigma}''  \Quad
     \hat{t}',\hat{\sigma}'' \hat{\normalise} \hat{t}'',\hat{\sigma}'''}
    {e,\hat{\sigma} \hat{\normalise} \hat{t}'',\hat{\sigma}'''}
    [\hat{\sigma}'\neq \hat{\sigma}''\vee \hat{t}\neq \hat{t}']



%% Handling %%

\newmacro{RelationH}
  {t,\hat{\sigma} \handle{j} \hat{t'},\hat{\sigma}'}


\newrule{H-Change}
  { }
  {\Edit v,\hat{\sigma} \xrightarrow[]{v}' \Edit v',\hat{\sigma}}
  [v,v':\tau]

\newrule{H-Empty}
  { }
  {\Edit v,\sigma \handle{\Empty} \Enter \tau,\sigma,\True}
  [v : \tau]

\newrule{H-Fill}
  { }
  {\Enter \tau,\hat{\sigma} \xrightarrow[]{v} \Edit v,\hat{\sigma}}
  [v:\tau]

\newrule{H-Update}
  { }
  {\Update l,\hat{\sigma} \xrightarrow[]{v} \Update l,\hat{\sigma}[l \mapsto v]}
  [\sigma(l),v:\tau]

\newrule{H-PassThen}
  {t_1,\sigma \xrightarrow[]{j} \hat{t_1'},\sigma'}
  {t_1 \Then e_2,\sigma \xrightarrow[]{j} \hat{t_1'} \Then e_2,\sigma'}

\newrule{H-PassNext}
  {t_1,\sigma \xrightarrow[]{j} \hat{t_1'},\sigma'}
  {t_1 \Next e_2,\sigma \xrightarrow[]{j} \hat{t_1'} \Next e_2,\sigma'}

\newrule{H-Next}
  {e_2\ \hat{v_1},\sigma \hat{\normalise} \hat{t_2},\hat{\sigma}'}
  {t_1 \Next e_2,\sigma \xrightarrow[]{\Continue} \hat{t_2},\hat{\sigma}'}
  [\Value{(t_1,\sigma)} = \hat{v_1} \land \neg\Failing{(\hat{t_2},\hat{\sigma}')}]


\newrule{H-FirstAnd}
  {t_1,\sigma \xrightarrow[]{j} \hat{t_1}',\hat{\sigma}'}
  {t_1 \And t_2,\sigma \xrightarrow[]{\First j} \hat{t_1}' \And t_2,\hat{\sigma}'}

\newrule{H-SecondAnd}
  {t_2,\sigma \xrightarrow[]{j} \hat{t_2}',\hat{\sigma}'}
  {t_1 \And t_2,\sigma \xrightarrow[]{\Second j} t_1 \And \hat{t_2}',\hat{\sigma}'}


\newrule{H-FirstOr}
  {t_1,\sigma \xrightarrow[]{j} \hat{t_1}',\hat{\sigma}'}
  {t_1 \Or t_2,\sigma \xrightarrow[]{\First j} \hat{t_1}' \Or t_2,\hat{\sigma}'}

\newrule{H-SecondOr}
  {t_2,\sigma \xrightarrow[]{j} \hat{t_2}',\hat{\sigma}' }
  {t_1 \Or t_2,\sigma \xrightarrow[]{\Second j} t_1 \Or \hat{t_2}',\hat{\sigma}'}


\newrule{H-PickLeft}
  {e_1,\sigma \normalise \hat{t_1},\hat{\sigma}'}
  {e_1 \Xor e_2,\sigma \xrightarrow[]{\Left} \hat{t_1},\hat{\sigma}'}
  [\neg\Failing(\hat{t_1},\hat{\sigma}')]

\newrule{H-PickRight}
  {e_2,\sigma \hat{\normalise} \hat{t_2},\hat{\sigma}'}
  {e_1 \Xor e_2,\sigma \xrightarrow[]{\Right} \hat{t_2},\hat{\sigma}'}
  [\neg\Failing(\hat{t_2},\hat{\sigma}')]



%% Driving %%

\newmacro{RelationI}
  {\hat{t},\hat{\sigma} \drive{j} \hat{t}',\hat{\sigma}'}



\newrule{I-Handle}
  {t,\sigma \xrightarrow[]{j} \hat{t}',\hat{\sigma}' \Quad
   \hat{t}',\hat{\sigma}' \hat{\normalise} \hat{t}'',\hat{\sigma}''}
  {t,\sigma \xRightarrow[]{j} \hat{t}'',\hat{\sigma}''}

% !TEX root=main.tex


%% Language %%%%%%%%%%%%%%%%%%%%%%%%%%%%%%%%%%%%%%%%%%%%%%%%%%%%%%%%%%%%%%%%%%%%

\newmacro{G-Language}{
  \begin{grammar}
    Expressions
      & e    &::= & \lambda x:\tau.\ e   & – abstraction \\
      &      &\mid& e_1\ e_2             & – application \\
      &      &\mid& x                    & – variable \\
      &      &\mid& s                    & – symbol \\
      &      &\mid& c                    & – constant \\
    \addlinespace
      &      &\mid& u\ e_1               & – unary operation \\
      &      &\mid& e_1\ o\ e_2          & – binary operation \\
      &      &\mid& \If{e_1}{e_2}{e_3}   & – branch \\
    \addlinespace
      &      &\mid& \unit                & – unit \\
      &      &\mid& \tuple{e_1, e_2}     & – pair \\
      &      &\mid& \Fst e               & – first projection \\
      &      &\mid& \Snd e               & – second projection \\
    \addlinespace
      &      &\mid& \Ref e               & – reference \\
      &      &\mid& !e                   & – dereference \\
      &      &\mid& e_1 := e_2           & – assignment \\
      % &      &\mid& e_1; e_2             & – sequence \\
      &      &\mid& l                    & – location \\
    \addlinespace
      &      &\mid& p                    & – pretask \\
    \addlinespace
    Constants
      & c    &::= & B                    & – boolean \\
      &      &\mid& I                    & – integer \\
      &      &\mid& S                    & – string \\
    \addlinespace
    Unary operations
      & u    &::= & \lnot                & – logical \\
      &      &\mid& -                    & – numerical \\
      &      &\mid& \#                   & – sequential \\
    \addlinespace
    Binary operations
      & o    &::= & \land \mid \lor                                     & – logical \\
      &      &\mid& < \mid \le \mid \equiv \mid \nequiv \mid \ge \mid > & – equational \\
      &      &\mid& + \mid - \mid \times \mid /                         & – numerical \\
      &      &\mid& \pp                                                 & – sequential \\
  \end{grammar}
}

\newmacro{G-Language-Compact}{
  \begin{grammar}
    \noalign{Expressions}\\
    & e &::= & \lambda x:\tau.\ e  \Mid  e_1\ e_2           & – abstraction, application \\
    &       & \mid  & x  \Mid  c \Mid \unit                        & – variable, constant, unit \\
    &       & \mid  & \fbox{$u\ e_1 \Mid e_1\ o\ e_2$}               & – unary, binary operation \\
    &       & \mid  & \If{e_1}{e_2}{e_3}                           & – branch \\
    &       & \mid  & \tuple{e_1, e_2}  \Mid  \Fst e  \Mid  \Snd e & – pair, projections \\
    &       & \mid  & \Ref e  \Mid  !e  \Mid  e_1 := e_2  \Mid  l  & – references, location \\
    &       & \mid  & p \Mid \fbox{$s$}                                     & – pretask, symbol \\
    \\
    \noalign{Constants}\\
    & c& ::= &  B  \Mid  I  \Mid  S                          & – boolean, integer, string\\
    \\
    \noalign{Unary Operations}\\
    & u &::= &  \lnot \Mid - \Mid \#                         & - not, negate, length\\
    \\
    \noalign{Binary Operations}\\
    & o &::= & < \Mid \le \Mid \equiv \Mid \nequiv \Mid \ge \Mid > & – equational \\
    &       & \mid  & + \Mid - \Mid \times \Mid /                         & – numerical \\
    &       & \mid  & \pp                                                 & – append \\
  \end{grammar}
}


\newmacro{G-Pretasks}{
  \begin{grammar}
    Pretasks
      & p    &::= & \Edit e              & – valued editor \\
      % &      &\mid& \View e              & – valued read-only editor \\
      &      &\mid& \Enter \tau          & – unvalued editor \\
      &      &\mid& \Update e            & – shared editor \\
      % &      &\mid& \Watch e             & – shared read-only editor \\
    \addlinespace
      &      &\mid& e_1 \Then e_2        & – step \\
      &      &\mid& e_1 \Next e_2        & – user step \\
    \addlinespace
      &      &\mid& e_1 \And e_2         & – composition \\
    \addlinespace
      &      &\mid& e_1 \Or e_2          & – choice \\
      &      &\mid& e_1 \Xor e_2         & – user choice \\
    \addlinespace
      &      &\mid& \Fail                & – fail task \\
  \end{grammar}
}

\newmacro{G-Pretasks-Compact}{
  \begin{grammar}
    \noalign{Pretasks}\\
      & p    &::= & \Edit e \Mid \Enter \tau \Mid \Update e            & – editors: valued, unvalued shared \\
      &      &\mid& e_1 \Then e_2 \Mid e_1 \Next e_2                   & – steps: internal, external \\
      &      &\mid& \Fail \Mid e_1 \And e_2                            & – fail, composition \\
      &      &\mid& e_1 \Or e_2 \Mid e_1 \Xor e_2                      & – choice: internal, external\\
  \end{grammar}
}


\newmacro{G-Types}{
  \begin{grammar}
    Types
      & \tau &::= & \tau_1 \to \tau_2    & – function type \\
      &      &\mid& \tau_1 \times \tau_2 & – product type \\
      &      &\mid& \Unit                & – unit type \\
      &      &\mid& \Reference \tau      & – reference type \\
      &      &\mid& \Task \tau           & – task type \\
      &      &\mid& \beta                & – basic type \\
      % &      &\mid& \alpha               & – universal type \\
    Basic types
      &\beta &::= & \Bool                & – boolean type \\
      &      &\mid& \Int                 & – integer type \\
      &      &\mid& \String              & – string type \\
  \end{grammar}
}

\newmacro{G-Types-Compact}{
  \begin{grammar}
    \noalign{Types}\\
      & \tau  &  ::= & \tau_1 \to \tau_2 \Mid \tau_1 \times \tau_2 \Mid \beta & – function, product, basic \\
      &       & \mid & \Reference \tau \Mid \Task \tau             & – reference, task \\
      \\
    \noalign{Basic types}\\
      & \beta &  ::= &\Unit \Mid \Bool                           & – unit, boolean\\
      &       & \mid & \Int \Mid   \String                       & – integer, string
  \end{grammar}
}


\newmacro{G-Values}{
  \begin{grammar}
    Values
      & v    &::= & \lambda x:\tau.\ e   & – abstraction \\
      &      &\mid& c                    & – constant \\
      &      &\mid& l                    & – location \\
    \addlinespace
      &      &\mid& s                    & – symbol \\
      &      &\mid& u\ v                 & – symbolic unary operation \\
      &      &\mid& v_1\ o\ v_2          & – symbolic binary operation \\
    \addlinespace
      &      &\mid& \tuple{v_1, v_2}     & – pair value \\
      &      &\mid& \unit                & – unit \\
    \addlinespace
      &      &\mid& t                    & – task \\
  \end{grammar}
}

\newmacro{G-Values-Compact}{
  \begin{grammar}
    \noalign{Values}\\
      & v &  ::= & \lambda x:\tau.\ e \Mid \tuple{v_1, v_2} \Mid \unit & – abstraction, pair, unit \\
      &   & \mid & c \Mid l \Mid t                                     & – constant, location, task \\
      &   & \mid & \fbox{$s \Mid u\ v \Mid v_1\ o\ v_2$}               & – symbol, unary op, binary op\\
  \end{grammar}
}


\newmacro{G-Tasks}{
  \begin{grammar}
    Tasks
      & t    &::= & \Edit v              & – valued editor \\
      % &      &\mid& \View v              & – valued read-only editor \\
      &      &\mid& \Enter \tau          & – unvalued editor \\
      &      &\mid& \Update l            & – shared editor \\
      % &      &\mid& \Watch l             & – shared read-only editor \\
    \addlinespace
      &      &\mid& t_1 \Then e_2        & – step \\
      &      &\mid& t_1 \Next e_2        & – user step \\
    \addlinespace
      &      &\mid& t_1 \And t_2         & – composition \\
    \addlinespace
      &      &\mid& t_1 \Or t_2          & – choice \\
      &      &\mid& e_1 \Xor e_2         & – user choice \\
    \addlinespace
      &      &\mid& \Fail                & – fail task \\
  \end{grammar}
}

\newmacro{G-Tasks-Compact}{
  \begin{grammar}
    \noalign{Tasks}\\
      & t &  ::= & \Edit v \Mid \Enter \tau \Mid \Update l           & – editors \\
      &   & \mid & t_1 \Then e_2 \Mid t_1 \Next e_2                  & – steps \\
      &   & \mid & \Fail \Mid t_1 \And t_2                           & – fail, combination \\
      &   & \mid & t_1 \Or t_2 \Mid e_1 \Xor e_2                     & – choices \\
  \end{grammar}
}

\newmacro{G-Inputs}{
  \begin{grammar}
    Symbolic inputs
      & i    & ::=& \fbox{$s$}           & – symbolic action \\
      &      &\mid& \First i             & – pass to first \\
      &      &\mid& \Second i            & – pass to second
  \end{grammar}
}

\newmacro{G-Inputs-Compact}{
  \begin{grammar}
    \noalign{Symbolic inputs}\\
      & i    & ::=& \fbox{$s$} \Mid \First i \Mid \Second i  & – symbolic action, to first, to second
  \end{grammar}
}

\newmacro{G-CInputs}{
  \begin{grammar}
    Concrete inputs
      & j    & ::=& \bar{a}              & – concrete action \\
      &      &\mid& \First j             & – pass to first \\
      &      &\mid& \Second j            & – pass to second \\
    Concrete actions
      & \bar{a} & ::=& c                    & – constant \\
      &         &\mid& \Continue            & – continue with next task \\
      &         &\mid& \Left                & – go left \\
      &         &\mid& \Right               & – go right \\
  \end{grammar}
}

\newmacro{G-CInputs-Compact}{
  \begin{grammar}
    \noalign{Concrete inputs}\\
      & j    & ::=& a \Mid \First j \Mid \Second j             & – action, to first, to second \\
      \\
    \noalign{Concrete actions}\\
      & a  & ::=& c                    & – constant \\
      &      &\mid& \Continue  \Mid \Left \Mid \Right          & – continue, go left, go right\\
  \end{grammar}
}


\newmacro{G-Predicates}{
  \begin{grammar}
    Predicates
      & \phi &::= & c                    & – constant \\
      &      &\mid& s                    & – symbol \\
      &      &\mid& \Continue            & – continue\\
      &      &\mid& \Left             & – go left \\
      &      &\mid& \Right            & – go right \\
      &      &\mid& u\ \phi              & – symbolic unary operation \\
      &      &\mid& \phi_1\ o\ \phi_2    & – symbolic binary operation \\
  \end{grammar}
}

\newmacro{G-Predicates-Compact}{
  \begin{grammar}
    \noalign{Path conditions}\\
      & \phi &::= & c \Mid s                   & – constant, symbol\\
      &      &\mid& \Continue  \Mid \Left \Mid \Right          & – continue, go left, go right\\
      &      &\mid& u\ \phi \Mid \phi_1\ o\ \phi_2  & – symbolic unary, binary operation
  \end{grammar}
}

% !TEX root=main.tex


%% Language %%%%%%%%%%%%%%%%%%%%%%%%%%%%%%%%%%%%%%%%%%%%%%%%%%%%%%%%%%%%%%%%%%%%

\newmacro{O-Value}{
  \begin{function}
    \signature{\Value : \mathrm{Tasks} \times \mathrm{States} \rightharpoonup \mathrm{Values}} \\
    \Value(\Edit v, \sigma)                &=& v \\
    \Value(\Enter \tau, \sigma)            &=& \bot \\
    \Value(\Update l, \sigma)              &=& \sigma(l) \\
    \Value(\Fail, \sigma)                  &=& \bot \\
    \Value(t_1 \Then e_2, \sigma)          &=& \bot \\
    \Value(t_1 \Next e_2, \sigma)          &=& \bot \\
    \Value(t_1 \And t_2, \sigma)           &=& \left\{
      \begin{array}{ll}
        \tuple{v_1, v_2}  & \ \when\ \Value(t_1, \sigma) = \
        v_1 \land \Value(t_2, \sigma) = v_2 \\
        \bot                          & \ \otherwise
      \end{array}
    \right. \\
    \Value(t_1 \Or t_2, \sigma)            &=& \left\{
      \begin{array}{ll}
        v_1                           & \ \when\ \Value(t_1, \sigma) = v_1 \\
        v_2                           & \ \when\ \Value(t_1, \sigma) = \bot \land \Value(t_2, \sigma) = v_2 \\
        \obox{\tuple{v_1, v_2}}{\bot} & \ \otherwise
      \end{array}
    \right. \\
    \Value(t_1 \Xor t_2, \sigma)           &=& \bot
  \end{function}
}

\newmacro{O-Failing}{
  \begin{function}
    \signature{\Failing : \mathrm{Tasks} \times \mathrm{States} \to \mathrm{Booleans}} \\
    \Failing(\Edit v,\sigma)       &=& \False \\
    \Failing(\Enter \tau,\sigma)   &=& \False \\
    \Failing(\Update l,\sigma)     &=& \False \\
    \Failing(\Fail,\sigma)         &=& \True \\
    \Failing(t_1 \Then e_2,\sigma) &=& \Failing(t_1,\sigma) \\
    \Failing(t_1 \Next e_2,\sigma) &=& \Failing(t_1,\sigma) \\
    \Failing(t_1 \And t_2,\sigma)  &=& \Failing(t_1,\sigma) \land \Failing(t_2,\sigma) \\
    \Failing(t_1 \Or t_2,\sigma)   &=& \Failing(t_1,\sigma) \land \Failing(t_2,\sigma) \\
    \Failing(e_1 \Xor e_2,\sigma)  &=& \Failing(t_1,s_1') \land \Failing(t_2,s_2') \quad\where\ e_1,\sigma \normalise t_1,s_1' \mathbf{\ and\ } e_2,\sigma \normalise t_2,s_2'
  \end{function}
}




%% Journal information
%% Supplied to authors by publisher for camera-ready submission;
%% use defaults for review submission.
% \acmJournal{PACMPL}
% \acmVolume{1}
% \acmNumber{ICFP} % CONF = POPL or ICFP or OOPSLA
% \acmArticle{1}
% \acmYear{2018}
% \acmMonth{1}
% \acmDOI{} % \acmDOI{10.1145/nnnnnnn.nnnnnnn}
% \startPage{1}


\acmConference[IFL'19]{International Symposium on Implementation and Application of Functional Languages}{September 2019}{Singapore}
\acmYear{2020}
\copyrightyear{2020}

%% Copyright information
%% Supplied to authors (based on authors' rights management selection;
%% see authors.acm.org) by publisher for camera-ready submission;
%% use 'none' for review submission.
\setcopyright{none}
%\setcopyright{acmcopyright}
%\setcopyright{acmlicensed}
%\setcopyright{rightsretained}
%\copyrightyear{2018}           %% If different from \acmYear

%% Bibliography style
\bibliographystyle{ACM-Reference-Format}
%% Citation style
%% Note: author/year citations are required for papers published as an
%% issue of PACMPL.
\citestyle{acmauthoryear}   %% For author/year citations






% version.tex must define the command \version
\IfFileExists{version.tex}
  {\input{version.tex}}
  {\newcommand{\version}{unknown version}}

\hypersetup
{ pdfcreator=\version
}

\usepackage{fancyhdr}
\fancyfoot[C]{\thepage}
%\fancyfoot[R]{v.\version}






\begin{document}

%% Title information
\title{A symbolic execution semantics for TopHat}
% \title{TopHat: A calculus for modular interactive workflows}
                                        %% [Short Title] is optional;
                                        %% when present, will be used in
                                        %% header instead of Full Title.
%\titlenote{with title note}             %% \titlenote is optional;
                                        %% can be repeated if necessary;
                                        %% contents suppressed with 'anonymous'
%\subtitle{Revisited edition}            %% \subtitle is optional
%\subtitlenote{with subtitle note}       %% \subtitlenote is optional;
                                        %% can be repeated if necessary;
                                        %% contents suppressed with 'anonymous'


%% Author information
%% Contents and number of authors suppressed with 'anonymous'.
%% Each author should be introduced by \author, followed by
%% \authornote (optional), \orcid (optional), \affiliation, and
%% \email.
%% An author may have multiple affiliations and/or emails; repeat the
%% appropriate command.
%% Many elements are not rendered, but should be provided for metadata
%% extraction tools.

\author{Nico Naus}
%\authornote{with author1 note}          %% \authornote is optional; can be repeated if necessary
%\orcid{nnnn-nnnn-nnnn-nnnn}             %% \orcid is optional
\affiliation{
  %\position{PhD}
  \department{Information and Computing Sciences}
                                        %% \department is recommended
  \institution{Utrecht University}      %% \institution is required
  \streetaddress{Princetonplein 5}
  \postcode{3584 CC}
  \city{Utrecht}
  %\state{State1}
  \country{The Netherlands}
}
\email{n.naus@uu.nl}                    %% \email is recommended

\author{Tim Steenvoorden}
%\authornote{with author1 note}          %% \authornote is optional; can be repeated if necessary
%\orcid{nnnn-nnnn-nnnn-nnnn}             %% \orcid is optional
\affiliation{
  %\position{PhD}
  \department{Software Science}
  %\department{Institute for Computing and Information Sciences}
                                        %% \department is recommended
  \institution{Radboud University}      %% \institution is required
  \streetaddress{Toernooiveld 212}
  \postcode{6525 EC}
  \city{Nijmegen}
  %\state{State1}
  \country{The Netherlands}
}
\email{tim@cs.ru.nl}                     %% \email is recommended

\author{Markus Klinik}
%\authornote{with author1 note}          %% \authornote is optional; can be repeated if necessary
%\orcid{nnnn-nnnn-nnnn-nnnn}             %% \orcid is optional
\affiliation{
  %\position{PhD}
  \department{Software Science}
  %\department{Institute for Computing and Information Sciences}
                                        %% \department is recommended
  \institution{Radboud University}
                                        %% \institution is required
  \streetaddress{Toernooiveld 212}
  \postcode{6525 EC}
  \city{Nijmegen}
  %\state{State1}
  \country{The Netherlands}
}
\email{m.klinik@cs.ru.nl}               %% \email is recommended




%% Paper note
%% The \thanks command may be used to create a "paper note" ---
%% similar to a title note or an author note, but not explicitly
%% associated with a particular element.  It will appear immediately
%% above the permission/copyright statement.
%\thanks{with paper note}                %% \thanks is optional
                                        %% can be repeated if necesary
                                        %% contents suppressed with 'anonymous'


%% Abstract
%% Must come before \maketitle command
\begin{abstract}
  % !TEX root=../main.tex

% Context
Task Oriented Programming (TOP) is a relatively new programming paradigm that provides an abstraction over workflow programs.
TOP is typically applied to domains where correctly functioning software is essential, it could have huge financial or strategical consequences.
% Inquiry
We aim to improve the quality of software written in TOP.
Currently, only testing is available as a measure to verify that programs are behaving as intended.
% Approach
Instead, we propose to apply formal techniques, namely symbolic execution, to guarantee that no aberrant behaviour will occur.
In order to do this, we develop a symbolic execution semantics for the TOPHAT language.
% Knowledge
The symbolic execution allows us to prove that certain properties over programs always hold for certain programs in TOPHAT.
% Grounding
The symbolic execution semantics is shown to be correct, by proving soundness and completeness with respect to the original semantics of TOPHAT.
% Importance
This work represents a huge step forward in the formal verification of TOP software.
Ensuring qualtiy is sessential in the domains that TOP is commonly applied to.


% Context: What is the broad context of the work? What is the importance of the general research area?
% Inquiry: What problem or question does the paper address? How has this problem or question been addressed by others (if at all)?
% Approach: What was done that unveiled new knowledge?
% Knowledge: What new facts were uncovered? If the research was not results oriented, what new capabilities are enabled by the work?
% Grounding: What argument, feasibility proof, artifacts, or results and evaluation support this work?
% Importance: Why does this work matter?

\end{abstract}

% \begin{teaserfigure}
%    \includegraphics[width=\textwidth]{figures/declrequest-part.pdf}
%    \caption{This is a teaser}
%    \label{fig:teaser}
% \end{teaserfigure}

%% 2012 ACM Computing Classification System (CSS) concepts
%% Generate at 'http://dl.acm.org/ccs/ccs.cfm'.

%% End of generated code


%% Keywords
%% comma separated list, optional
%\keywords{workflow, dataflow, visual programming, program generation}


%% Note: \maketitle command must come after title commands, author
%% commands, abstract environment, Computing Classification System
%% environment and commands, and keywords command.
\maketitle

% !TEX root=../main.tex

\section{Introduction}

The Task-Oriented Programming paradigm (\TOP) is an abstraction over workflow specifications.
The idea of \TOP is to describe the work that needs to be done, in which order, by which person.
From this specification, an application can be generated that helps to coordinate people and machines to execute the work.
The \ITASKS framework~\cite{DBLP:conf/ppdp/PlasmeijerLMAK12} is an implementation of the paradigm in the functional programming language Clean.
In earlier work \cite{Steenvoorden2019}, we presented the programming language TopHat, written \TOPHAT, to distill the core features of \TOP into a language suitable for formal treatment.
%
The usefulness of \TOP has been demonstrated in several projects that applied it to implement various applications.
It has been used by the Netherlands Royal Navy~\cite{jansen2018dynamic}, the Dutch Tax Office~\cite{conf/sfp/StutterheimAP17} and the Dutch Coast Guard~\cite{lijnse2012incidone}. % of moeten we hier verwijzen naar Consequence Management – Declarative Modelling of Maritime C2-systems
Furthermore, it has potential for application in domains like healthcare and Internet of Things~\cite{DBLP:conf/cgo/KoopmanLP18}.

Applications in these kinds of domains are often mission critical, where programming mistakes can have severe consequences.
% Currently, iTasks programs are verified by running manually written test-cases. % do we have a source?
% It is evident that testing alone is not sufficient.
In order to verify that a \TOPHAT program behaves as intended, we would like to show that it satisfies a given property.
A common way to do this is to write test cases, or to generate random input, and verify that all outcomes fulfil the property.
Writing tests manually is time consuming and cumbersome.
Testing interactive applications needs people to operate the application, maybe making use of a way to record and replay interactions.
With this kind of testing there is no guarantee that all possible execution paths are covered.

To overcome these issues, we apply symbolic execution.
Instead of executing tasks with test input, or letting a user interactively test the application,
we run tasks on symbolic input.
Symbolic input consists of tokens that represent any value of a certain type.
When a program branches, the execution engine records the conditions over the symbolic input that lead to the different branches.
These conditions can then be compared to a given predicate to check if the predicate holds under all conditions.
We let an \SMT solver verify these statements.

In this way we can guarantee that given predicates over the outcome of a \TOP program always hold.
Since iTasks is not suitable for formal reasoning, we instead apply symbolic execution to \TOPHAT~\cite{Steenvoorden2019}, by systematically changing the semantic rules of the original language.



\subsection{Contributions}

This paper makes the following contributions.

\begin{itemize}
  \item We present a symbolic execution semantics for \TOPHAT, a programming language for workflows embedded in the simply typed $\lambda$-calculus.
  \item We prove soundness and completeness of the symbolic semantics with respect to the original \TOPHAT semantics.
  \item We present an implementation of the symbolic execution semantics in Haskell.
\end{itemize}



\subsection{Structure}

\Cref{sec:intuition} gives a brief overview of \TOPHAT and its concepts.
\Cref{sec:examples} introduces some examples to demonstrate the goal of our symbolic execution analysis.
In \cref{sec:language}, the \TOPHAT language is defined.
\Cref{sec:semantics} goes on to define the formal semantics of the symbolic execution.
In \cref{sec:properties}, soundness and completeness are shown for the symbolic execution semantics with respect to the original \TOPHAT semantics.
In \cref{sec:relatedwork} related work is discussed, and \cref{sec:conclusion} concludes.

% !TEX root=../main.tex

\section{Intuition}
\label{sec:intuition}

This section briefly introduces the task-oriented programming language \TOPHAT,
and discuss our vision about symbolic evaluation of this language.

The \TOPHAT language consists two parts, the host language and the task language.
Programs in \TOPHAT are called \emph{tasks}.
The basic elements of tasks are editors.
Using combinators, tasks can be combined into larger tasks.

The task language is embedded in a simply typed lambda calculus with references, conditionals, booleans, integers, strings, pairs, lists and unary and binary operations on these types.
References allow tasks to communicate with each other, sharing information across task boundaries.
The full syntax of the host language is listed in Section~\ref{sec:language}.
Next, we discuss the main constructs of the task language.


\subsection{Editors}

Editors are the basic method for programs to communicate with the outside world.
They are an abstraction over widgets in a \GUI library or on webpage forms.
Users can change the value held by an editor, in the same way they can manipulate widgets in a \GUI.

When a \TOP implementation generates an application from a task specification, it derives user interfaces for the editors.
The appearance of an editor is influenced by its type.
For example, an editor for a string can be represented by a simple input field, a date by a calendar, and a location by a pin on a map.

There are three different editors in \TOPHAT.
\begin{description}
  \item[$\Edit v$] Valued editor.\\
    This editor holds a value $v$ of a certain type.
    The user can replace the value by a new value of the same type.
  \item[$\Enter \tau$] Unvalued editor.\\
    This editor holds no value, and can receive a value of type $\tau$.
    When that happens, it turns into a valued editor.
  \item[$\Update l$] Shared editor.\\
    This editor refers to a store location $l$.
    Its observable value is the value stored at that location.
    When it receives a new value, this value will be stored at location $l$.
\end{description}



\subsection{Combinators}

Editors can be combined into larger tasks using combinators.
Combinators describe the way people collaborate.
Tasks can be performed in sequence or in parallel, or there is a choice between two tasks.

The following combinators are available in \TOPHAT.
\begin{description}
  \item[$t \Then e$] Step.\\
    Users can work on task $t$.
    As soon as $t$ has a value, that value is passed on to the right hand side $e$, with which it continues.
  \item[$t \Next e$] User Step.\\
    Users can work on task $t$.
    When $t$ has a value, the step becomes enabled.
    Users can then send a continue event to the combinator.
    When that happens, the value of $t$ is passed to the right hand side, with which it continues.
  \item[$t_1 \And t_1$] Composition.\\
    Users can work on tasks $t_1$ and $t_2$ in parallel.
  \item[$t_1 \Or t_2$] Choice.\\
    The system chooses between $t_1$ or $t_2$,
    based on which task first has a value.
    If both tasks have a value, the system chooses the left one.
  \item[$e_1 \Xor e_1$] User choice.\\
    A user has to make a choice between either the left or the right hand side.
    The user continues to work on the chosen task.
\end{description}

In addition to editors and combinators, \TOPHAT also contains the fail task ($\Fail$).
Programmers can use this task to indicate that a task is not reachable or viable.
For example, when the right hand side of a step combinator is $\Fail$, the step will not proceed to that task.



\subsection{Observations}

Several observations can be made on tasks.
Using the value function $\Value$, the current value of a task can be determined.
The value function is a partial function, since not all tasks have a value.
For example empty editors and steps do not have a value.

One can also observe whether or not a task is failing, by means of the failing function $\Failing$.
The task $\Fail$ is failing, as is a parallel combination of failing tasks ($\Fail \And \Fail$).

The step combinator makes use of both functions in order to determine if it can step.
First, it uses $\Value$ to see if the left hand side produces a value.
If that is the case, it uses the $\Failing$ function to see if it is safe to step to the right hand side.
The complete definition of the value and failing function are discussed in Section~\ref{subsec:observations}.

% !TEX root=../main.tex


\section{Examples}
\label{sec:examples}

This section will briefly introduce the task-oriented programming language \TOPHAT,
accompanied by two examples to illustrate how the language works and what kind of properties we would like to prove.

\subsection{\TOPHAT}

\TOPHAT is a task-oriented programming language.
Its aim is to model real world collaboration.

Programs in \TOPHAT are called tasks.
The smallest elements of a task is called an editor.

Editors are the basic method for communicating with the outside world.
There are three different Editors.
\begin{description}
  \item[$\Edit v$] Valued editor.\\
    This editor holds a value $v$ of a certain type.
    A new value of that type can be given as input.
  \item[$\Enter \tau$] Unvalued editor.\\
    This editor holds no value, and can receive a value of type $\tau$.
    It will then turn into a valued editor.
  \item[$\Update l$] Shared editor.\\
    This editor refers to a shared location $l$.
    Its observable value is the value stored at that location.
    It can receive a new value, this value will then be stored at location $l$.
\end{description}

Editors can be combined into tasks using combinators.
These combinators describe the way people collaborate.
The following combinators are available in \TOPHAT.

\begin{description}
  \item[$t \Then e$] Step.\\
  Users can work on task $t$.
  As soon as $t$ yields a value, that value is passed on to the right hand side, with which it continues.
  \item[$t \Next e$] User Step.\\
  Users can work on task $t$.
  When they are done, and $t$ yields a value, the user can send a continue event to the combinator.
  The value of $t$ is then passed on to the right hand side, with which it continues.
  \item[$t_1 \And t_1$] Composition.\\
  Users can work on tasks $t_1$ and $t_2$ at the same time.
  \item[$t_1 \Or t_2$] Choice.\\
  The system chooses between $t_1$ or $t_2$. If either of those returns a value, the system chooses that task.
  \item[$e_1 \Xor e_1$] User choice.\\
  A user has to make a choice between either the left or the right hand side.
  The user continues to work on the chosen task.
\end{description}

In addition to editors and combinators, \TOPHAT also contains a fail task $\Fail$.
This task is used by programmers to indicate that a task is not reachable or viable.
For example, when the right hand side of a step combinator is $\Fail$, the step will not proceed onto that task.

This language of tasks and combinators is embedded in the simply typed lambda calculus, augmented with references, pairs, if-then-else, booleans, integers and string, and unary and binary operations on these constants.
The full syntax of this host language is listed in Section~\ref{expressions}.

The references present in the host language allow tasks to communicate with each other,
sharing information that is globally available.

Finally, several observations can be made over tasks.
Using the value function $\Value$, the current value of a task can be determined.
The value function is a partial function, since not all tasks have a value.
For example, $\Enter \tau$ holds no value.
We can also observe wether or not a task is failing, by means of the failing function $\Failing$.
The step combinator makes use of both functions in order to determine if it can step.
First, it uses $\Value$ to see if the left hand side produces a value.
If that is the case, it uses the $\Failing$ function to see if it is safe to step to the right hand side using that value.
The complete definition of the value and failing function are listed in Section~\ref{subsec:observations}.


\subsection{Tax subsidy request}

\citet{conf/sfp/StutterheimAP17} worked with the Dutch tax office to develop a demonstrator for a fictional but realistic law about solar panel subsidies.
In this section we study a simplified version of this, translated to \TOPHAT, to illustrate how symbolic TopHat can be used to prove that the program implements the law.

This example proves that a citizen will get subsidy only under the following conditions.
\begin{itemize}
\item The company has confirmed that they installed solar panels for the citizen.
\item The tax officer has approved the request.
\item The tax officer can only approve the request if the company has confirmed, and the request is filed within one year of the invoice date.
\item The amount of the granted subsidy is maximal 600 EUR.
\end{itemize}

\begin{figure}
\begin{TASK}
  provideCitizenInformation
      >>= \ <<address, today>> .
  provideDocuments <&> companyConfirm
      >>= \ <<<<invoiceAmount, invoiceDate>>, cConfirmed>> .
  officerApprove invoiceDate today cConfirmed
      >>= \ decision .
  let subsidyAmount = if decision
          then min 600 (invoiceAmount / 10) else 0 in
  edit <<subsidyAmount, decision, cConfirmed, invoiceAmount, invoiceDate, today>>
  provideCitizenInformation = enter <<Address, Date>>
  provideDocuments = enter <<Amount, Date>>
  companyConfirm = edit True <?> edit False
  officerApprove invoiceDate today cConfirmed =
      edit False <?> if (today - invoiceDate < 365 && cConfirmed)
                 then edit True else fail
\end{TASK}
\caption{Subsidy request and approval workflow at the Dutch tax office.}
\label{fig:thetaxman}
\end{figure}

\Cref{fig:thetaxman} shows the program.
It works as follows.
First, the citizen has to enter their home address and today's date.
Then, in parallel the citizen has to provide the invoice documents of the installed solar panels, while the solar panel company has to confirm that they actually installed solar panels at the citizen's address.
Once the invoice and the confirmation are there, the tax officer has to approve the request.
The officer can always decline the request, but they can only approve it if the company has confirmed and the application date is within one year after the invoice date.
The result of the program is the amount of the subsidy, together with all information needed to prove the required properties.



\subsection{Flightbooking}

% !TEX root=../main.tex


\pagebreak
\section{Calculus}
\label{sec:language}

Differences with the normal grammar are discussed in each subsection.


\subsection{Expressions}

\usemacro{G-Language}

\begin{itemize}
  \item
    Symbols $s$ can be part of any expression.
    As locations $l$, they are not intended to be used by programmers,
    they are generated by the semantics.
    Symbols are intuitively a read input from the end user.
  \item
    Unary and binary operations are made explicit.
\end{itemize}

\usemacro{G-Pretasks}

\begin{itemize}
  \item
    There are read-only versions of valued and shared editors.
    \fixme{Are they needed?}
\end{itemize}



\subsection{Types}

\usemacro{G-Types}

(Nothing changes here.)
\todo{Add typing rule for symbols.}


\subsection{Values}

\usemacro{G-Values}

\begin{itemize}
  \item
    Symbols $s$ are now values,
    as are unary and binary operations containing symbols.
    Operations on normal values are normally evaluated.
    Note that pairs of symbols are allowed!
    This could be extended to lists of symbols in the future.
\end{itemize}

\usemacro{G-Tasks}

\begin{itemize}
  \item Read-only editors are added in the obvious way.
\end{itemize}



\subsection{Inputs}

\usemacro{G-Inputs}

\begin{itemize}
  \item Editors cannot be set to a concrete value $v$, but only to a symbol $s$.
\end{itemize}



\subsection{Predicates}

\todo{Add predicate grammar}
\usemacro{G-Predicate}

% !TEX root=../main.tex


\section{Semantics}
\label{sec:semantics}



\subsection{Observations}

\todo{Extend for read-only editors.}

\begin{center}
  \usemacro{O-Value}
\end{center}

\begin{center}
  \usemacro{O-Failing}
\end{center}

\usemacro{O-Inputs}

\fixme{Are these still correct?}



\subsection{Evaluation}

\todo{Add rules for $\Fst$ and $\Snd$.}

\begin{gather*}
  \boxed{\RelationE} \Break
  \userule{Sym-E-Value}\Quad
  \userule{Sym-E-Pair} \Break
  \userule{Sym-E-App} \Break
  \userule{Sym-E-If} \Break
  %\userule{Sym-E-IfTrue} \Break
  %\userule{Sym-E-IfFalse} \Break
  \userule{Sym-E-Ref} \Quad
  \userule{Sym-E-Deref} \Break
  \userule{Sym-E-Assign} \Break
  \userule{Sym-E-Edit} \Quad
  \userule{Sym-E-Enter}\Quad
  \userule{Sym-E-Update}\Break
  \userule{Sym-E-Then}\Quad
  \userule{Sym-E-Next}\Break
  \userule{Sym-E-And}\Break
  \userule{Sym-E-Or} \Break
  \userule{Sym-E-Xor}\Quad
  \userule{Sym-E-Fail}
\end{gather*}



\subsection{Normalisation}

\begin{gather*}
  \boxed{\RelationS} \Break
  \userule{Sym-S-ThenStay} \Break
  \userule{Sym-S-ThenFail} \Break
  \userule{Sym-S-ThenCont} \Break
  \userule{Sym-S-OrLeft} \Break
  \userule{Sym-S-OrRight} \Break
  \userule{Sym-S-OrNone} \Break
  \userule{Sym-S-Edit} \Quad
  \userule{Sym-S-Fill} \Break
  \userule{Sym-S-Update} \Quad
  \userule{Sym-S-Fail} \Break
  \userule{Sym-S-Xor} \Quad
  \userule{Sym-S-Next} \Break
  \userule{Sym-S-And}
\end{gather*}


\begin{gather*}
  \boxed{\RelationN} \Break
  \userule{Sym-N-Done} \Break
  \userule{Sym-N-Repeat}
\end{gather*}



\subsection{Handling}

\begin{gather*}
  \boxed{\RelationH} \Break
  \userule{Sym-H-Change} \Quad
  \userule{Sym-H-Fill} \Break
  \userule{Sym-H-Update}\Break
  \userule{Sym-H-Next} \Break
  \userule{Sym-H-PassThen} \Quad
  \userule{Sym-H-PassNext} \Break
  \userule{Sym-H-PickLeft} \Break
  \userule{Sym-H-PickRight}\Break
  \userule{Sym-H-Pick}\Break
  \userule{Sym-H-And}\Break
  \userule{Sym-H-Or}
\end{gather*}

Note that \refrule{Sym-H-Empty} is omitted as it does nothing useful for symbolic execution.


\begin{gather*}
  \boxed{\RelationI} \Break
  \userule{Sym-I-Handle}
\end{gather*}


\subsection{Driving}
\label{subsec:driving}

This section describes the top level symbolic execution function.

This function called drive, which takes $t,I,\sigma,\phi$, is recursively called to produce a list of end states an predicates.
We consider a task, state and predicate to be an end state if the task value can be observed, and the predicate is satisfiable.

A recursive call is terminated when one of the two stop conditions is met.

When the predicate cannot be satisfied, we know that all future steps won't be satisfiable either.
In fact, no future state will be satisfiable, and we can therefore safely remove it.

The second stop condition is when an execution does not alter anymore between steps.
In other words, the program is stuck, no value can be observed, an no value will ever be observed.

We prove that both criteria only remove paths that are invalid or infinite in Section~\ref{sec:properties}.


\begin{function}
  \signature{\mathit{drive} :: \Task \times [\mathrm{Inputs}] \times \mathrm{State} \times \mathrm{Predicate} \rightarrow [(\Task,[\mathrm{Inputs}],\mathrm{State},\mathrm{Predicate})]} \\
  \mathit{drive}\ t\ I\ \sigma \ \phi  = \mathit{map} \ (\mathit{concat . drive'}) (t,\sigma \drive{})\\
                \mathit{where}\ \mathit{drive'}\ t'\ i\ \sigma'\ \phi'=\\
                        \begin{array}{ll}
                          [\ ] & \neg \text{SAT } (\phi'\land\phi)\\
                  \relax [(t',I\oplus[i],\sigma',\phi\land\phi')] & \Value(t',\sigma') \equiv v \wedge \text{SAT } (\phi'\land\phi)\\
                  \relax [\ ]            & t' \equiv t \wedge \phi' \equiv \True \wedge \Value(t',\sigma') \equiv \bot\\
                        \mathit{drive}\ t'\ (I\oplus[i])\ \sigma'\ (\phi\land\phi') & (t' \neq t \vee \neg\phi') \wedge \Value(t',\sigma') \equiv \bot
                                  \end{array}
\end{function}

\begin{definition}[Satisfiability of predicates]
  \label{def:Sat}
  $\text{SAT }\phi \iff \exists M=[s_0\mapsto c_0,\cdots,s_n\mapsto c_n]. M\phi\equiv\True$
\end{definition}



\subsection{Solving}

\fixme{Something about SMT.}

% !TEX root=../main.tex



\section{Properties}
\label{sec:properties}

In this section, we describe what it means for the symbolic execution semantics to be correct.
We demonstrate it to be sound and complete with respect to the original semantics of \TOPHAT.
% Furthermore, we demonstrate that the pruning performed in Section~\ref{subsec:driving} is safe,
% and only removes branches that are either unfeasible, or infinite.

% \subsection{Removal of Not SAT}
%
% Subsection~\ref{subsec:driving} describes two criteria that are used to prune away branches of the symbolic execution.
% The first removes branches that result in a predicate that is not satisfiable.
% As stated in Definition~\ref{def:Sat}, this means that there is no substitution for the symbols in the predicate,
% that will make the predicate true.
%
% In other words, these symbolic branches have no counterpart in the concrete semantics, and are therefore invalid.
% This property is described in the following theorem.
%
% \begin{theorem}[Not sat is safe to remove]
%   \label{thm:notSat}
%
% For all tasks $t$ and states $\sigma$ such that
% $t,\sigma\drive{}\overline{t',\sigma',i,\phi}$\quad
% $t',\sigma'\drive{}\overline{t'',\sigma'',i',\phi'}$
%
% we have that $\neg\Sat(\phi\land\phi')$ implies
% that there is no mapping $M=[s_0\mapsto c_0,\cdots,s_n\mapsto c_n]$ such that
% $t,M\sigma\xrightarrow[]{Mi}\bar{t'},\bar{\sigma'}\xrightarrow[]{Mi'}\bar{t''},\bar{\sigma''}$
%
% \end{theorem}
%
% The proof for this theorem is listed in the appendix.
%
% It is important to note that this holds for all future states of the unsatisfiable symbolic executions too.
% This can be concluded from the way in which the predicates are constructed, new predicates are put in conjunction with the old one.
%
% %%%%%%%%% STUCK properties  %%%%%%%%%%%%%%%%%%%%%%%%%%%%%%%%
%
% \subsection{When are branches stuck?}
%
% The second pruning criteria deals with terms that allow for infinite input streams,
% but that will not result in a value.
%
% Observe the following small program. $\Edit 1 \Next \lambda x . \Edit x + 1$.
%
% The symbolic execution will tell us that we can do one of two things;
% send a new symbolic value to the editor, or we can send $\Continue$,
% to proceed to the right hand side of the expression.
%
% When we elect the first, sending a new symbolic value, we can then again send yet another
% symbolic value to the editor. And then again, and again and again.
%
% No extra information is gained, $s_0$ and $s_1$ are erased from the program, and are not contained in the state or
% predicate.
%
% \begin{lemma}[Stuck really is stuck]
% For all tasks $t$ and states $\sigma$ such that $t,\sigma \handle{i} t',\sigma',\phi$,
% then $t=t'$ and $\phi=\phi'$ implies that there is no step $t',\sigma'\handle{i'} t'',\sigma'',\phi'$ such that either $t'\neq t''$ or $\phi\neq\phi'$.
% \label{lemma:stuck}
% \end{lemma}
% \fixme{$\phi'$ is not in scope on the left side of the implication.}
% \\
% \fixme{this cannot be proven untill the drive function is fixed}

In order to relate the two semantics, we use the of concrete inputs listed in Figure~\ref{fig:inputsConcrete}.

\begin{figure}[h]
  \usemacro{G-CInputs-Compact}
  \caption{Syntax of concrete inputs}
  \label{fig:inputsConcrete}
\end{figure}


\subsection{Soundness}
\label{sec:soundess}




%For completeness and soundness we define what it means to completely evaluate a task, given a list of concrete inputs.

% \begin{function}
%   \signature{\mathit{eval} :: \Task \times [\textrm{Concrete inputs}] \times \mathrm{State} \rightarrow \Task \times \mathrm{State}} \\
%   \mathit{eval}\ t\ (j:js)\ \sigma = \begin{array}{ll}
%                               (t',\sigma')      & \Value(t',\sigma') \equiv v \\
%                               \mathit{eval}\ t'\ \sigma' & \Value(t',\sigma') \equiv \bot
%                                   \end{array}
%                               \textrm{with } t,s\xrightarrow[]{j} t',\sigma'
% \end{function}

% \begin{theorem}[Soundness of symbolic execution]
% \label{thm:sound}
%
% For all tasks $t$, mappings $M=[s_0\mapsto c_0,\cdots,s_n\mapsto c_n]$,
% for all elements $(t',I,\sigma,\phi)\in \mathit{drive}\ t\ [\ ]\ \emptyset\ \True$,
% $M \phi\equiv\True$ implies
% $eval\ t\ (M I)\ \emptyset=(t_n,\sigma_n)$,  $t_n=M t'$ and $\sigma_n=M \sigma$.
% \end{theorem}
% \fixme{this cannot be proven untill the drive function is fixed}

In order to validate the symbolic execution semantics, we want to show that for every symbolic execution, there exists a corresponding concrete one.
This soundness property is expressed by Theorem~\ref{thm:sounddrive}.

\begin{theorem}[Soundness of driving]
  \label{thm:sounddrive}

  For all tasks $t$, states $\sigma$ and mappings $M=[s_0\mapsto c_0,\cdots,s_n\mapsto c_n]$,
  such that $t,\sigma\drive{} \overline{t',\sigma',i,\phi}$
  we have $M\phi$ implies
  $t,\sigma M \xRightarrow[]{M i} t'',\sigma''$, $t'M \equiv t''$ and $\sigma' M \equiv \sigma''$.
\end{theorem}

The proof for this lemma is rather straightforward.
Since the driving semantics makes use of the handling and the normalisation semantics, we require two lemmas.
One showing that the handling semantics is sound, Lemma~\ref{lem:soundhandle}, and one showing that the normalisation semantics is sound, Lemma~\ref{lem:soundnorm}.

\begin{lemma}[Soundness of handling]
  \label{lem:soundhandle}

  For all tasks $t$, states $\sigma$ and mappings $M = [s_0\mapsto c_0,\cdots,s_n\mapsto c_n]$,
  such that $t,\sigma\handle{i} \overline{t',\sigma',\phi}$,
  we have $\phi M = \True$ implies
  $t,M \sigma \xrightarrow[]{M i} t'',\sigma''$, $t'M \equiv t'' $ and $\sigma' M \equiv \sigma''$
\end{lemma}

Lemma~\ref{lem:soundhandle} is proven by induction over $t$.
The full proof is listed in the appendix.

\begin{lemma}[Soundness of normalisation]
  \label{lem:soundnorm}

  For all expressions $e$, states $\sigma$ and mappings $M=[s_0\mapsto c_0,\cdots,s_n\mapsto c_n]$,
  such that $e,\sigma\normalise \overline{t,\sigma',\phi}$,
  we have $\phi M =\True$ implies
  $e,M \sigma \bar{\normalise}t',\sigma''$, $t M \equiv t'$ and $\sigma' M \equiv \sigma''$.

\end{lemma}

Since Lemma~\ref{lem:soundnorm} makes use of both the striding and the evaluation semantics,
we are required to show soundness for those too.

\begin{lemma}[Soundness of striding]
  \label{lem:soundstride}

  For all tasks $t$, states $\sigma$ and mappings $M=[s_0\mapsto c_0,\cdots,s_n\mapsto c_n]$,
  such that $t,\sigma\stride \overline{t',\sigma',\phi}$,
  $M \phi= \True$ implies
  $t,M\sigma \bar{\stride}\bar{t'},\bar{\sigma'}$, $M t'\ \equiv \bar{t'} \land M\sigma' \equiv \bar{\sigma'}$.

\end{lemma}

\begin{lemma}[Soundness of evaluation]
  \label{lem:soundeval}

  For all expressions $e$, states $\sigma$ and mappings $M=[s_0\mapsto c_0,\cdots,s_n\mapsto c_n]$,
  such that $e,\sigma\eval \overline{v,\sigma',\phi}$,
  we have that $M\phi \implies t,M\sigma \bar{\eval}\bar{v},\bar{\sigma'} \land Mv \equiv \bar{v} \wedge M\sigma' \equiv \bar{\sigma'}$.

\end{lemma}

The full proof of \cref{lem:soundnorm,lem:soundstride,lem:soundeval} are listed in the appendix.





\subsection{Completeness}

We also want to show that for every concrete execution, a symbolic one exists.
This completeness property is listed in Theorem~\ref{thm:completeDrive}.

% \begin{theorem}[Completeness of symbolic execution]
% For all tasks $t$ and input lists $[j_0,\cdots,j_n]$ such that $\mathit{eval}\ t\ [j_0,\cdots,j_n]\ \emptyset = (t',\sigma')$,
% then there exists an element $(t'',I,\sigma'',\phi'')\in \mathit{drive}\ t\ [\ ]\ \emptyset\ \True$ and mapping $[s_0\mapsto c_0,\cdots,s_n\mapsto c_n]$ such that
% $I[s_0\mapsto c_0,\cdots,s_n\mapsto c_n]=[j_0,\cdots,j_n]$ and $t'=t''[s_0\mapsto c_0,\cdots,s_n\mapsto c_n]$.
%   \label{thm:complete}
% \end{theorem}
%
% \fixme{Proof of thm 5.2 is required for this thm}

\begin{theorem}[Completeness of driving]
  \label{thm:completeDrive}
  $\forall t,\sigma,j$ such that $t,\sigma \xRightarrow[]{j} t',\sigma'$
  there exists an $i\sim j$ such that $t,\sigma''\drive{i} t'',\sigma''',\phi$.
\end{theorem}

Where with $i\sim j$ we mean that the symbolic input $i$ follows the same direction as the concrete input does.
This relation is properly defined below.

\begin{definition}[Input simulation]
  A symbolic input $i$ simulates a concrete input $j$ denoted as $i\sim j$ in the following cases.\\
  $s\sim a$, where $s$ is a symbol and $a$ a concrete action.\\
  $i\sim j\implies \First i \sim \First j$\\
  $i\sim j\implies \Second i \sim \Second j$
\end{definition}

The proof of Theorem~\ref{thm:completeDrive} is rather simple.
We show that handling is complete, Lemma~\ref{lem:completeHandle},
and that the subsequent normalisation is complete too, Lemma~\ref{lem:completeNormalise}.


\begin{lemma}[Completeness of handling]
  \label{lem:completeHandle}
  $\forall t,\sigma,j$ such that $t,\sigma \xrightarrow[]{j} t',\sigma'$
  there exists an $i\sim j$ such that $t,\sigma''\handle{i} t'',\sigma''',\phi$.
\end{lemma}

Lemma~\ref{lem:completeHandle} is shown by induction over $t$.
In essence, we merely need to show that every concrete execution also is a symbolic one.
The only change needed to convert from concrete to symbolic, is the adaption of the input.

Since handling makes use of the normalisation, striding and eventually the evaluation semantics,
we need to prove that they too are complete.
These properties are listed in \cref{lem:completeNormalise,lem:completeStride,lem:completeEval}

\begin{lemma}[Completeness of normalisation]
  \label{lem:completeNormalise}
  $\forall e,\sigma$ such that $e,\sigma \bar{\normalise}t',\sigma'$
  there exists a symbolic execution $t,\sigma''\normalise t'',\sigma''',\phi$.
\end{lemma}


\begin{lemma}[Completeness of striding]
  \label{lem:completeStride}
  $\forall t,\sigma$ such that $t,\sigma \bar{\stride}t',\sigma'$
  there exists a symbolic execution $t,\sigma''\eval t'',\sigma''',\phi$.
\end{lemma}


\begin{lemma}[Completeness of evaluation]
  \label{lem:completeEval}
  $\forall e,\sigma$ such that $t,\sigma \bar{\eval}v,\sigma'$
  there exists a symbolic execution $t,\sigma''\eval v',\sigma''',\phi$.
\end{lemma}


Proof for \cref{lem:completeNormalise,lem:completeStride,lem:completeEval} is trivial,
since every concrete execution in these semantics is also a symbolic one.

% !TEX root=../main.tex



\section{Related work}
\label{sec:relatedwork}

\paragraph{Symbolic execution}
Symbolic execution \cite{King1975,Boyer1975} is typically being applied to imperative programming languages, for example \citet{BucurKC2014} prototype a symbolic execution engine for interpreted imperative languages.
\citet{CadarDE2008} use it to generate test cases for programs that can be compiled to LLVM bytecode.
\citet{JaffarMNS2012} use it for verifying safety properties of C programs.

In recent years it has been used for functional programming languages as well.
To name some examples, there is ongoing work by \citet{HallahanXP2017} and \citet{Xue2019} to implement a symbolic execution engine for Haskell.
\citet{GiantsiosPS2017} use symbolic execution for a mix of concrete and symbolic testing of programs written in a subset of Core Erlang.
Their goal is to find executions that lead to a runtime error, either due to an assertion violation or an unhandled exception.
\citet{ChangKT2018} present a symbolic execution engine for a typed lambda calculus with mutable state where only some language constructs recognize symbolic values.
They claim that their approach is easier to implement than full symbolic execution and simplifies the burden on the solver, while still considering all execution paths.

The difficulty of symbolic execution for functional languages lies in symbolic higher-order values, that is functions as arguments to other functions.
Hallahan et al solve this with a technique called \emph{defunctionalization}, which requires all source code to be present, so that a symbolic function can only be one of the present lambda expressions or function definitions.
Giantosis et al also require all source code to be present, but they only analyze first-order functions.
They can execute higher-order functions, but only with concrete arguments.
Our method also requires closed well-typed terms, so we never execute a higher-order function in isolation.
Furthermore, we currently do not allow functions as task values.
Together, this means that symbolic values can never be functions.



\paragraph{Contracts}
Another method for guaranteeing correctness of functional programs are \emph{contracts}.
Contracts are assertions for higher-order languages.
They are enforced at run time.
Contracts were first presented by \citet{FindlerF2002}, who implement a contract checker for Scheme.
Contracts can be used to specify properties more fine-grained than what a static type system could check.
It is possible, for example, to refine the arguments or return values of functions to numbers in a certain range, to positive numbers or non-empty lists.


\paragraph{Axiomatic program verification}
One of the classical methods of proving partial correctness of programs is Hoare's axiomatic approach \cite{Hoare1969}, which is based on pre- and postconditions.
See \citet{NielsonN1992} for a nice introduction to the topic.
The axiomatic approach is usually applied to imperative programs, requires manually stating loop invariants, and manually carrying out proofs.

Some work has been done to bring the axiomatic method to functional programming.
The current state of SMT solving allows for automated extraction and solving of a large amount of proof obligations.
Notable works in this field are for example the Hoare Type Theory by \citet{NanevskiMB2006}, the Hoare and Dijkstra Monads by \citet{NanevskiMSGB08, SwamyWSCL2013}, or the Hoare logic for the state monad by \citet{Swierstra2009}.

The difference between the work cited here and our work is that the axiomatic method focuses on stateful computations, while we try to incorporate input as well.

% !TEX root=../main.tex

\section{Conclusion}

\label{sec:conclusion}

\subsection{Future work}

%different kinds of analysis

%fitting it to iTasks?

% loops perhaps?

% euality on tasks

%....



%% Acknowledgments
\begin{acks}                            %% acks environment is optional
                                        %% contents suppressed with 'anonymous'
  %% Commands \grantsponsor{<sponsorID>}{<name>}{<url>} and
  %% \grantnum[<url>]{<sponsorID>}{<number>} should be used to
  %% acknowledge financial support and will be used by metadata
  %% extraction tools.
  \small
  % !TEX root=../main.tex

This research is supported by the Dutch Technology Foundation STW, which is part
of the Netherlands Organisation for Scientific Research (NWO), and which is
partly funded by the Ministry of Economic Affairs.

\end{acks}


%% Bibliography
\bibliography{bibliography}


%% Appendix
\pagebreak \appendix

\onecolumn
% % !TEX root=../../main.tex

\onecolumn



\begin{proof}[Proof of Theorem~\ref{thm:notSat}]
  By Definition~\ref{def:Sat}, the premise of Theorem~\ref{thm:notSat}, $\neg\text{SAT }(\phi\land\phi')$,
  means that there does not exist a mapping $M=[s_0\mapsto c_0,\cdots,s_n\mapsto c_n]$ such that $M(\phi\land\phi')\equiv\True$.
  This can be rewritten to for all mappings $M$, $\neg M\phi \lor \neg M\phi'$.
  By Lemma~\ref{lem:notSatDrive}, etc etc etc \fixme{complete this}
\end{proof}

\begin{lemma}[Unsatisfiable symbolic evaluation]
  \label{lem:notSatEval}
  For all expressions $e$ and states $\sigma$ such that $e,\sigma\eval\overline{v,\sigma',\phi}$, if we have $\neg\text{SAT }(\phi)$ then $\neg\exists M$ such that $e,M\sigma\eval\bar{v},\bar{\sigma'}$

\end{lemma}

\begin{proof}[Proof of Lemma~\ref{lem:notSatEval}]

\end{proof}

\begin{lemma}[Unsatisfiable symbolic striding]
  \label{lem:notSatStride}
  For all tasks $t$ and states $\sigma$ such that $t,\sigma\stride\overline{t',\sigma',\phi}$, if we have $\neg\text{SAT }(\phi)$ then $\neg\exists M$ such that $t,M\sigma\bar{\stride}\bar{t'},\bar{\sigma'}$

\end{lemma}



\begin{proof}[Proof of Lemma~\ref{lem:notSatStride}]

\end{proof}

\begin{lemma}[Unsatisfiable symbolic normalisation]
  \label{lem:notSatNorm}
  For all expressions $e$ and states $\sigma$ such that $e,\sigma\normalise\overline{t,\sigma',\phi}$, if we have $\neg\text{SAT }(\phi)$ then $\neg\exists M$ such that $e,M\sigma\bar{\normalise}\bar{t},\bar{\sigma'}$

\end{lemma}



\begin{proof}[Proof of Lemma~\ref{lem:notSatNorm}]

\end{proof}

\begin{lemma}[Unsatisfiable symbolic handling]
  \label{lem:notSatHandle}
  For all tasks $t$ and states $\sigma$ such that $t,\sigma\handle{}\overline{t',\sigma',i,\phi}$, if we have $\neg\text{SAT }(\phi)$ then $\neg\exists M$ such that $t,M\sigma\xrightarrow[]{Mi}\bar{t'},\bar{\sigma'}$

\end{lemma}



\begin{proof}[Proof of Lemma~\ref{lem:notSatHandle}]

\end{proof}

\begin{lemma}[Unsatisfiable symbolic driving]
  \label{lem:notSatDrive}
  For all tasks $t$ and states $\sigma$ such that $t,\sigma\drive{}\overline{t',\sigma',i,\phi}$, if we have $\neg\text{SAT }(\phi)$ then $\neg\exists M$ such that $t,M\sigma\xRightarrow[]{Mi}\bar{t'},\bar{\sigma'}$

\end{lemma}

\begin{proof}[Proof of Lemma~\ref{lem:notSatDrive}]

\end{proof}

% % !TEX root=../../main.tex

\begin{proof}[Proof of Lemma~\ref{lemma:stuck}]
\fixme{we cannot prove this, definition of drive function is incorrect}
\end{proof}

% !TEX root=../../main.tex

\section{Complete symbolic semantics}

\subsection{Symbolic execution}

\begin{gather*}
  \small
  \boxed{\RelationSE} \Break
  \userule{SE-Value}\Quad
  \userule{SE-Pair} \Quad
  \userule{SE-First}\Quad
  \userule{SE-Second}\Break
  \userule{SE-Cons}\Quad
  \userule{SE-Head}\Quad
  \userule{SE-Tail}\Break
  \userule{SE-App} \Break
  \userule{SE-If} \Quad
  \userule{SE-Ref} \Quad
  \userule{SE-Deref} \Break
  \userule{SE-Assign} \Quad
  \userule{SE-Edit} \Quad
  \userule{SE-Enter}\Quad
  \userule{SE-Update}\Break
  \userule{SE-Then}\Quad
  \userule{SE-Next}\Quad
  \userule{SE-And}\Break
  \userule{SE-Or} \Quad
  \userule{SE-Xor}\Quad
  \userule{SE-Fail}
\end{gather*}

\subsection{Symbolic striding semantics}

\begin{gather*}
  \small
  \boxed{\RelationSS} \Break
  \userule{SS-ThenStay} \Quad
  \userule{SS-ThenFail} \Break
  \userule{SS-ThenCont} \Quad
  \userule{SS-OrLeft} \Break
  \userule{SS-OrRight} \Break
  \userule{SS-OrNone} \Break
  \userule{SS-Edit} \Quad
  \userule{SS-Fill} \Quad
  \userule{SS-Update} \Quad
  \userule{SS-Fail} \Break
  \userule{SS-Xor} \Quad
  \userule{SS-Next} \Quad
  \userule{SS-And}
\end{gather*}

\subsection{Symbolic normalisation semantics}

\begin{gather*}
  \small
  \boxed{\RelationSN} \Break
  \userule{SN-Done} \Quad
  \userule{SN-Repeat}
\end{gather*}


\subsection{Symbolic handling semantics}

\begin{gather*}
  \small
  \boxed{\RelationSH} \Break
  \userule{SH-Change} \Quad
  \userule{SH-Fill} \Quad
  \userule{SH-Update}\Break
  \userule{SH-Next} \Quad
  \userule{SH-PassThen} \Break
  \userule{SH-PassNext} \Break
  \userule{SH-PassNext1}\Quad
  \userule{SH-PickLeft} \Break
  \userule{SH-PickRight}\Quad
  \userule{SH-Pick}\Break
  \userule{SH-And}\Quad
  \userule{SH-Or}
\end{gather*}

\subsection{Symbolic driving semantics}

\begin{gather*}
  \small
  \boxed{\RelationSI}\Break
  \userule{SI-Handle}
\end{gather*}

% !TEX root=../../main.tex

\section{\TOPHAT semantics}

\subsection{Typing rules}

  \begin{gather*}
    \boxed{\RelationT} \Break
    \userule{T-Var} \Quad
    \userule{T-Loc} \Quad
    \userule{T-Pair} \Quad
    \userule{T-First} \Quad
    \userule{T-Second} \Break
    \userule{T-ListEmpty} \Quad
    \userule{T-ListCons} \Quad
    \userule{T-ListHead} \Quad
    \userule{T-ListTail}\Break
    \userule{T-Abs} \Quad
    \userule{T-App} \Quad
    \userule{T-If} \Break
    \userule{T-Ref} \Quad
    \userule{T-Deref} \Quad
    \userule{T-Assign}\Quad
    \userule{T-Loc}\Quad
    \userule{T-Edit}\Break
    \userule{T-Enter}\Quad
    \userule{T-Update}\Quad
    \userule{T-Fail}\Quad
    \userule{T-Then}\Quad
    \userule{T-Next}\Break
    \userule{T-And}\Quad
    \userule{T-Or}\Quad
    \userule{T-Xor}
  \end{gather*}

\subsection{Evaluation rules}

  \begin{gather*}
    %\boxed{\RelationE} \Break
    \userule{E-App} \Quad
    \userule{E-IfTrue} \Quad
    \userule{E-Ref} \Break
    \userule{E-IfFalse} \Quad
    \userule{E-Deref} \Quad
    \userule{E-Value} \Quad
  \userule{E-Assign} \Quad
    \userule{E-Pair} \Break
    \userule{E-First} \Quad
    \userule{E-Second}\Quad
    \userule{E-Nil}\Quad
    \userule{E-Cons}\Quad
    \userule{E-Head}\Break
    \userule{E-Tail}\Quad
    \userule{E-Edit} \Quad
    \userule{E-Enter} \Quad
    \userule{E-Update} \Quad
    \userule{E-Then} \Break
    \userule{E-Next} \Quad
    \userule{E-And} \Quad
    \userule{E-Fail} \Quad
    \userule{E-Or} \Quad
    \userule{E-Xor}
  \end{gather*}

\subsection{Striding rules}

\begin{gather*}
  \userule{S-ThenStay} \Quad
  \userule{S-ThenFail} \Break
  \userule{S-ThenCont}\Quad
  \userule{S-OrLeft} \Break
  \userule{S-OrRight} \Break
  \userule{S-OrNone}\Quad
  \userule{S-Edit} \Quad \userule{S-Fill} \Quad \userule{S-Update} \Break
  \userule{S-Fail} \Quad \userule{S-Xor}\Quad
  \userule{S-Next} \Quad
  \userule{S-And}
\end{gather*}

\subsection{Normalisation rules}

\begin{gather*}
  \userule{N-Done} \Quad
  \userule{N-Repeat}
\end{gather*}

\subsection{Handling rules}

  \begin{gather*}
    \userule{H-Change} \Quad
    \userule{H-Fill} \Quad
    \userule{H-Update}\Break
    \userule{H-Next} \Quad
    \userule{H-PickLeft} \Quad
    \userule{H-PickRight}\Quad
    \userule{H-PassThen} \Break
    \userule{H-PassNext} \Quad
    \userule{H-FirstAnd} \Quad \userule{H-SecondAnd} \Quad
    \userule{H-FirstOr}  \Quad \userule{H-SecondOr}
  \end{gather*}


\subsection{Driving semantics}

  \begin{gather*}
    \userule{I-Handle}
  \end{gather*}

% !TEX root=../../main.tex


\begin{proof}[Proof of Theorem~\ref{thm:sound}]
    \fixme{todo!}
\end{proof}

\begin{proof}[Proof of Lemma~\ref{lem:soundeval}]
  We prove Lemma~\ref{lem:soundeval} by induction over $e$.

\case{$e=v$}
  {One rule applies, namely \userule{Sym-E-Value}
  Since this rule does not generate constraints, any $M$ will do.
  Since neither the state, nor the expression is altered by the evaluation rule \userule{E-Value},
  this case holds true trivially.
  }

\case{$e=\tuple{e_1,e_2}$}
  {One rule applies, namely \userule{Sym-E-Pair}\\
  Provided that $M\phi_1\wedge M\phi_2$,
  we need to demonstrate that  \userule{E-Pair} with $\bar{\sigma}=M\sigma$,
  $M\tuple{v_1,v_2} \equiv \tuple{\bar{v_1},\bar{v_2}}$ and $M\sigma''\equiv\bar{\sigma''}$.

  From the induction hypothesis, we obtain the following.\\
  $\forall M_1 .  M_1\phi_1 \implies e_1, M_1\sigma \bar{\eval}\bar{v_1},\bar{\sigma'}\land  M_1v_1\equiv \bar{v_1} \land  M_1\sigma'\equiv\bar{\sigma'}$ and
  $\forall M_2 . M_2\phi_2 \implies e_2,M_2\sigma' \bar{\eval}\bar{v_2},\bar{\sigma''}\land M_2v_2\equiv \bar{v_2} \wedge M_2\sigma'\equiv\bar{\sigma''}$

  Since $M$ satisfies both $\phi_1$ and $\phi_2$,
  and we know that $M\sigma'\equiv \bar{\sigma'}$
  we obtain that $e_1,M\sigma \bar{\eval}\bar{v_1},\bar{\sigma'}$,
  $e_2,M\sigma'\bar{\eval}\bar{v_2},\bar{\sigma''}$, $M\sigma'\equiv \bar{\sigma'}$,
 $M v_1\equiv \bar{v_1}$ and $M v_2 \equiv \bar{v_2}$ and therefore $M\tuple{v_1,v_2} \equiv \tuple{\bar{v_1},\bar{v_2}}$.
  From the IH we directly obtain that $M \sigma'' \equiv\bar{\sigma''}$.
  }

\case{$e=e_1 e_2$}
  {One rule applies, namely \userule{Sym-E-App}\\
  Provided that $M\phi_1 \land M\phi_2\land M\phi_3$,
  we need to demonstrate that
  \userule{E-App} with $\bar{\sigma}=M\sigma$,
   $M v_1 \equiv \bar{v_1}$ and $M\sigma'''\equiv\bar{\sigma'''}$.

  From the induction hypothesis, we obtain the following.\\
  $\forall M_1 . M_1\phi_1 \implies e_1,M_1\sigma\bar{\eval}\lambda x : \tau.\bar{e_1'},\bar{\sigma'}
  \land M_1\lambda x : \tau.e_1' \equiv \lambda x : \tau.\bar{e_1'} \land M_1\sigma'\equiv\bar{\sigma'}$
  and\\
  $\forall M_2 . M_2\phi_2 \implies e_2,M_2\sigma'\bar{\eval}\bar{v_2},\bar{\sigma''}
  \land M_2 v_2 \equiv \bar{v_2} \land M_2\sigma'' \equiv\bar{\sigma''}$
  and\\
  $\forall M_3 . M_3\phi_3 \implies \bar{e_1'}[x\mapsto \bar{v_2}],M_3\sigma''\bar{\eval}\bar{v_1},\bar{\sigma'''}
  \land M_3 v_1\equiv \bar{v_1} \land M_3 \sigma'''\equiv\bar{\sigma'''}$.

  Since $M$ satisfies both $\phi_1$, $\phi_2$ and $\phi_3$, and we know that
  $M\sigma' \equiv\bar{\sigma'}$ and $M\sigma'' \equiv\bar{\sigma''}$,
  we obtain that $e_1,M\sigma \bar{\eval}\lambda x : \tau.\bar{e_1'},\bar{\sigma'}$, $e_2,M\sigma' \bar{\eval}\bar{v_2},\bar{\sigma''}$ and $\bar{e_1'}[x\mapsto \bar{v_2}],M\sigma'' \bar{\eval}\bar{v_1},\bar{\sigma'''}$.
  We can then directly conclude that $M v_1 \equiv \bar{v_1}$ and $M\sigma'''\equiv\bar{\sigma'''}$.
  }

\case{$e=\If{e_1}{e_2}{e_3}$}
   {\fixme{Correct this case}


  % Two rules apply to this case.\\
  % \case{\userule{Sym-E-IfTrue}}
  %   {Provided that $M\phi_1 \land M\phi_2 \land M v_1 M$, and\\
  %   \userule{E-IfTrue} with $\bar{\sigma}=M \sigma$,
  %   we need to demonstrate that $M v_2 \equiv \bar{v_2}$ and $M \sigma''\equiv\bar{\sigma''}$.
  %
  %   From the induction hypothesis, we obtain the following.\\
  %   $\forall M_1 .M_1 \phi_1 \land M_1 v_1 \land e_1,M_1 \sigma \bar{\eval}\True,\bar{\sigma'}\implies M_1 v_1 \equiv \True \land M_1\sigma' \equiv\bar{\sigma'}$ and\\
  %   $\forall M_2 . M_2 \phi_2 \wedge e_2,M_2 \sigma' \bar{\eval}\bar{v_2},\bar{\sigma''}\implies M_2 v_2\equiv \bar{v_2} \land M_2\sigma''\equiv\bar{\sigma''}$
  %
  %   Since $M$ satisfies both $\phi_1$, $v_1$ and $\phi_2$,
  %   and we know from the premise that $e_1,M\sigma \bar{\eval}\True,\bar{\sigma'}$,
  %   and $e_2,M\sigma' \bar{\eval}\bar{v_2},\bar{\sigma''}$ since $M\sigma' \equiv \bar{\sigma}$,
  %   we obtain that $M v_2 \equiv \bar{v_2}$ and $M \sigma'' \equiv\bar{\sigma''}$.
  %
  %   }
  % \case{\userule{Sym-E-IfFalse}}
  %   {Provided that $M\phi_1 \land M\phi_2 \land \lnot M v_1 M$, and\\
  %   \userule{E-IfFalse} with $\bar{\sigma}=M \sigma$,
  %   we need to demonstrate that $M v_2 \equiv \bar{v_2}$ and $M \sigma''\equiv\bar{\sigma''}$.
  %
  %   From the induction hypothesis, we obtain the following.\\
  %   $\forall M_1 .M_1 \phi_1 \land\lnot M_1 v_1 \land e_1,M_1 \sigma \bar{\eval}\False,\bar{\sigma'}\implies M_1 v_1 \equiv \False \land M_1\sigma' \equiv\bar{\sigma'}$ and\\
  %   $\forall M_2 . M_2 \phi_2 \wedge e_3,M_2 \sigma \bar{\eval'}\bar{v_3},\bar{\sigma''}\implies M_2 v_3\equiv \bar{v_3} \land M_2\sigma''\equiv\bar{\sigma''}$
  %
  %   Since $M$ satisfies both $\phi_1$, $\lnot v_1$ and $\phi_2$,
  %   and we know from the premise that $e_1,M\sigma \bar{\eval}\False,\bar{\sigma}$,
  %   and $e_3,M\sigma' \bar{\eval}\bar{v_3},\bar{\sigma'}$ since $M\sigma' \equiv \bar{\sigma'}$,
  %   we obtain that $M v_3 \equiv \bar{v_3}$ and $M \sigma'' \equiv\bar{\sigma''}$.
  %
  %   }
  }

\case{$e=\Ref e$}
  {One rule applies, namely \userule{Sym-E-Ref}\\
  Provided that $M\phi$,
  we need to demonstrate that \userule{E-Ref} with $\bar{\sigma}=M\sigma$,
  $M l \equiv l$ and $M\sigma'[l\mapsto v]\equiv\bar{\sigma'}[l\mapsto\bar{v}]$.

  From the induction hypothesis, we obtain the following.\\
  $\forall M_1 .  M_1\phi \implies e, M_1\sigma \bar{\eval}\bar{v},\bar{\sigma'}\land  M_1v\equiv \bar{v} \wedge  M_1\sigma'\equiv\bar{\sigma'}$.

  We assume that the assignment of location references happens in a deterministic manner, and that we can therefore conclude that exactly the same $l$ is used in both cases. Since $l$ cannot contain any symbols, $M l \equiv l$ holds trivially.

  Since $M$ satisfies $\phi$,
  we obtain that $e,M\sigma \bar{\eval}\bar{v},\bar{\sigma'}$ and $M v\equiv \bar{v}$.
  This, together with $M \sigma' \equiv\bar{\sigma'}$ obtained from the induction hypothesis, we can conclude that $M\sigma'[l\mapsto v]\equiv\bar{\sigma'}[l\mapsto\bar{v}]$.
  }

\case{$e=!e$}
  {One rule applies, namely \userule{Sym-E-Deref}\\
  Provided that $M\phi$, we need to demonstrate that \userule{E-Deref} with $\bar{\sigma}=M\sigma$,
  $M \sigma'(l) \equiv \bar{\sigma'}(l)$ and $M\sigma'\equiv\bar{\sigma'}$.

  From the induction hypothesis, we obtain the following.\\
  $\forall M_1 .  M_1\phi \implies e, M_1\sigma \bar{\eval}l,\bar{\sigma'}\land  M_1l\equiv l \land  M_1\sigma'\equiv\bar{\sigma'}$.

  Note that since $l$ cannot contain any symbols, $M l \equiv l$ holds trivially.

  Since $M$ satisfies $\phi$,
  we immediately obtain $e,M\sigma \bar{\eval}l,\bar{\sigma'}$,
  and $M\sigma'\equiv\bar{\sigma'}$.
}

\case{$e=e_1:=e_2$}
  {
  One rule applies, namely \userule{Sym-E-Assign}\\
  Provided that $M\phi_1\wedge M\phi_2$,
  we need to demonstrate that \userule{E-Assign} with $\bar{\sigma}=M\sigma$,
  $M\unit \equiv \unit$, which holds true trivially,
  and $M\sigma''[l\mapsto v_2]\equiv\bar{\sigma''}[l\mapsto\bar{v_2}]$.

  From the induction hypothesis, we obtain the following.\\
  $\forall M_1 .  M_1\phi_1 \implies e_1, M_1\sigma \bar{\eval}l,\bar{\sigma'}\land  M_1 l\equiv l \land  M_1\sigma'\equiv\bar{\sigma'}$ and\\
  $\forall M_2 . M_2\phi_2 \implies e_2,M_2\sigma' \bar{\eval}\bar{v_2},\bar{\sigma''}\land M_2v_2\equiv \bar{v_2} \land M_2\sigma'\equiv\bar{\sigma''}$

  Since $M$ satisfies both $\phi_1$ and $\phi_2$, and we know that $M\sigma'\equiv \bar{\sigma'}$,
  we obtain that $e_1,M\sigma \bar{\eval}l,\bar{\sigma'}$,
  $e_2,M\sigma'\bar{\eval}\bar{v_2},\bar{\sigma''}$,
  $M l\equiv l$, $M v_2 \equiv \bar{v_2}$ and $M\sigma''\equiv\bar{\sigma''}$ and therefore $M\sigma''[l\mapsto v_2]\equiv\bar{\sigma''}[l\mapsto\bar{v_2}]$.
  }

\case{$e=\Edit e$}
  {One rule applies, namely \userule{Sym-E-Edit}\\
  Provided that $M\phi$,
  we need to demonstrate that \userule{E-Edit} with $\bar{\sigma}=M\sigma$,
  $M \Edit v \equiv \Edit \bar{v}$ and $M\sigma'\equiv\bar{\sigma'}$.

  From the induction hypothesis, we obtain the following.\\
  $\forall M_1 .  M_1\phi \implies e, M_1\sigma \bar{\eval}\bar{v},\bar{\sigma'}\land  M_1v\equiv \bar{v} \land  M_1\sigma'\equiv\bar{\sigma'}$.

  Since $M$ satisfies $\phi$,
  we obtain that $e,M\sigma \bar{\eval}\bar{v},\bar{\sigma'}$,
  $M \Edit v\equiv \Edit \bar{v}$.
  We can furthermore directly conclude that $\sigma' M\equiv\bar{\sigma'}$.

  }

\case{$e=\Enter \tau$}
  {
  One rule applies, namely \userule{Sym-E-Enter}\\
  Provided that $M\phi$, we need to demonstrate that \userule{E-Enter} with $\bar{\sigma}=M\sigma$,
  $M \Enter \tau \equiv \Enter \tau$, which holds trivially since types do not hold symbols,
  and $M\sigma\equiv\bar{\sigma}$, which also hols trivially from the premise.
  }

\case{$e=\Update e$}
  {One rule applies, namely \userule{Sym-E-Update}\\
  Provided that $M\phi$, we need to demonstrate that \userule{E-Update} with $\bar{\sigma}=M\sigma$,
  $M \Update l \equiv \Update l$ and $M\sigma'\equiv\bar{\sigma'}$.

  From the induction hypothesis, we obtain the following.\\
  $\forall M_1 .  M_1\phi \implies e, M_1\sigma \bar{\eval}l,\bar{\sigma'}\land  M_1 l\equiv l \land  M_1\sigma'\equiv\bar{\sigma'}$.

  Since $M$ satisfies $\phi$,
  we obtain that $e,M\sigma \bar{\eval}l,\bar{\sigma'}$,
  and $M \Update l\equiv \Update l$.
  We can furthermore directly conclude that $M \sigma' \equiv\bar{\sigma'}$.

  }

\case{$e=e_1\Then e_2$}
  {One rule applies, namely \userule{Sym-E-Then}\\
  Provided that $M\phi$, we need to demonstrate that \userule{E-Then} with $\bar{\sigma}=M\sigma$,
  $M t_1\Then e_2 \equiv \bar{t_1}\Then e_2$ and $M\sigma'\equiv\bar{\sigma'}$.

  From the induction hypothesis, we obtain the following.\\
  $\forall M_1 .  M_1\phi \implies e, M_1\sigma \bar{\eval}\bar{t_1},\bar{\sigma'}\and  M_1 t_1\equiv \bar{t_1} \land  M_1\sigma'\equiv\bar{\sigma'}$.

  Since $M$ satisfies $\phi$,
  we obtain that $e,M\sigma \bar{\eval}\bar{t_1},\bar{\sigma'}$
  and $M t_1\Then e_2 \equiv \bar{t_1}\Then e_2$.
  We can furthermore directly conclude that $M \sigma' \equiv\bar{\sigma'}$.

  }

\case{$e=e_1\Next e_2$}
  {One rule applies, namely \userule{Sym-E-Next}\\
  Provided that $M\phi$, we need to demonstrate that \userule{E-Next} with $\bar{\sigma}=M\sigma$,
  $M t_1\Next e_2 \equiv \bar{t_1}\Then e_2$ and $M\sigma'\equiv\bar{\sigma'}$.

  From the induction hypothesis, we obtain the following.\\
  $\forall M_1 .  M_1\phi \implies e, M_1\sigma \bar{\eval}\bar{t_1},\bar{\sigma'}\land  M_1 t_1\equiv \bar{t_1} \land  M_1\sigma'\equiv\bar{\sigma'}$.

  Since $M$ satisfies $\phi$,
  we obtain that $e,M\sigma \bar{\eval}\bar{t_1},\bar{\sigma'}$
  and $M t_1\Next e_2 \equiv \bar{t_1}\Next e_2$.
  We can furthermore directly conclude that $M \sigma' \equiv\bar{\sigma'}$.

  }

\case{$e=e_1\Or e_2$}
  {One rule applies, namely \userule{Sym-E-Or}\\
  Provided that $M\phi_1\wedge M\phi_2$, we need to demonstrate that \userule{E-Or} with $\bar{\sigma}=M\sigma$, $M t_1\Or t_2 \equiv \bar{t_1}\Or\bar{t_2}$ and $M\sigma''\equiv\bar{\sigma''}$.

  From the induction hypothesis, we obtain the following.\\
  $\forall M_1 .  M_1\phi_1 \implies e_1, M_1\sigma \bar{\eval}\bar{t_1},\bar{\sigma'}\land  M_1 t_1\equiv \bar{t_1} \land  M_1\sigma'\equiv\bar{\sigma'}$ and\\
  $\forall M_2 . M_2\phi_2 \implies e_2,M_2\sigma \bar{\eval}\bar{t_1},\bar{\sigma'}\land M_2t_2\equiv \bar{t_2} \land M_2\sigma'\equiv\bar{\sigma'}$

  Since $M$ satisfies both $\phi_1$ and $\phi_2$, and we know that $M\sigma'\equiv \bar{\sigma'}$,
  we obtain that $e_1,M\sigma \bar{\eval}\bar{t_1},\bar{\sigma'}$,
  $e_2,M\sigma'\bar{\eval}\bar{t_2},\bar{\sigma}$, $M t_1\equiv \bar{t_1}$ and $M t_2 \equiv \bar{t_2}$ and therefore $M t_1\Or t_2 \equiv \bar{t_1}\Or\bar{t_2}$.
  From the IH we directly obtain that $M \sigma'' \equiv\bar{\sigma''}$.

  }

\case{$e=e_1\Xor e_2$}
  {  One rule applies, namely \userule{Sym-E-Xor}\\
    Provided that $M\phi$, we need to demonstrate that \userule{E-Xor} with $\bar{\sigma}=M\sigma$,
    $M e_1\Xor e_2 \equiv \bar{e_1}\Xor\bar{e_2}$,
    which holds trivially,
    and $M\sigma\equiv\bar{\sigma}$, which also holds trivially from the premise.

  }

\case{$e=\Fail$}
  {  One rule applies, namely \userule{Sym-E-Fail}\\
    Provided that $M\phi$, we need to demonstrate that \userule{E-Fail} with $\bar{\sigma}=M\sigma$,
    $M \Fail \equiv \Fail$, which holds trivially since fail do not hold symbols,
    and $M\sigma\equiv\bar{\sigma}$, which also hols trivially from the premise.

  }
\end{proof}



\begin{proof}[Proof of Lemma~\ref{lem:soundstride}]
  We prove Lemma~\ref{lem:soundstride} by induction over $t$.

\case{$t=t_1\Then e_2$}
  {
  Three rules apply.\\
  \case{\userule{Sym-S-ThenStay}}
    {Provided that $M\phi\equiv\True$
    we need to demonstrate that
    \userule{S-ThenStay} with $\bar{\sigma}=M\sigma$,
    $M t_1'\Then e_2 \equiv \bar{t_1'}\Then e_2 $ and $ M\sigma'\equiv \bar{\sigma'}$.

    From the induction hypothesis, we obtain the following.\\
    $\forall M_1 . M_1 \phi \implies t_1,M_1\sigma \bar{\stride} \bar{t_1'},\bar{\sigma'}\land M_1 t_1'\equiv\bar{t_1'}\land M_1\sigma' \equiv \bar{\sigma'}$.

    Since $M$ satisfies $\phi$,
    we know that
    $t_1,M\sigma \bar{\stride} \bar{t_1'},\bar{\sigma'}$
    and $M t_1'\equiv\bar{t_1'}$,
    and therefore also $M t_1'\Then e_2 \equiv \bar{t_1'}\Then e_2$,
    and from the induction hypothesis, we directly obtain  $M\sigma'\equiv \bar{\sigma'}$.
    }
  \case{\userule{Sym-S-ThenFail}}
    {Provided that $M\phi\equiv\True$
    we need to demonstrate that
    \userule{S-ThenFail} with $\bar{\sigma}=M\sigma$,
    $M t_1'\Then e_2 \equiv \bar{t_1'}\Then e_2$ and $M\sigma'\equiv \bar{\sigma'}$.

    From the induction hypothesis, we obtain the following.\\
    $\forall M_1 . M_1 \phi \implies t_1,M_1\sigma \bar{\stride} \bar{t_1'},\bar{\sigma'}\land M_1 t_1'\equiv\bar{t_1'}\land M_1\sigma' \equiv \bar{\sigma'}$.

    Since $M$ satisfies $\phi$,
    we know that
    $t_1,M\sigma \bar{\stride} \bar{t_1'},\bar{\sigma'}$
    and $M t_1'\equiv\bar{t_1'}$,
    and therefore also $M t_1'\Then e_2 \equiv \bar{t_1'}\Then e_2$,
    and from the induction hypothesis, we directly obtain  $M\sigma'\equiv \bar{\sigma'}$.
    }
  \case{\userule{Sym-S-ThenCont}}
    {Provided that $M\phi_1\land M\phi_2$
    we need to demonstrate that
    \userule{S-ThenCont} with $\bar{\sigma}=M\sigma$,
    $M t_2 \equiv \bar{t_2}\Then e_2$ and $M\sigma''\equiv \bar{\sigma''}$.

    From the induction hypothesis, we obtain the following.\\
    $\forall M_1 . M_1 \phi_1 \implies t_1,M_1\sigma \bar{\stride} \bar{t_1'},\bar{\sigma'}\implies M_1 t_1'\equiv\bar{t_1'}\land M_1\sigma' \equiv \bar{\sigma'}$.\\
    From Lemma~\ref{lem:soundeval} we know that\\
    $\forall M_2 . M_2 \phi_2 \implies e_2\bar{v_1}M_2\sigma'\bar{\eval}\bar{t_2},\bar{\sigma''}\and M_2 t_2\equiv \bar{t_2}\land M_2\sigma''\equiv \bar{\sigma''}$.

    Since $M$ satisfies both $\phi_1$ and $\phi_2$,
    we know that
    $t_1,M\sigma \bar{\stride} \bar{t_1'},\bar{\sigma'}$ and $e_2\bar{v_1}M\sigma'\bar{\eval}\bar{t_2},\bar{\sigma''}$,
    $M t_2\equiv\bar{t_2}$,
    and from the induction hypothesis, we directly obtain  $M\sigma''\equiv \bar{\sigma''}$.

    }
  }

\case{$t=t_1\Or t_2$}
  {
  Three rules apply.\\
  \case{\userule{Sym-S-OrLeft}}
    {Provided that $M\phi\equiv\True$
    we need to demonstrate that
    \userule{S-OrLeft} with $\bar{\sigma}=M\sigma$,
    $M t_1'\equiv \bar{t_1'}$ and $M\sigma'\equiv \bar{\sigma'}$.

    From the induction hypothesis, we obtain the following.\\
    $\forall M_1 . M_1 \phi \implies t_1,M_1\sigma \bar{\stride} \bar{t_1'},\bar{\sigma'}\and M_1 t_1'\equiv\bar{t_1'}\land M_1\sigma' \equiv \bar{\sigma'}$.

    Since $M$ satisfies $\phi$, we know that $t_1,M\sigma \bar{\stride} \bar{t_1'},\bar{\sigma'}$,
    $M t_1'\equiv\bar{t_1'}$ and $M\sigma'\equiv \bar{\sigma'}$.

    }
  \case{\userule{Sym-S-OrRight}}
    {Provided that $M\phi_1\land M\phi_2$
    we need to deomstrate that
    \userule{S-OrRight} with $\bar{\sigma}=M\sigma$,
    $M t_2'\equiv \bar{t_2'}$ and $M\sigma''\equiv \bar{\sigma''}$.

    From the induction hypothesis, we obtain the following.\\
    $\forall M_1 . M_1 \phi_1 \implies t_1,M_1\sigma \bar{\stride} \bar{t_1'},\bar{\sigma'}\land M_1 t_1'\equiv\bar{t_1'}\land M_1\sigma' \equiv \bar{\sigma'}$ and\\
    $\forall M_2 . M_2 \phi_2 \implies t_2,M_2\sigma' \bar{\stride} \bar{t_2'},\bar{\sigma''}\land M_2 t_2'\equiv\bar{t_2'}\land M_2\sigma'' \equiv \bar{\sigma''}$.

    Since $M$ satisfies both $\phi_1$ and $\phi_2$,
    we know that
    $t_1,M\sigma \bar{\stride} \bar{t_1'},\bar{\sigma'}$ and $t_2,M\sigma'\bar{\stride}\bar{t_2'},\bar{\sigma''}$,
    $M t_2'\equiv\bar{t_2'}$ and $M\sigma''\equiv \bar{\sigma''}$.

    }
  \case{\userule{Sym-S-OrNone}}
    {Provided that $M\phi_1\land M\phi_2$
    we need to demonstrate that
    \userule{S-OrNone} with $\bar{\sigma}=M\sigma$,
    $M t_1'\Or t_2'\equiv \bar{t_1'}\Or\bar{t_2'}$ and $M\sigma''\equiv \bar{\sigma''}$.

    From the induction hypothesis, we obtain the following.\\
    $\forall M_1 . M_1 \phi_1 \implies t_1,M_1\sigma \bar{\stride} \bar{t_1'},\bar{\sigma'}\land M_1 t_1'\equiv\bar{t_1'}\land M_1\sigma' \equiv \bar{\sigma'}$ and\\
    $\forall M_2 . M_2 \phi_2 \implies t_2,M_2\sigma' \bar{\stride} \bar{t_2'},\bar{\sigma''}\land M_2 t_2'\equiv\bar{t_2'}\land M_2\sigma'' \equiv \bar{\sigma''}$.

    Since $M$ satisfies both $\phi_1$ and $\phi_2$,
    we know that $t_1,M\sigma \bar{\stride} \bar{t_1'},\bar{\sigma'}$ and $t_2,M\sigma'\bar{\stride}\bar{t_2'},\bar{\sigma''}$,
    $M t_1'\Or t_2'\equiv \bar{t_1'}\Or\bar{t_2'}$ and $M\sigma''\equiv \bar{\sigma''}$.

    }
  }

\case{$t=\Edit v$}
  {One rule applies, namely \userule{Sym-S-Edit}\\
  Provided that $M\True$, we need to demonstrate that \userule{S-Edit} with $\bar{\sigma}=M\sigma$,
  $M\Edit v \equiv \Edit \bar{v}$ and $M\sigma \equiv \bar{\sigma}$.
  This holds trivially.

  }

\case{$t=\Enter \tau$}
  {One rule applies, namely \userule{Sym-S-Fill}\\
  Provided that $M\True$, we need to demonstrate \userule{S-Fill} with $\bar{\sigma}=M\sigma$,
  $M\Enter \tau \equiv \Enter \tau$ and $M\sigma \equiv \bar{\sigma}$.
  This holds trivially.
  }

\case{$t=\Update l$}
  {One rule applies, namely \userule{Sym-S-Update}\\
  Provided that $M\True$, we need to demonstrate \userule{S-Update} with $\bar{\sigma}=M\sigma$,
  $M\Update l \equiv \Update l$ and $M\sigma \equiv \bar{\sigma}$.
  This holds trivially.
  }

\case{$t=\Fail$}
  {One rule applies, namely \userule{Sym-S-Fail}\\
  Provided that $M\True$, we need to demonstrate that \userule{S-Fail} with $\bar{\sigma}=M\sigma$,
  $M\Fail \equiv \Fail$ and $M\sigma \equiv \bar{\sigma}$.
  This holds trivially.
  }

\case{$t=e_1\Xor e_2$}
  {One rule applies, namely \userule{Sym-S-Xor}\\
  Provided that $M\True$, we need to demonstrate that \userule{S-Xor} with $\bar{\sigma}=M\sigma$,
  $M e_1\Xor e_2 \equiv e_1\Xor e_2$ and $M\sigma \equiv \bar{\sigma}$.
  This holds trivially.
  }

\case{$t=t_1\Next e_2$}
  {One rule applies, namely \userule{Sym-S-Next}\\
  Provided that $M\phi$,
  we need to demonstrate that \userule{S-Xor} with $\bar{\sigma}=M\sigma$,
  $M t_1'\Next e_2\equiv \bar{t_1'}\Next e_2$ and $M \sigma'\equiv\bar{\sigma'}$.

  From the induction hypothesis, we obtain the following.\\
  $\forall M_1 . M_1 \phi \implies t_1,M_1\sigma \bar{\stride} \bar{t_1'},\bar{\sigma'}\land M_1 t_1'\equiv\bar{t_1'}\land M_1\sigma' \equiv \bar{\sigma'}$.

  Since $M$ satisfies $\phi$, we directly obtain $t_1,M_1\sigma \bar{\stride} \bar{t_1'},\bar{\sigma'}$,
  $M t_1'\Next e_2\equiv \bar{t_1'}\Next e_2$ and $M \sigma'\equiv\bar{\sigma'}$.

  }

\case{$t=t_1\And t_2$}
  {One rule applies, namely \userule{Sym-S-And}\\
  Provided that $M\phi_1\land M\phi_2$
  we need to demonstrate $\userule{S-And}$ with $\bar{\sigma}=M\sigma$,
  $M t_1'\And t_2'\equiv \bar{t_1'}\And\bar{t_2'}$ and $M\sigma''\equiv \bar{\sigma''}$.

  From the induction hypothesis, we obtain the following.\\
  $\forall M_1 . M_1 \phi_1 \implies t_1,M_1\sigma \bar{\stride} \bar{t_1'},\bar{\sigma'}\and M_1 t_1'\equiv\bar{t_1'}\land M_1\sigma' \equiv \bar{\sigma'}$ and\\
  $\forall M_2 . M_2 \phi_2 \implies t_2,M_2\sigma' \bar{\stride} \bar{t_2'},\bar{\sigma''}\and M_2 t_2'\equiv\bar{t_2'}\land M_2\sigma'' \equiv \bar{\sigma''}$.

  Since $M$ satisfies both $\phi_1$ and $\phi_2$,
  we know that $t_1,M\sigma \bar{\stride} \bar{t_1'},\bar{\sigma'}$ and $t_2,M\sigma'\bar{\stride}\bar{t_2'},\bar{\sigma''}$,
  $M t_1'\Or t_2'\equiv \bar{t_1'}\Or\bar{t_2'}$ and $M\sigma''\equiv \bar{\sigma''}$.

  }

\end{proof}



\begin{proof}[Proof of Lemma~\ref{lem:soundnorm}]
  We prove Lemma~\ref{lem:soundnorm} by induction over $e$.

  The base case is when the Sym-N-Done rule applies.\\
  \userule{Sym-N-Done}\\

  Provided that $M\phi_1\land M\phi_2$\\
  we need to demonstrate that
  \userule{N-Done} with $\bar{\sigma}=M\sigma$,
  $M t'\equiv \bar{t'}$ and $M\sigma''\equiv \bar{\sigma''}$.

  By Lemma~\ref{lem:soundeval} and Lemma~\ref{lem:soundstride}, we know that\\
  $\forall M_1. M_1\phi_1 \implies e,M_1\sigma \bar{eval}\bar{t},\bar{\sigma'}\land M_1 t \equiv \bar{t} \land M_1 \sigma'\equiv \bar{\sigma'}$ and\\
  $\forall M_2.M_2\phi_2\implies t,M_2\sigma'\bar{\stride}\bar{t'},\bar{\sigma''}\land M_2 t'\equiv\bar{t'}\land M_2\sigma''\equiv \bar{\sigma''}$.

  We assume $M$ to satisfy both $\phi_1$ and $\phi_2$, we have $e,M\sigma \bar{\eval}\bar{t},\bar{\sigma'}$ since $M\sigma\equiv \bar{\sigma}$.
  We also have $t,M\sigma'\bar{\stride}\bar{t'},\bar{\sigma''}$,
  from wich we can directly conclude what we needed to prove,
  namely $M t'\equiv \bar{t'}\land M\sigma''\equiv \bar{\sigma''}$.

  The only induction step is when\\
  \userule{Sym-N-Repeat} applies.
  In this case, where we have that $M\phi_1\land M\phi_2 \land M\phi_3$,
  we need to demonstrate that
  \userule{N-Repeat} with $\bar{\sigma}=M\sigma$,
  $M t''\equiv \bar{t''}$ and $M\sigma'''\equiv \bar{\sigma'''}$.

  Again by Lemma~\ref{lem:soundeval} and Lemma~\ref{lem:soundstride}, we know that\\
  $\forall M_1. M_1\phi_1 \implies e,M_1\sigma \bar{eval}\bar{t},\bar{\sigma'}\land M_1 t \equiv \bar{t} \land M_1 \sigma'\equiv \bar{\sigma'}$ and\\
  $\forall M_2.M_2\phi_2\implies t,M_2\sigma'\bar{\stride}\bar{t'},\bar{\sigma''}\land M_2 t'\equiv\bar{t'}\land M_2\sigma''\equiv \bar{\sigma''}$.\\
  Furthermore, we know by applying the induction hypothesis that $\forall M_3.M_3\phi_3 \implies t',M_3\sigma''\bar{\normalise} \bar{t''},\bar{\sigma'''}\land M_3 t''\equiv \bar{t''}\land M_3 \sigma'''\equiv \bar{\sigma'''}$.

  Since $M$ satisfies $\phi_1$, $\phi_2$ and $\phi_3$,
  we can conclude that

  $e,M\sigma \bar{\eval}\bar{t},\bar{\sigma'}$, $M t \equiv \bar{t} \land M \sigma'\equiv \bar{\sigma'}$ and $t,M\sigma'\bar{\stride}\bar{t'},\bar{\sigma''}$ and $M t'\equiv\bar{t'}\land M\sigma''\equiv \bar{\sigma''}$.
  This finally gives us $t',M\sigma''\bar{\normalise} \bar{t''},\bar{\sigma'''}$ from which we can conclude that which we needed to prove,
  namely $M t''\equiv \bar{t''}\land M\sigma'''\equiv \bar{\sigma'''}$.
\end{proof}



\begin{proof}[Proof of Lemma~\ref{lem:soundhandle}]
  We prove Lemma~\ref{lem:soundhandle} by induction over $t$.\\

  \case{$t=\Edit v$}
    {One rule applies, namely \userule{Sym-H-Change}\\
    Provided that $M\True$ we need to demonstrate that \userule{H-Change} with $\bar{\sigma}=M\sigma$ and $M s = v'$,
    $M \Edit s\equiv \Edit v'$ and $ M\sigma\equiv \bar{\sigma}$.

    This follows trivially from the premise.

    }

  \case{$t=\Enter \tau$}
  {One rule applies, namely \userule{Sym-H-Fill}\\
  Provided that $M\True$ we need to demonstrate that \userule{H-Fill} with $\bar{\sigma}=M\sigma$ and $M s = v$,
  $M \Edit s \equiv \Edit v$ and $ M\sigma\equiv \bar{\sigma}$.

  This follows trivially from the premise.

  }

  \case{$t=\Update l$}
  {One rule applies, namely \userule{Sym-H-Update}\\
  Provided that $M\True$
  we need to demonstrate that  \userule{H-Update} with $\bar{\sigma}=M\sigma$ and $M s = v$,
  $M \Update l \equiv \Update l$ and $ M\sigma[l\mapsto s]\equiv \bar{\sigma}[l\mapsto v]$.

  \userule{H-Update} with $\bar{\sigma}=M\sigma$ follows trivially.
  $M \Update l \equiv \Update l$ follows trivially, since locations cannot contain symbols. $ M\sigma[l\mapsto s]\equiv \bar{\sigma}[l\mapsto v]$ can be concluded from the fact that $\bar{\sigma}=M\sigma$ and $M s = v$.

  }

  \case{$t=t_1\Next e_2$}
  {One rule applies, namely \userule{Sym-H-Next}\\

  In the case $M\phi_1$, we need to demonstrate that \userule{H-PassNext} with $\bar{\sigma}=M\sigma$ and $j= M i$,
  $M t_1' \Next e_2 \equiv \bar{t_1'}\Next e_2$ and $M\sigma'\equiv\bar{\sigma'}$.

  By the induction hypothesis we obtain the following.\\
  $\forall M_1 . M_1 \phi_1 \implies t_1,M_1\sigma \xrightarrow[]{M_1 i} \bar{t_1'},\bar{\sigma'}\land M_1 t_1'\equiv\bar{t_1'}\land M_1\sigma' \equiv \bar{\sigma'}$

  Since $M$ satisfies $\phi$, we have $t_1,M\sigma \xrightarrow[]{M i} \bar{t_1'},\bar{\sigma'}$, $M\sigma'\equiv\bar{\sigma'}$,
  which we needed to show, as well as $M t_1' \Next e_2 \equiv \bar{t_1'}\Next e_2$ since this can be concluded from $M t_1'\equiv \bar{t_1'}$.

  In the case $M\phi_2$, we need to demonstrate that \userule{H-Next} with $\bar{\sigma}=M\sigma$,
  $M t_2 \equiv \bar{t_2}$ and $M\sigma'\equiv\bar{\sigma'}$.

  From Lemma~\ref{lem:soundnorm} we obtain that $\forall M_1. M_1 \phi \implies e_2 v_1,M\sigma\bar{normalise}\bar{t_2},\bar{\sigma'}\land M t_2\equiv\bar{t_2}\land M \sigma'\equiv\bar{\sigma'}$.

  This gives us exactly what we needed to prove this case.
  }

  \case{$t=t_1\Then e_2$}
  {One rule applies, namely \userule{Sym-H-PassThen}\\
  Provided that $M\phi$ and \userule{H-PassThen} with $\bar{\sigma}=M\sigma$ and $j= M i$,
  we need to demonstrate that $M t_1'\Then e_2\equiv \bar{t_1'}\Then e_2$ and $M\sigma'\equiv\bar{\sigma'}$.

  By the induction hypothesis we obtain the following.\\
  $\forall M_1 . M_1 \phi_1 \land t_1,M_1\sigma \xrightarrow[]{M_1 i} \bar{t_1'},\bar{\sigma'}\implies M_1 t_1'\equiv\bar{t_1'}\land M_1\sigma' \equiv \bar{\sigma'}$

  Since $M$ satisfies $\phi$, and $t_1,M\sigma \xrightarrow[]{M i} \bar{t_1'},\bar{\sigma'}$ we have $M\sigma'\equiv\bar{\sigma'}$,
  which we needed to show, as well as $M t_1' \Then e_2 \equiv \bar{t_1'}\Then e_2$ since this can be concluded from $M t_1'\equiv \bar{t_1'}$.

  }

  \case{$t=e_1\Xor e_2$}
  {
  In this case, three rules apply.\\
    \case{\userule{Sym-H-Pick}}
    {
    \fixme{write this case}
    }
    \case{\userule{Sym-H-PickLeft}}
    {Provided that $M(\phi\wedge s=\Left)$ and \userule{H-PickLeft} with $\bar{\sigma}=M\sigma$,
    we need to demonstrate that $M t_1\equiv \bar{t_1}$ and $M\sigma'\equiv \bar{\sigma'}$.

    From Lemma~\ref{lem:soundnorm} we obtain that $\forall M_1. M_1 \phi \land e_1,M\sigma\bar{\normalise}\bar{t_1},\bar{\sigma'}\implies M t_1\equiv\bar{t_1}\land M \sigma'\equiv\bar{\sigma'}$.

    Since $M$ satisfies $\phi$, and $e_1,M\sigma \bar{\normalise}[] \bar{t_1},\bar{\sigma'}$ we have $M\sigma'\equiv\bar{\sigma'}$,
    which we needed to show, as well as $M t_1 \equiv \bar{t_1}$.

    }
    \case{\userule{Sym-H-PickRight}}
    {Provided that $M(\phi\wedge s=\Right)$ and \userule{H-PickRight} with $\bar{\sigma}=M\sigma$,
    we need to demonstrate that $M t_2\equiv \bar{t_2}$ and $M\sigma'\equiv \bar{\sigma'}$.

    From Lemma~\ref{lem:soundnorm} we obtain that $\forall M_1. M_1 \phi \land e_2,M\sigma\bar{\normalise}\bar{t_2},\bar{\sigma'}\implies M t_2\equiv\bar{t_2}\land M \sigma'\equiv\bar{\sigma'}$.

    Since $M$ satisfies $\phi$, and $e_2,M\sigma \bar{\normalise}[] \bar{t_2},\bar{\sigma'}$ we have $M\sigma'\equiv\bar{\sigma'}$,
    which we needed to show, as well as $M t_2 \equiv \bar{t_2}$.
    }
  }

  \case{$t=t_1\And t_2$}
  {
  In this case, two rules apply.\\
    \case{\userule{Sym-H-FirstAnd}}
    {Provided that $M\phi$ and \userule{H-FirstAnd} with $\bar{\sigma}=M\sigma$,
    we need to demonstrate that $M t_1'\And t_2\equiv \bar{t_1'}\And t_2$ and $M\sigma'\equiv \bar{\sigma'}$.

    By the induction hypothesis we obtain the following.\\
    $\forall M_1 . M_1 \phi_1 \land t_1,M_1\sigma \xrightarrow[]{M_1 i} \bar{t_1'},\bar{\sigma'}\implies M_1 t_1'\equiv\bar{t_1'}\land M_1\sigma' \equiv \bar{\sigma'}$

    Since $M$ satisfies $\phi$, and $t_1,M\sigma\xrightarrow[]{M i} \bar{t_1'},\bar{\sigma'}$ we have $M\sigma'\equiv\bar{\sigma'}$,
    which we needed to show, as well as $M t_1'\And t_2\equiv \bar{t_1'}\And t_2$, which follows from $M t_1' \equiv \bar{t_1'}$.

    }
    \case{\userule{Sym-H-SecondAnd}}
    {Provided that $M\phi$ and \userule{H-SecondAnd} with $\bar{\sigma}=M\sigma$,
    we need to demonstrate that $M t_1\And t_2'\equiv t_1\And \bar{t_2}$ and $M\sigma'\equiv \bar{\sigma'}$.

    By the induction hypothesis we obtain the following.\\
    $\forall M_1 . M_1 \phi_1 \land t_2,M_1\sigma \xrightarrow[]{M_1 i} \bar{t_2'},\bar{\sigma'}\implies M_1 t_2'\equiv\bar{t_2'}\land M_1\sigma' \equiv \bar{\sigma'}$

    Since $M$ satisfies $\phi$, and $t_2,M\sigma\xrightarrow[]{M i} \bar{t_2'},\bar{\sigma'}$ we have $M\sigma'\equiv\bar{\sigma'}$,
    which we needed to show, as well as $M t_1\And t_2'\equiv t_1\And \bar{t_2'}$, which follows from $M t_2' \equiv \bar{t_2'}$.}
  }

  \case{$t=e_1\Or e_2$}
  {\fixme{write the new case here}
  % In this case, two rules apply.\\
  %   \case{\userule{Sym-H-FirstOr}}
  %   {Provided that $M\phi$ and \userule{H-FirstOr} with $\bar{\sigma}=M\sigma$,
  %   we need to demonstrate that $M t_1'\Or t_2\equiv \bar{t_1'}\And t_2$ and $M\sigma'\equiv \bar{\sigma'}$.
  %
  %   By the induction hypothesis we obtain the following.\\
  %   $\forall M_1 . M_1 \phi_1 \land t_1,M_1\sigma \xrightarrow[]{M_1 i} \bar{t_1'},\bar{\sigma'}\implies M_1 t_1'\equiv\bar{t_1'}\land M_1\sigma' \equiv \bar{\sigma'}$
  %
  %   Since $M$ satisfies $\phi$, and $t_1,M\sigma\xrightarrow[]{M i} \bar{t_1'},\bar{\sigma'}$ we have $M\sigma'\equiv\bar{\sigma'}$,
  %   which we needed to show, as well as $M t_1'\Or t_2\equiv \bar{t_1'}\And t_2$, which follows from $M t_1' \equiv \bar{t_1'}$.
  %
  %   }
  %   \case{\userule{Sym-H-SecondOr}}
  %   {Provided that $M\phi$ and \userule{H-SecondOr} with $\bar{\sigma}=M\sigma$,
  %   we need to demonstrate that $M t_1\Or t_2'\equiv t_1\And \bar{t_2}$ and $M\sigma'\equiv \bar{\sigma'}$.
  %
  %   By the induction hypothesis we obtain the following.\\
  %   $\forall M_1 . M_1 \phi_1 \land t_2,M_1\sigma \xrightarrow[]{M_1 i} \bar{t_2'},\bar{\sigma'}\implies M_1 t_2'\equiv\bar{t_2'}\land M_1\sigma' \equiv \bar{\sigma'}$
  %
  %   Since $M$ satisfies $\phi$, and $t_2,M\sigma\xrightarrow[]{M i} \bar{t_2'},\bar{\sigma'}$ we have $M\sigma'\equiv\bar{\sigma'}$,
  %   which we needed to show, as well as $M t_1\Or t_2'\equiv t_1\And \bar{t_2'}$, which follows from $M t_2' \equiv \bar{t_2'}$.}

  }
\end{proof}



\begin{proof}[Proof of Lemma~\ref{lem:sounddrive}]
  We prove Lemma~\ref{lem:sounddrive} as follows.
  There is only one rule that applies, namely \userule{Sym-I-Handle}.

  Provided that $M\phi_1\land\phi_2$ and \userule{I-Handle} with $\bar{\sigma}=M\sigma$ and $j=M i$,
  we need to demonstrate that $M t'' \equiv \bar{t''}$ and $M\sigma''\equiv \bar{\sigma''}$.


  Lemma~\ref{lem:soundhandle} and Lemma~\ref{lem:soundnorm} respectively give us that\\
$\forall M_1 . M_1 \phi_1 \land t_1,M_1\sigma \xrightarrow[]{M_1 i} \bar{t'},\bar{\sigma'}\implies M_1 t'\equiv\bar{t'}\land M_1\sigma' \equiv \bar{\sigma'}$ and \\
$\forall M_2 . M_2 \phi_2 \land t',M_2\sigma' \bar{\normalise} \bar{t''},\bar{\sigma''}\implies M_2 t''\equiv\bar{t''}\land M_2\sigma'' \equiv \bar{\sigma''}$.

Since $M$ satisfies both $\phi_1$ and $\phi_2$, and we have from the premise that
$t_1,M\sigma \xrightarrow[]{M i} \bar{t'},\bar{\sigma'}$ and
$t',M\sigma' \bar{\normalise} \bar{t''},\bar{\sigma''}$,
we obtain exactly what we needed to prove,
namely $M t'' \equiv \bar{t''}$ and $M\sigma''\equiv \bar{\sigma''}$.

\end{proof}

% !TEX root=../../main.tex






\begin{proof}[Proof of Theorem~\ref{thm:complete}]
  \fixme{we cannot prove this, definition of drive function is incorrect}
\end{proof}



\begin{proof}[Proof of Lemma~\ref{lem:completeEval}]
This lemma holds trivially; assume $\sigma''=\sigma$, then we have $t,\sigma\eval v,\sigma',\True$.
In other words; every concrete evaluation is also a valid symbolic execution.
\end{proof}



\begin{proof}[Proof of Lemma~\ref{lem:completeStride}]
This lemma holds trivially; assume $\sigma''=\sigma$, then we have $t,\sigma\stride t',\sigma',\True$.
In other words; every concrete evaluation is also a valid symbolic execution.
\end{proof}



\begin{proof}[Proof of Lemma~\ref{lem:completeNormalise}]
This lemma holds trivially; assume $\sigma''=\sigma$, then we have $t,\sigma\normalise t',\sigma',\True$.
In other words; every concrete evaluation is also a valid symbolic execution.
\end{proof}



\begin{proof}[Proof of Lemma~\ref{lem:completeHandle}]

  We prove Lemma~\ref{lem:completeHandle} by induction over $t$.\\

  \case{$t=\Edit v$}
  {One rule applies in this case, namely \userule{H-Change}\\
  Take $i=s$ where s is a free symbol in $\sigma$ and assume $\sigma''=\sigma$.
  Then by the Sym-H-Change rule,
  we know that a symbolic execution exists.
  When applying the substitution $[s\mapsto v']$,
  we get $\Edit s[s\mapsto v'] = \Edit v'$ and $\sigma[s\mapsto v']=\sigma$ since $s$ is free in $\sigma$.
  \fixme{what about H-Clear?????}
  }

  \case{$t=\Enter \tau$}
  {One rule applies in this case, namely \userule{H-Fill}\\
  Take $i=s$ where s is a free symbol in $\sigma$ and assume $\sigma''=\sigma$.
  Then by the Sym-H-Fill rule,
  we know that a symbolic execution exists.
  When applying the substitution $[s\mapsto v']$,
  we get $\Edit s[s\mapsto v'] = \Edit v'$ and $\sigma[s\mapsto v']=\sigma$ since $s$ is free in $\sigma$. }

  \case{$t=\Update l$}
  {One rule applies in this case, namely \userule{H-Update}\\
  Take $i=s$ where s is a free symbol in $\sigma$ and assume $\sigma''=\sigma$.
  Then by the Sym-H-Update rule,
  we know that a symbolic execution exists.
  When applying the substitution $[s\mapsto v']$,
  we get $\Edit s[s\mapsto v'] = \Edit v'$ and $\sigma[s\mapsto v']=\sigma$ since $s$ is free in $\sigma$. }

  \case{$t=t_1\Next e_2$}
  {Two rules apply in this case.
    \case{\userule{H-Next}}
    {
    Take $i=s$ where s is a free symbol in $\sigma$ and $t$, and assume $\sigma''=\sigma$.
    Then by the Sym-H-Next rule,
    we know that a symbolic execution exists.
    When applying the substitution $[s\mapsto \Continue]$,
    we get $t_2[s\mapsto \Continue] = t_2'$ and $\sigma'[s\mapsto \Continue]=\sigma'$ since $s$ is free in both $\sigma$ and $t$.
    }
    \case{\userule{H-PassNext}}
    {
    Take $i=s$ where s is a free symbol in $\sigma$ and $t$, and assume $\sigma''=\sigma$.
    Then by the Sym-H-PassNext rule,
    we know that a symbolic execution exists.
    When applying the substitution $[s\mapsto j]$, and by application of the induction hypothesis, we obtain the following.\\
    $t_1,\sigma\xrightarrow[]{j}t_1',\sigma'\implies \exists t_1'',\sigma''\handle{i}t_1''',\sigma'''\land t_1'''[s\mapsto j] = t_1'\land \sigma'''[s\mapsto j]=\sigma'$.\\
    From this we obtain $t_1'''\Next e_2[s\mapsto j] = t_1'\Next e_2$ and $\sigma'''[s\mapsto j]=\sigma'$.
    }
  }


  \case{$t=t_1\Then e_2$}{}
  \case{$t=e_1\Xor e_2$}{}
  \case{$t=t_1\Or t_2$}{}
  \case{$t=t_1\And t_2$}{}

\end{proof}



\begin{proof}[Proof of Lemma~\ref{lem:completeDrive}]
  \fixme{todo}
\end{proof}



\end{document}
